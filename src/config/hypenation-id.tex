%--------------------------------------------------------------------%
%
% Hypenation untuk Bahasa Indonesia
%
% @author Petra Barus
%
%--------------------------------------------------------------------%
%
% Secara otomatis LaTeX dapat langsung memenggal kata dalam dokumen,
% tapi sering kali terdapat kesalahan dalam pemenggalan kata. Untuk
% memperbaiki kesalahan pemenggalan kata tertentu, cara pemenggalan
% kata tersebut dapat ditambahkan pada dokumen ini. Pemenggalan
% dilakukan dengan menambahkan karakter '-' pada suku kata yang
% perlu dipisahkan.
%
% Contoh pemenggalan kata 'analisa' dilakukan dengan 'a-na-li-sa'
%
%--------------------------------------------------------------------%

\hypenation {
  % A
  %
  a-na-li-sa
  a-pli-ka-si
  a-lo-ka-si
  an-ta-ra
  
  % B
  %
  be-be-ra-pa
  ber-ge-rak
  be-ri-kut
  ber-ko-mu-ni-ka-si
  
  % C
  %
  ca-ri
  con-strained
  
  % D
  %
  da-e-rah
  di-nya-ta-kan
  de-fi-ni-si
  di-bu-tuh-kan
  di-gu-na-kan
  di-tam-bah-kan-nya
  di-tem-pat-kan
  
  % E
  %
  e-ner-gi
  eks-klu-sif
  
  % F
  %
  fa-si-li-tas
  
  % G
  %
  ga-bung-an
  
  % H
  %
  ha-lang-an
  
  % I
  % 
  i-nduk
  in-for-ma-si
  im-ple-men-tasi
  
  % J
  %
  ka-me-ra
  kua-li-tas
  
  % K
  %
  kom-po-si-si
  
  
  % L
  %
  
  % M
  %
  me-ngu-ra-ngi
  meng-eva-lu-a-si
  me-nge-lo-la
  men-da-lam
  mak-si-mal
  me-nye-le-sai-kan
  me-ngunjungi
  % N
  %
  
  % O
  %
  
  % P
  %
  pro-vi-der
  pe-ru-sa-ha-an
  
  % Q
  %
  
  % R
  %
  re-source
  
  % S
  se-la-in
  %
  
  % T
  % 
  
  % U
  %
  
  % V
  %
  
  % W
  %
  
  % X
  %
  
  % Y
  % 
  
  % Z
  %
  zoo-keeper
}
