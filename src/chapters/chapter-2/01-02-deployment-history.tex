\subsection{Sejarah}

Awal mulanya \textit{deployment} dilakukan dengan membungkus \textit{software} menjadi sebuah \textit{installer}. \textit{Installer} dapat diunduh dan dijalankan oleh \textit{pengguna} untuk membuat \textit{software} tersedia pada perangkat yang digunakan. Jika terdapat \textit{update software}, pengguna harus mengunduh \textit{installer} dengan versi terbaru untuk digunakan \parencite{softwareDeploymentCarzaniga1998characterization}.

Tentunya hal ini kurang efektif karena pengguna harus melakukan pengecekan secara berkala untuk mengetahui apakah terdapat versi terbaru atau tidak. Oleh karena itu, muncul teknologi \textit{deployment} yaitu \textit{package manager}. \textit{Package Manager} dapat digunakan untuk proses \textit{installing, updating, and generally managing software} \parencite{softwareDeploymentCarzaniga1998characterization}. Namun, muncul beberapa masalah seperti \textit{dependency hell}, \textit{dependency conflict} serta \textit{platfrom-constrained} karena \textit{package manager} tidak tersedia di semua \textit{operating system}.

Pada tahun 2014, muncul teknologi baru yaitu \textit{docker, a cloud-centric platform-as-a-service} yang bersifat \textit{}, bertujuan untuk menyelesaikan masalah \textit{dependency conflict dan dependency hell} \parencite{merkel2014docker}. Docker menerapkan \textit{deployment} berbasis \textit{container} yang menggunakan namespace pada linux kernel dan \textit{cgroups} dalam melakukan manajemen \textit{resource} nya sehingga dapat dibuat sebuah sistem yang berjalan khusus untuk sebuah \textit{software} dengan minimal tanpa perlu adanya \textit{conflict} pada \textit{dependency}. Dengan adanya docker, mulai bermunculan banyak teknologi baru yang berfokus pada \textit{container} seperti \textit{podman dan kubernetes} serta \textit{deployment} mulai beralih pada \textit{cloud based deployment}.