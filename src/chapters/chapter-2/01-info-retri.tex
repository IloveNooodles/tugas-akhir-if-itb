\section{\textit{Information Retrieval}}
\textit{Information Retrieval} (IR) adalah proses mencari bahan, biasanya berbentuk dokumen, yang bersifat tak terstruktur, biasanya teks, yang memenuhi kebutuhan informasi dari dalam koleksi besar \parencite{inforetrieval}. IR saat ini sangat sering dilakukan contohnya dalam pencarian informasi berkaitan dengan representasi, penyimpanan, pengaturan, dokumen, halaman web, katalog online, catatan, dan objek multimedia. 

Tujuan utama dari IR adalah penyediaan akses yang efektif dan efisien ke informasi yang dibutuhkan oleh pengguna dalam situasi tertentu. Pengguna IR dapat beragam seperti individu, organisasi, maupun sistem yang membutuhkan informasi yang relevan. IR melibatkan penggunaan teknik-teknik seperti \textit{indexing, searching, retrieval,} dan \textit{evaluation} untuk memastikan informasi yang dihasilkan relevan, tepat, dan sesuai dengan kebutuhan pengguna.

Biasanya IR tersusun oleh beberapa komponen, diantaranya.
\begin{enumerate}
    \item \textit{Indexing}, proses mengubah dokumen menjadi bentuk yang lebih mudah dicari oleh sistem IR.
    \item \textit{Searching}, proses pencarian yang dilakukan oleh pengguna dengan memberikan \textit{query} ke dalam sistem IR dan sistem memberikan keluaran berupa dokumen yang paling relevan.
    \item \textit{Retrieval}, proses sistem IR mengambil dokumen yang relevan dan mengurutkan dokumen berdasarkan kesesuaian dengan \textit{query}.
    \item \textit{Evaluation}, proses penilaian kualitas sistem IR yang biasanya mencakup metrik evaluasi seperti \textit{precision}, \textit{recall}, dan skor F1. 
\end{enumerate}

Dalam melakukan proses-proses tersebut, komponen tersebut menggunakan beberapa teknik, diantaranya.
\begin{enumerate}
    \item \textit{Term Weighting}, teknik pemberian bobot terhadap kata-kata untuk membedakan yang lebih penting dan yang kurang.
    \item \textit{Query Expansion}, teknik menambahkan kata-kata yang relevan pada \textit{query} pengguna untuk meningkatkan akurasi hasil pencarian.
    \item \textit{Clustering}, teknik mengelompokkan dokumen yang mirip untuk memudahkan pengambilan.
\end{enumerate}