
\subsection{China Electronic Toll Colleciton}
China memiliki masyarakat yang sangat banyak dan setiap masyarakat memiliki sekurang-kurangnya satu kendaraan. Seiring dengan bertambahnya jumlah masyarakat di China, jalanan yang ada di China semakin penuh dengan kendaraan yang sehingga menghasilkan kemacetan terutama pada tol bagian pembayaran. Untuk mengatasi masalah ini, China menggunakan \textit{Electronic Toll Colleciton} (ETC) yang di integrasikan dengan setiap kendaraan untuk mempercepat proses ini \parencite{penelitianterkait1}. Arsitektur dapat dilihat pada \ref{fig:china-highways}.

\begin{figure}[ht]
  \centering
  \includegraphics[width=0.8\textwidth]{resources/chapter-2/china-highways.jpg}
  \caption{Implementasi Sistem \textit{ETC} di China \parencite{penelitianterkait1}}
  \label{fig:china-highways}
\end{figure}

China menggunakan KubeEdge untuk melakukan proses \textit{deployment} ETC untuk 100,000 \textit{nodes} dengan total 500,000 aplikasi yang diluncurkan menggunakan KubeEdge tersebar untuk 29 dari 34 provinsi. Proses \textit{deployment} dilakukan secara otomatis dengan membuat sistem \textit{workflow engine} pada Kubernetes sehingga proses \textit{deployment} dapat dilakukan dengan cepat dan mudah. Dengan menggunakan metode ini, sistem pembayaran tol di China menjadi 10x lebih cepat dari sebelumnya \parencite{penelitianterkait1}. Gambar dapat dilihat pada \ref{fig:architecture-china-highways}.

\begin{figure}[ht]
  \includegraphics[width=0.8\textwidth]{resources/chapter-2/arsitektur-china-highways.jpg}
  \caption{Arsitektur Sistem \textit{ETC} di China \parencite{penelitianterkait1}}
  \label{fig:architecture-china-highways}
\end{figure}