\section{Java \textit{Virtual Machine}}
\textit{Java Virtual Machine} atau JVM adalah program yang dapat membaca program java yang telah dikompilasi atau biasa dikenal sebagai \textit{java bytecode} dan menginterpretasikannya menjadi bahasa mesin yang dapat dieksekusi oleh komputer \parencite{java12}. Secara tidak langsung, JVM merupakan komponen utama dalam menjalankan program bahasa Java. Struktur JVM terdiri dari \textit{runtime data structure} di memori dan dua subsistem yang berhubungan langsung dengan \textit{runtime data structure} yaitu \textit{class loader} dan \textit{execution engine}. Semua program yang dibuat dengan Java akan dikompilasi ke \textit{java bytecode} yang nantinya akan diinterpretasikan oleh JVM untuk dieksekusi komputer.

\textit{Elastic Search}, yang terbuat dari bahasa Java, berjalan di atas platform Java Virtual Machine (JVM). JVM sendiri akan bertanggung jawab untuk menjalankan kode Java dan mengelola sumber daya yang dibutuhkan oleh program, seperti memori, prosesor, dan jaringan. Sehingga, JVM memiliki peran penting dalam menjalankan \textit{Elastic Search} dan memastikan kinerjanya baik. JVM menyediakan opsi pengaturan yang membuat pengguna dapat mengontrol alokasi memori dan penggunaan CPU pada aplikasi \textit{Elastic Search}. Dengan mengatur parameter JVM yang tepat, pengguna dapat memperbaiki kinerja \textit{Elastic Search} dan memaksimalkan penggunaan sumber daya sistem.

Parameter yang berhubungan erat dengan alokasi sumber daya pada \textit{Elastic Search} dan mempengaruhi kinerja JVM adalah parameter \textit{heap size} (-Xms dan -Xmx) yang mengatur ukuran \textit{heap memory} yang dialokasikan untuk JVM. \textit{Heap memory} adalah tempat JVM menyimpan objek dan data dari aplikasi Java yang sedang berjalan. Parameter -Xms menentukan ukuran \textit{heap memory} awal yang dialokasikan ketika JVM dimulai, sedangkan parameter -Xmx menentukan batas maksimal ukuran \textit{heap memory} yang dapat dipakai oleh JVM. Selain itu, terdapat parameter lain yang dapat mempengaruhi kinerja JVM dan \textit{Elastic Search}, seperti \textit{thread pool size}, \textit{circuit breaker settings}, dan lain-lain. Parameter-parameter ini dapat diatur melalui file konfigurasi \textit{Elastic Search} atau melalui command line arguments saat menjalankan \textit{Elastic Search}.