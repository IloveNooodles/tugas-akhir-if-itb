\section{\textit{IoT}}

\textit{Internet of Things (IoT)} adalah paradigma teknologi yang mengintegrasikan objek fisik dengan sensor, perangkat keras, dan teknologi jaringan, memungkinkan objek-objek ini untuk mengumpulkan dan bertukar data secara \textit{real-time}. Konsep ini merupakan perwujudan dari evolusi teknologi informasi, di mana objek sehari-hari bertransformasi menjadi entitas cerdas yang mampu berinteraksi dengan lingkungan sekitarnya dan jaringan digital secara lebih luas. \textit{IoT} memperkenalkan kemungkinan baru dalam otomatisasi dan pengambilan keputusan yang berbasis data, membuka jalan bagi inovasi lintas sektor \parencite{madakam2015internet}.

IoT memiliki aplikasi yang luas di berbagai sektor, termasuk industri, kesehatan, transportasi, dan pertanian. Dalam sektor industri, IoT memungkinkan otomatisasi proses dan pemantauan efisiensi mesin secara real-time. Di bidang kesehatan, IoT berkontribusi pada pengembangan perangkat medis yang terhubung untuk pemantauan kesehatan pasien. Dalam transportasi, IoT mendukung pengembangan kendaraan otonom dan sistem manajemen lalu lintas cerdas. Di sektor pertanian, IoT digunakan untuk memantau kondisi tanah dan iklim, membantu petani dalam pengambilan keputusan.

Meskipun IoT menawarkan banyak manfaat, terdapat tantangan yang harus dihadapi, termasuk keamanan dan privasi data, skalabilitas sistem, dan interoperabilitas antar perangkat. Keamanan dan privasi menjadi perhatian utama, mengingat jumlah data yang besar dan sensitif yang dikumpulkan dan dikomunikasikan oleh perangkat IoT. Skalabilitas sistem diperlukan untuk menangani peningkatan jumlah perangkat dan volume data. Interoperabilitas menjamin bahwa perangkat dari berbagai vendor dapat berkomunikasi dan bekerja sama dengan efektif.