\subsubsection{\textit{Vertical Autoscaler}}
\textit{Vertical Autoscaler} (VA) merupakan mekanisme scaling pada Kubernetes yang menyesuaikan skala aplikasi dengan memanipulasi sumber daya CPU dan memori pada level kontainer atau pod.
Dibandingkan dengan HA, VA digunakan untuk otomatisasi pengaturan alokasi sumber daya aplikasi pada level yang lebih rendah dibandingkan dengan \textit{horizontal autoscaler} (jumlah replika pod).
VA berguna untuk mengotomasi reservasi CPU dan memori yang sesuai. 
Penyesuaian ini dapat meningkatkan kluster kubernetes spesifiknya pada utilitasi sumber daya karena dapat mengurangi alokasi CPU dan memori pada sebuah \textit{node} agar bisa dipakai oleh pod lain, \parencite{vpa2}.

Dalam sebuah thesis pascasarjana di KTH Royal Institute of Technology, Swedia, \parencite{predictiveva}, terdapat pembahasan penggunaan VA pada lingkungan Kubernetes dengan mengevaluasi dan membandingkan beberapa strategi VA, seperti \textit{resource-based} VA dan \textit{performance-based} VA. \textit{Resource-based} VA menentukan alokasi sumber daya berdasarkan penggunaan memori dan CPU, sedangkan \textit{performance-based} VA menentukan alokasi sumber daya berdasarkan kinerja aplikasi yang diukur dengan metrik tertentu. Hasil dari studi ini menunjukkan bahwa VA pada Kubernetes dapat membantu meningkatkan efisiensi penggunaan sumber daya dan kinerja aplikasi. Namun, pemilihan strategi VA yang tepat sangat bergantung pada kebutuhan dan karakteristik aplikasi yang akan di-\textit{deploy} pada Kubernetes.

% Selain itu, studi literatur yang dilakukan oleh Lu et al. pada tahun 2020 juga membahas tentang VA pada Kubernetes. Peneliti tersebut mengusulkan sebuah algoritma VA yang disebut sebagai Lightweight Vertical Autoscaler (LVA) yang menggunakan teknik regresi linier untuk memprediksi penggunaan sumber daya aplikasi pada level pod. Hasil pengujian menunjukkan bahwa LVA mampu mengalokasikan sumber daya secara efektif dan menghasilkan performa aplikasi yang lebih baik dibandingkan dengan strategi VA lainnya.

% Secara keseluruhan, studi literatur tersebut menunjukkan bahwa VA merupakan mekanisme scaling yang penting pada Kubernetes dan dapat membantu meningkatkan efisiensi penggunaan sumber daya dan performa aplikasi. Selain itu, ada beberapa strategi VA yang dapat dipilih, tergantung pada karakteristik dan kebutuhan aplikasi yang di-deploy pada Kubernetes.
