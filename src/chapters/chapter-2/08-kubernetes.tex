\section{Kubernetes}

Kubernetes adalah \textit{platform open-source} yang digunakan untuk mengelola, otomatisasi, dan deployment aplikasi yang dikemas dalam container. Platform ini membantu mengelola infrastruktur aplikasi secara efisien dan konsisten. Kubernetes dirancang untuk dapat mengelola, secara otomatis, aplikasi yang berjalan pada lingkungan yang terdistribusi dan skala yang besar. Kubernetes berfokus pada konsep "\textit{container orchestration}", yang berarti kubernetes membantu mengatur dan mengelola container secara otomatis.

Kubernetes bekerja dengan mengolaborasikan komponen yang ada di dalamnya, diantaranya, \textit{node, pod, service,}dan \textit{deployment}. Secara umum, \textit{Node} adalah mesin fisik atau virtual dimana container dijalankan. \textit{Pod} merupakan unit terkecil dalam kubernetes dan berisi satu atau beberapa container yang berjalan bersamaan. \textit{Service} digunakan untuk mengakses aplikasi pada pod, sedangkan \textit{deployment} digunakan untuk melakukan konfigurasi pada \textit{pod} dan \textit{container}.

Kubernetes dilengkapi dengan fitur-fitur seperti \textit{auto scaling, load balancing,} dan \textit{self-recovery}, sehingga aplikasi pada kubernetes selalu tersedia dan terus berjalan bahkan akan mencoba untuk memperbaiki sendiri pada saat terjadi masalah atau kegagalan pada node atau aplikasi. Dengan fitur-fitur tersebut, kubernetes memudahkan dalam mengelola aplikasi yang kompleks pada lingkungan yang skala besar dan terdistribusi.