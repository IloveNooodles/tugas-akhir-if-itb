\section{Kubernetes}

Kubernetes adalah sebuah solusi \textit{open source} yang berguna untuk melakukan orkestrasi berbagai aplikasi yang telah di bungkus dalam suatu lingkungan yang disebut sebagai \textit{container}. Kubernetes berfungsi untuk melakukan \textit{deployment} otomatis, \textit{auto scaling} secara otomatis, serta membuat \textit{network} untuk menghubungkan \textit{container} dengan \textit{container} lainnya. Kubernetes membantu mengelola dan mempercepat proses pengembangan layanan yang rumit dengan skala yang besar \parencite{helmkubernetes}.

Kubernetes memiliki beberapa komponen yang digunakan ketika mengelola layanan. \textit{Node, pod, service} dan \textit{deployment} merupakan empat komponen utama yang sering kali menjadi komponen utama ketika membuat suatu layanan dengan kubernetes. \textit{Node} dapat dianalogikan dengan lingkungan \textit{virutual machine} yang memiliki kemampuan terbatas. \textit{Pod} merupakan suatu tempat untuk menjalankan berbagai macam \textit{container} didalamnya. Satu pod dapat memiliki lebih dari satu \textit{container} untuk dioperasikan. \textit{Service} berfungsi untuk membuka akses eksternal ke dalam \textit{pod} yang secara umum bersifat internal dan tidak dapat diakses dari luar. Terakhir yaitu \textit{deployment} adalah suatu konfigurasi untuk menjalankan layanan yang akan dibuat, konfigurasi deployment juga melingkupi komponen \textit{pod} dan \textit{service}.

Kubernetes memiliki fitur seperti \textit{auto scaling, self-healing, device discovery, load balancing} serta \textit{rollout} dan \textit{rollback}. \textit{Auto scaling} sering digunakan untuk menjaga sistem untuk terus beroperasi dengan cara melakukan replikasi layanan dengan jumlah yang ditentukan. \textit{Self healing} memastikan bahwa layanan yang sedang mengalami kegagalan dapat diperbaiki secara otomatis. \textit{Rollout} dan \textit{Rollback} sering digunakan ketika melakukan proses \textit{deployment} untuk mencegah adanya \textit{downtime} ketika meluncurkan layanan versi baru. Seluruh fitur yang disebutkan bekerja sama untuk membuat kubernetes dapat mengelola aplikasi dalam skala yang masif dan mempertahankan kualitas layanan yang dibuat.

\subsection{\textit{Pod component}}

\textit{Pod} merupakan abstraksi unit atau komponen terkecil pada Kubernetes untuk memudahkan proses pengembangan. \textit{Pod} merupakan sebuah lingkunan linux yang digunakan secara bersama namun memiliki sumber daya yang terpisah dan terbatas melalui teknologi linux cgroups dan namespace untuk menjalankan satu atau lebih \textit{container}. \textit{Pod} bersifat \textit{ephemeral} sehingga seluruh sumber daya akan hilang apabila \textit{pod} mengalami kegagalan \parencite{pod}.

Untuk memastikan bahwa layanan dapat selalu berjalan dengan baik, semua kegiatan yang berkaitan dengan \textit{pod} mulai dari \textit{scaling} hingga \textit{health check} akan dilakukan oleh Kubernetes. Kubernetes akan bertanggung jawab untuk melakukan \textit{penjadwalan} serta pencocokan konfigurasi maupun siklus hidup dari \textit{pod} tersebut. Dengan abstraksi \textit{pod}, proses pengembangan layanan menggunakan Kubernetes semakin mudah untuk dipahami dan dilakukan.

\subsection{\textit{Service component}}

Service merupakan suatu abstraksi untuk membuat \textit{pod} dapat diakses secara eksternal. \textit{Pod} bersifat dan \textit{ephemeral} dan tidak dapat diakses dari luar \textit{pod}. Melalui abstraksi \textit{service} Kubernetes, \textit{pod} tersebut dapat diakses secara eksternal dengan cara membuat suatu layanan intermediet untuk meneruskan \textit{request} dari eksternal.
Layanan intermediet yang disediakan oleh \textit{service} diantaranya \textit{ClusterIp, NodePort, LoadBalancer} dan  \textit{ExternalName}.

ClusterIP bersifat internal dan sangat berkaitan erat dengan \textit{pod}. Apabila \textit{service} tidak dibuat konfigurasinya, \textit{pod} akan selalu memiliki \textit{service} dengan tipe \textit{ClusterIP}. \textit{NodePort} dan \textit{LoadBalancer} akan membuat \textit{pod} dapat ditemukan oleh layanan eksternal dan diakses melalui perangkat lain. Kedua tipe tersebut akan membuat sebuah \textit{endpoint} untuk meneruskan \textit{request} yang masuk berdasarkan label yang diletakan ketika membuat \textit{pod} \parencite{service}
\subsection{\textit{Deployment component}}

\textit{Deployment} merupakan abstraksi yang menggabungkan kedua abstraksi \textit{pod} dan \textit{service}. \textit{Deployment} dapat melihat kondisi seluruh sistem pada saat keadaan awal maupun keadaan berubah. \textit{Deployment} akan menyimpan keadaan sistem secara berkala pengecekan secara deklaratif untuk mendeteksi perubahan keadaan sistem saat ini dengan keadaan sistem yang diinginkan \parencite{deployment}.

\textit{Deployment} memiliki fitur \textit{rollout} dan \textit{rollback} untuk meningkatkan \textit{availability} layanan. \textit{rollback} berarti melakukan penurunan versi dari layanan yang saat ini sedang berjalan sedangkan \textit{rollout} untuk melakukan \textit{upgrade} layanan. Kedua fitur ini berjalan dengan cara membuat \textit{replica} dari layanan yang sedang berjalan untuk mencegah penurunan \textit{availability}. Ketika \textit{deployment} ingin menaikkan versi layanan yang digunakan, \textit{deployment} akan membuat \textit{pods} dengan versi terbaru. Ketika \textit{pods} ini sudah aktif beroprasi dan memiliki status \textit{healthy}, layanan dengan versi sebelumnya baru bisa dihapus \parencite{deployment}.

% Node merupakan suatu lingkungan terisolasi yang menjadi komponen dasar dalam pengelolaan aplikasi dengan kubernetes. 