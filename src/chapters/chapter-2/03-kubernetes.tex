\section{Kubernetes}

Kubernetes adalah sebuah solusi \textit{open source} yang berguna untuk melakukan orkestrasi berbagai aplikasi yang telah di bungkus dalam suatu lingkungan yang disebut sebagai \textit{container}. Kubernetes berfungsi untuk melakukan \textit{deployment} otomatis, \textit{auto scaling} secara otomatis, serta membuat \textit{network} untuk menghubungkan \textit{container} dengan \textit{container} lainnya. Kubernetes membantu mengelola dan mempercepat proses pengembangan layanan yang rumit dengan skala yang besar \parencite{helmkubernetes}.

Kubernetes memiliki beberapa komponen yang digunakan ketika mengelola layanan. \textit{Node, pod, service} dan \textit{deployment} merupakan empat komponen utama yang sering kali menjadi komponen utama ketika membuat suatu layanan dengan kubernetes. \textit{Node} dapat dianalogikan dengan lingkungan \textit{virutual machine} yang memiliki kemampuan terbatas. \textit{Pod} merupakan suatu tempat untuk menjalankan berbagai macam \textit{container} didalamnya. Satu pod dapat memiliki lebih dari satu \textit{container} untuk dioperasikan. \textit{Service} berfungsi untuk membuka akses eksternal ke dalam \textit{pod} yang secara umum bersifat internal dan tidak dapat diakses dari luar. Terakhir yaitu \textit{deployment} adalah suatu konfigurasi untuk menjalankan layanan yang akan dibuat, konfigurasi deployment juga melingkupi komponen \textit{pod} dan \textit{service}.

Kubernetes memiliki fitur seperti \textit{auto scaling, self-healing, service discovery, load balancing} serta \textit{rollout} dan \textit{rollback}. \textit{Auto scaling} sering digunakan untuk menjaga sistem untuk terus beroperasi dengan cara melakukan replikasi layanan dengan jumlah yang ditentukan. \textit{Self healing} memastikan bahwa layanan yang sedang mengalami kegagalan dapat diperbaiki secara otomatis. \textit{Rollout} dan \textit{Rollback} sering digunakan ketika melakukan proses \textit{deployment} untuk mencegah adanya \textit{downtime} ketika meluncurkan layanan versi baru. Seluruh fitur yang disebutkan bekerja sama untuk membuat kubernetes dapat mengelola aplikasi dalam skala yang masif dan mempertahankan kualitas layanan yang dibuat.


% Node merupakan suatu lingkungan terisolasi yang menjadi komponen dasar dalam pengelolaan aplikasi dengan kubernetes. 