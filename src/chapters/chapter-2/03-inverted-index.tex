\section{\textit{Inverted Index}}
\label{sec:invertedindex}

\textit{Inverted Index} merupakan struktur data yang biasa digunakan untuk mesin pencari \parencite{invertedindex2}. Tujuan dari implementasi struktur data ini pada mesin pencari adalah untuk mengoptimalkan kecepatan query dalam mencari dokumen yang mengandung kata kunci tertentu. Struktur data ini melakukan pemetaan terhadap kata dan kumpulan tupel yang berisikan \textit{identifier} (ID) dokumen dan posisi karakter \parencite{invertedindex}. Struktur data ini biasanya dipakai untuk menggantikan \textit{Forward Index}. \textit{Forward Index} adalah struktur data yang menyimpan seluruh kata dalam sebuah dokumen sehingga jika \textit{forward index} di-\textit{query}, maka akan memerlukan iterasi sekuensial pada setiap dokumen dan kata kunci untuk membuktikan dokumen relevan. Sumber daya waktu, memori, dan pemrosesan yang dibutuhkan untuk melakukan query semacam itu tidak realistis dan praktis karena nyatanya, mesin pencari harus melakukan hal tersebut ke ratusan hinga jutaan dokumen. Dengan inverted index yang dibuat, query dapat diselesaikan dengan cara langsung melompat ke \textit{identifier} (ID) kata kunci melalui \textit{random access} pada inverted index untuk mendapatkan \textit{identifier} (ID) dokumen dan posisi karakter.