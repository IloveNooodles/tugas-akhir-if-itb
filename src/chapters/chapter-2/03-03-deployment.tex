\subsection{\textit{Deployment}}

\textit{Deployment} merupakan abstraksi yang menggabungkan kedua abstraksi \textit{pod} dan \textit{service}. \textit{Deployment} dapat melihat kondisi seluruh sistem pada saat keadaan awal maupun keadaan berubah. \textit{Deployment} akan menyimpan keadaan sistem secara berkala pengecekan secara deklaratif untuk mendeteksi perubahan keadaan sistem saat ini dengan keadaan sistem yang diinginkan \parencite{deployment}.

\textit{Deployment} memiliki fitur \textit{rollout} dan \textit{rollback} untuk meningkatkan \textit{availability} layanan. \textit{rollback} berarti melakukan penurunan versi dari layanan yang saat ini sedang berjalan sedangkan \textit{rollout} untuk melakukan \textit{upgrade} layanan. Kedua fitur ini berjalan dengan cara membuat \textit{replica} dari layanan yang sedang berjalan untuk mencegah penurunan \textit{availability}. Ketika \textit{deployment} ingin menaikkan versi layanan yang digunakan, \textit{deployment} akan membuat \textit{pods} dengan versi terbaru. Ketika \textit{pods} ini sudah aktif beroprasi dan memiliki status \textit{healthy}, layanan dengan versi sebelumnya baru bisa dihapus \parencite{deployment}.