\section{Pengujian}
\label{sec:pengujian}


Tujuan dari pengujian ialah untuk memastikan apakah seluruh kebutuhan fungsional dan non-fungsional dari sistem telah terpenuhi.
Untuk pengujian kebutuhan fungsional, pengujian dibagi menjadi dua bagian yaitu pengujian yang dilakukan per komponen lalu dilanjutkan dengan pengujian sistem. Setiap skenario pengujian dijelaskan tujuannya, skenario yang dilakukan, dan hasil pengujian yang didapatkan. Skenario pengujian akan memiliki ID dengan awalan P diikuti dengan dua angka. Seluruh pemetaan pengujian terdapat pada lampiran \ref{chapter:tabel-pengujian}

\subsection{Batasan Pengujian}
Berikut adalah batasan yang ditetapkan dalam melakukan pengujian \textit{sistem remote deployment}.

\begin{enumerate}
  \item Pengujian dilakukan di tiga kluster yang berbeda
        \begin{enumerate}
          \item Kubernetes lokal \textit{cluster} dengan jumlah 4 nodes
          \item \textit{Google Cloud Platform Compute Engine} Kubernetes \textit{cluster} dengan jumlah 2 nodes
          \item RaspberryPi Cluster dengan jumlah 2 nodes
        \end{enumerate}
  \item Sistem \textit{remote deployment} dijalankan pada komputer lokal yang memilki spesifikasi yang telah dijelaskan pada bagian \ref{sec:lingkungan-implementasi}.
  \item Untuk beberapa fungsionalitas admin digunakan \textit{HTTP Client} yaitu Postman untuk membuat \textit{request} kepada \textit{service}
  \item Cluster sudah tersedia dan siap diakses
  \item Setiap \textit{request} akan memiliki header X-Api-Key.
  \item Setiap \textit{request} yang mengarah ke /admin-api/ akan memiliki \textit{header} berupa X-Admin-API-Key.
  \item Database sudah terisi sebagian untuk memudahkan proses pengujian
\end{enumerate}

\subsection{Persiapan Pengujian}
Pada proses pengujian, terdapat tiga lingkungan pengujian yang digunakan untuk menguji sistem \textit{remote deployment}. Ketiga lingkungan tersebut yaitu kubernetes \textit{cluster} lokal, kubernetes \textit{cluster} yang terdapat di \textit{Cloud (GCP)}, serta \textit{cluster} pada RaspberryPi. Ketiga \textit{cluster} ini memiliki jumlah node yang berbeda sesuai dengan penjelasan pada \ref{subsec:batasan-pengujian}.

\subsubsection{Kubernetes Lokal}
Untuk pengujian pada kubernetes lokal, dilakukan pembuatan \textit{cluster} dengan kakas kind untuk membuat cluster yang bernama \textit{testing-cluster-two-nodes} yang memiliki 2 node. Konfigurasi pembuatan sama seperti konfigurasi pada bagian \ref{subsec:persiapan-kubernetes-cluster} hanya saja jumlah nodes yang digunakan yaitu 2.
Karena nodes berjumlah dua maka terdapat 1 \textit{master nodes} dan 1 \textit{slave} node pada \textit{cluster}. Konfigurasi pembuatan cluster dapat dilihat pada gambar \ref{fig:kubernetes-lokal-config-testing}.

\begin{figure}[ht]
  \centering
  \includegraphics[width=1\textwidth]{resources/chapter-4/pengujian/kubernetes-lokal-config.jpg}
  \caption{Konfigurasi Pembuatan \textit{Kubernetes Testing Cluster} Dengan Kakas \textit{Kind}}
  \label{fig:kubernetes-lokal-config-testing}
\end{figure}

Setelah itu jalankan perintah "kind create cluster --config testing-cluster.yaml" untuk membuat \textit{cluster} pada \textit{docker}. Hasil dari perintahh ini yaitu tercipta dua buah \textit{container} pada \textit{docker} yang memiliki peran \textit{master} dan \textit{slave} seperti pada gambar \ref{fig:kubernetes-lokal-config-testing-result}.

\begin{figure}[ht]
  \centering
  \includegraphics[width=1\textwidth]{resources/chapter-4/pengujian/kubernetes-lokal-config-result.jpg}
  \caption{Hasil \textit{Kubernetes Testing Cluster} pada \textit{Docker}}
  \label{fig:kubernetes-lokal-config-testing-result}
\end{figure}

\subsubsection{Kubernetes GCP}
\label{subsubsec:kubernetes-gcp}
Pada lingkungan ini dibuat dua buah \textit{compute engine (virtual machine)} pada \textit{GCP}. Masing masing dari \textit{virtual machine} akan berperan sebagai kubernetes \textit{cluster} yang bernama \textit{prod-cluster-example}.

\begin{enumerate}
  \item Buat dua \textit{virtual machine} pada \textit{compute engine GCP} dengan spesifikasi berikut. Hasil pembuatan \textit{virtual machine} dapat dilihat pada gambar \ref{fig:hasil-pembuatan-virtual-machine-gcp}
        \begin{enumerate}
          \item Ubuntu 24.04
          \item 2GB Memory
          \item 10GB Storage Persistent Disk
          \item 0.5 - 2Vcpu (1 shared core)
          \item Region Asia southeast2-c
        \end{enumerate}
  \item Membuat \textit{Firewall Rule} Untuk membuka port yang digunakan oleh \textit{kubernetes}. Untuk daftar opsi setiap port yang dibuka dapat dilihat pada gambar \ref{fig:daftar-kegunaan-port}. Untuk hasil pembuatan firewall dapat dilihat pada bagian \ref{fig:hasil-firewall-rule-pada-gcp}
  \item Konfigurasi \textit{gare-test-kubernetes-server} sebagai master nodes. Konfigurasi dilakukan dengan cara mengunduh instalasi dari k3s dengan perintah seperti pada gambar \ref{fig:instalasi-master-node-gcp}.
  \item Konfigurasi virutal machine lainnya yaitu \textit{gare-test-kubernetes-server-node} sebagai \textit{worker node}. Untuk meregistrasi \textit{node} ke dalam \textit{cluster} perlu adanya autentikasi untuk memasitikan hanya \textit{node} yang benar yang boleh masuk ke dalam \textit{cluster}. K3s memiliki token generator yang dapat digunakan untuk mencegah akses yang tidak diinginkan, registrasi token dapat dilihat pada gambar \ref{fig:pengambilan-token-registrasi-cluster}. Token tersebut akan digunakan untuk meregistrasi \textit{node} ini ke \textit{master} dengan \textit{public ip} node tersebut. Ilustrasi dapat dilihat pada gambar \ref{fig:instalasi-worker-node-gcp}.
  \item Ambil konfigurasi \textit{cluster} di \textit{master node} dan pindahkan ke lokasi \textit{server berjalan} untuk meregistrasi \textit{cluster} ke dalam sistem. Setelah melakukan kelima langkah ini \textit{cluster} sudah terintegrasi dengan sistem. Ilustrasi pemindahan konfigurasi dapat dilihat pada gambar \ref{fig:konfigurasi-cluster-master-node-gcp} dan \ref{fig:proses-pemindahan-konfigurasi-master-gcp}.
\end{enumerate}

\subsubsection{Kubernetes RaspberryPi}
Pada lingkungan ini dibuat cluster dengan dua nodes pada RaspberryPi. Cluster yang dibuat bernama cluster-raspi yang memiliki spesifikasi hardware RaspberryPi berikut.

\begin{enumerate}
  \item Master node menggunakan Raspberry Pi 3 Model B Rev 1.2 dengan 1GB RAM dan 4 CPU @ 1.2GHz dengan hostname masterpi. Informasi lebih lengkap dapat dilihat pada gambar \ref{fig:hostname-raspi-master-nodes} dan \ref{fig:spesifikasi-raspi-master-nodes}
  \item Worker node menggunakan Raspberry Pi 2 Model B Rev 1.1 dengan 1GB RAM dan 4 CPU @ 900MHz dengan hostname raspberrypi. Informasi lebih lengkap dapat dilihat pada gambar \ref{fig:hostname-raspi-worker-nodes} dan \ref{fig:spesifikasi-raspi-worker-nodes}
\end{enumerate}

Berikut merupakan tata cara pembuatan cluster kubernetes pada \textit{RaspberryPi}. Diasumsikan \textit{device} \textit{RaspberryPi} sudah terhubung ke dalam jaringan yang sama sehingga tidak perlu \textit{port forwarding} / \textit{public ip}.

\begin{enumerate}
  \item Konfigurasi \textit{hostname masterpi} sebagai master nodes. Perlu dilakukan konfigurasi tambahan untuk menambahkan \textit{cgroups} pada raspberrypi karena secara default opsi ini \textit{disabled}. \textit{Cgroups} merupakan kepanjangan dari \textit{Control Groups} yang berfungsi sebagai \textit{resource management} pada linux dan digunakan dalam proses kontainerisasi. Selanjutnya mirip serperti konfigurasi pada bagian \ref{subsubsec:kubernetes-gcp}, perlu mengunduh instalasi dari k3s dengan perintah seperti pada gambar \ref{fig:instalasi-master-raspi-nodes}.
  \item Konfigurasi \textit{node} lainnya yaitu \textit{hostname raspberrypi} sebagai \textit{worker node}. Untuk meregistrasi \textit{node} ke dalam \textit{cluster} perlu adanya autentikasi untuk memasitikan hanya \textit{node} yang benar yang boleh masuk ke dalam \textit{cluster}. K3s memiliki token generator yang dapat digunakan untuk mencegah akses yang tidak diinginkan, registrasi token dapat dilihat pada gambar \ref{fig:raspi-master-gen-token}. Token tersebut akan digunakan untuk meregistrasi \textit{node} ini ke \textit{master} dengan \textit{ip local} node master. Ilustrasi dapat dilihat pada gambar \ref{fig:instalasi-worker-raspi-node}.
  \item Ambil konfigurasi \textit{cluster} di \textit{master node} dan pindahkan ke lokasi \textit{server berjalan} untuk meregistrasi \textit{cluster} ke dalam sistem. Setelah melakukan langkah ini \textit{cluster} sudah terintegrasi dengan sistem. Ilustrasi pemindahan konfigurasi dapat dilihat pada gambar \ref{fig:raspi-kube-config} dan \ref{fig:raspi-add-kubeconfig}.
\end{enumerate}

\subsection{Pengujian Komponen}
Pengujian di level komponen memastikan bahwa seluruh fungsionalitas yang tidak melibatkan servis eksternal di dalam komponen bekerja dengan baik. Pengujian ini akan dibagi menjadi beberapa bagian sesuai dengan domain yang telah dijelaskan pada bagian \ref{subsec:implementasi-service}. Pada masing masing domain terdapat tabel yang memetakan hubungan antara kebutuhan fungsional serta pengujian yang bersesuaian.

\subsubsection{Fungsionalitas pada Domain Company}

Pengujian dengan ID P01 dilakukan dengan skenario yaitu admin berhasil membuat \textit{company} baru dengan nama \textit{cluster} yang tersedia pada sistem. Tersedia artinya konfigurasi kubernetes \textit{cluster} terdapat pada kubernetes \textit{config server}. Daftar \textit{cluster name} yang tersedia dapat dilihat pada gambar \ref{fig:list-cluster-tersedia}. Admin akan membuat request dengan Postman kepada \textit{server} dengan request seperti berikut.

\begin{enumerate}
  \item Mengisi \textit{field name} dengan nilai "new company"
  \item Mengisi \textit{field cluster\textunderscore name} dengan nilai "kind-psa-with-cluster-pss"
\end{enumerate}

\textit{Request} dibuat dengan membuat request menggunakan Postman pada url /admin-api/v1/companies dengan metode POST. \textit{request dan response} dapat dilihat pada gambar \ref{fig:pengujian-p01}

\begin{figure}[ht]
  \centering
  \includegraphics[width=0.8\textwidth]{resources/chapter-4/pengujian/p01.jpg}
  \caption{\textit{Request dan Response Pengujian} P01}
  \label{fig:pengujian-p01}
\end{figure}

Pengujian dengan ID P02 dilakukan dengan skenario yaitu admin tidak berhasil membuat \textit{company} karena nama cluster yang tidak tersedia pada server. Tersedia berarti, konfigurasi kubernetes cluster terdapat pada kubernetes \textit{config server} Admin akan membuat request dengan Postman kepada server dengan request seperti berikut.

\begin{enumerate}
  \item Mengisi \textit{field name} dengan nilai "new company"
  \item Mengisi \textit{field cluster\textunderscore name} dengan nilai "kind-psa-with-cluster-ps"
\end{enumerate}

\textit{Request} dibuat dengan membuat request menggunakan Postman pada url /admin-api/v1/companies dengan metode POST. \textit{request dan response} dapat dilihat pada gambar

\begin{figure}[ht]
  \centering
  \includegraphics[width=0.8\textwidth]{resources/chapter-4/pengujian/p02.jpg}
  \caption{\textit{Request dan Response Pengujian} P02}
  \label{fig:pengujian-p02}
\end{figure}

Pengujian dengan ID P03 dilakukan dengan skenario yaitu admin ingin mendapatkan seluruh \textit{company} yang terdaftar pada sistem. Admin akan membuat GET request dengan Postman ke url /admin-api/v1/companies. Berikut merupakan balikan dari \textit{request} yang dikirimkan

\begin{figure}[ht]
  \centering
  \includegraphics[width=0.8\textwidth]{resources/chapter-4/pengujian/p03.jpg}
  \caption{\textit{Request dan Response Pengujian} P03}
  \label{fig:pengujian-p03}
\end{figure}

Pengujian dengan ID P04 dilakukan dengan skenario yaitu admin ingin menghapus \textit{company} dengan id tertentu dari daftar \textit{company} pada sistem. Admin akan membuat DELETE request dengan Postman ke url /admin-api/v1/companies/:id. Admin menggunakan id company yang berhasil dibuat sesuai dengan gambar \ref{fig:pengujian-p01} sebagai parameter untuk \textit{company} yang dihapus. Berikut merupakan balikan dari \textit{request} yang dikirimkan.

\begin{figure}[ht]
  \centering
  \includegraphics[width=0.8\textwidth]{resources/chapter-4/pengujian/p04.jpg}
  \caption{\textit{Request dan Response Pengujian} P04}
  \label{fig:pengujian-p04}
\end{figure}

Seluruh rekap pengujian pada domain \textit{company} dapat dilihat pada tabel .Berdasarkan hasil yang diperoleh, terbukti bahwa kebutuhan fungsional dengan ID tersebut telah terimplementasi dengan baik


\bgroup
\begin{table}[ht]
  \def\arraystretch{1.7}
  \caption{Skenario dan Hasil Pengujian Domain \textit{Company}}
  \label{tab:pengujian-domain-company}
  \centering
  \begin{tabular}{|p{2cm}|p{2cm}|p{3cm}|p{3cm}|p{2cm}|}
    \hline
    ID Fungsional & ID Pengujian                               & Skenario                                                                                                                                   & Ekspektasi & Realita \\
    \hline
    UC01          & Mendaftarkan perusahaan                    & Sistem memberikan akses kepada admin untuk mendaftarkan perusahaan yang ingin mendaftar ke dalam sistem                                                           \\
    \hline
    UC02          & Mendaftarkan \textit{user}                 & Sistem memberikan akses kepada admin untuk mendaftarkan \textit{user} ke perusahaan tertentu                                                                      \\
    \hline
    UC03          & Manajemen perusahaan                       & Sistem memberikan akses kepada admin untuk melakukan manajemen terhadap seluruh perusahaan yang terdaftar pada sistem                                             \\
    \hline
    UC04          & Manajemen \textit{user}                    & Sistem memberikan akses kepada admin untuk melakukan manajemen terhadap seluruh \textit{user} yang terdaftar pada sistem                                          \\
    \hline
    UC05          & Login                                      & Sistem memberikan akses kepada \textit{user}                                                                                                                      \\
    \hline
    UC06          & Melihat detail perusahaan                  & Sistem memberikan akses kepada \textit{user} untuk melihat perusahaanya                                                                                           \\
    \hline
    UC07          & Melihat \textit{user} pada satu perusahaan & Sistem memberikan akses kepada \textit{user} user lainnya pada satu perusahaan                                                                                    \\

    \hline
    UC08          & Manajemen \textit{perangkat}               & Sistem memberikan akses kepada \textit{user} untuk melihat, membuat, serta menghapus \textit{perangkat} yang terdaftar pada sistem                                \\
    \hline
    UC09          & Manajemen \textit{groups}                  & Sistem memberikan akses kepada \textit{user} untuk melihat, membuat, serta menghapus \textit{groups} yang terdaftar pada sistem                                   \\
    \hline
    UC10          & Manajemen \textit{deployment images}       & Sistem memberikan akses kepada \textit{user} untuk melihat, membuat, serta menghapus \textit{deployment images} yang terdaftar pada sistem                        \\
    \hline
    UC11          & Manajemen \textit{deployment plan}         & Sistem memberikan akses kepada \textit{user} untuk melihat, membuat, serta menghapus \textit{deployment plan} yang terdaftar pada sistem                          \\
    \hline
    UC12          & Melakukan \textit{Remote deployment}       & Sistem memberikan akses kepada \textit{user} untuk melakukan \textit{deployment} kepada target perangakt ataupun \textit{groups}                                  \\
    \hline
    UC13          & Melihat riwayat \textit{deployment}        & Sistem memberikan akses kepada \textit{user} untuk melihat riwayat \textit{deployment} yang telah dilakukan                                                       \\
    \hline
  \end{tabular}
\end{table}
\egroup



\subsubsection{Fungsionalitas pada Domain User}

Pengujian dengan ID P05 dilakukan dengan skenario yaitu admin berhasil membuat \textit{user} baru dengan nama, email, password, serta companyId yang valid. Admin akan membuat request dengan Postman kepada \textit{server} dengan request seperti berikut.

\begin{enumerate}
  \item Mengisi \textit{field name} dengan nilai "new user"
  \item Mengisi \textit{field email} dengan nilai "newuser@gmail.com"
  \item Mengisi \textit{field password} dengan nilai "inicontohpasswordges"
  \item Mengisi \textit{field company\textunderscore id} dengan nilai yang telah didapat sebelumnya yaitu "d6c03902-3758-47bd-994d-616e1917cc61"
\end{enumerate}

\textit{Request} dibuat dengan membuat request menggunakan Postman pada url /admin-api/v1/users dengan metode POST. \textit{Request dan response} dapat dilihat pada gambar \ref{fig:pengujian-p05}


Pengujian dengan ID P06 dilakukan dengan skenario yaitu admin ingin menghapus \textit{user} dengan id tertentu dari daftar \textit{user} pada sistem. Admin akan membuat DELETE request dengan Postman ke url /admin-api/v1/users/:id. Admin menggunakan id user yang berhasil dibuat sesuai dengan gambar \ref{fig:pengujian-p05} sebagai parameter untuk \textit{user} yang dihapus. \textit{Request dan Response} dapat dilihat pada gambar \ref{fig:pengujian-p06}.

Pengujian dengan ID P07 dilakukan dengan skenario yaitu user ingin login ke dalam sistem. Langkah langkah yang dilakukan:

\begin{enumerate}
  \item Mengunjungi halaman /login
  \item Memasukan \textit{field email} dengan "raspi@gmail.com"
  \item Memasukan \textit{field password} dengan "inicontohpasswordges"
  \item Kilk tombol "login"
\end{enumerate}

Setelah seluruh langkah dilakukan, muncul sebuah modal pada kanan bawah dengan pesan "sucess login, redirecting" serta redirect ke halaman / yang menunjukan bahwa proses login telah berhasil. Hasil dapat dilihat pada gambar \ref{fig:pengujian-p07}

Pengujian dengan ID P08 dilakukan dengan skenario yaitu user ingin login ke dalam sistem dengan kredensial yang tidak valid. Langkah langkah yang dilakukan:

\begin{enumerate}
  \item Mengunjungi halaman /login
  \item Memasukan \textit{field email} dengan "raspi@gmail.co"
  \item Memasukan \textit{field password} dengan "inicontohpasswordge"
  \item Kilk tombol "login"
\end{enumerate}

Setelah seluruh langkah dilakukan, muncul sebuah modal pada kanan bawah dengan pesan "invalid combination email or password". Dengan adanya modal ini \textit{user} menjadi \textit{aware} bahwa proses login yang dilakukan gagal dan perlu melakukan pengecekan kembali input yang dimasukkan. Hasil dapat dilihat pada gambar \ref{fig:pengujian-p08}

Pengujian dengan ID P09 dilakukan dengan skenario yaitu user ingin logout dari sistem. Pengujian dilakukan dengan cara menekan tombol logout pada bagian kiri bawah \textit{sidebar}. Setelah itu cookie akan dihapus dan sistem akan melakukan \textit{redirect} ke /login page.

Pengujian dengan ID P10, P11 dilakukan dengan skenario yaitu setelah \textit{user} berhasil login, \textit{user} ingin mendapatkan informasi detail \textit{company} serta seluruh \textit{user} pada \textit{company} tersebut dengan mengunjungi halaman /account. Halaman ini dapat dikunjungi dengan cara menekan tombol "account" pada sidebar. Hasil dapat dilihat pada gambar \ref{fig:pengujian-p08-09}.

Seluruh rekap pengujian pada domain \textit{user} dapat dilihat pada tabel \ref{tab:pengujian-domain-user}. Berdasarkan hasil yang diperoleh, terbukti bahwa kebutuhan fungsional dengan ID F04 hingga F09 telah terimplementasi dengan baik.


\subsubsection{Fungsionalitas pada Domain \textit{Device}}

Pengujian dengan ID P12 dilakukan dengan skenario \textit{user} yang telah terautentikasi ingin menambahkan \textit{device} yang dia miliki ke dalam sistem dengan \textit{node name} yang valid. Langkah langkah yang dilakukan:
\begin{enumerate}
  \item Mengunjungi halaman /devices
  \item Menekan tombol "Add Device"
  \item Mengisi \textit{field name} dengan "raspi-device-1"
  \item Mengisi \textit{field node name} dengan "testing-cluster-two-nodes-control-plane"
  \item Mengisi \textit{field type} dengan nilai "raspi"
  \item Mengisi label dengan "testing=true"
\end{enumerate}

Setelah pengujian dilakukan, muncul deskripsi \textit{device} yang telah dibuat pada tabel yang sebelumnya kosong. Pada bagian kanan bawah juga terdapat modal yang menunjukan bahwa pembuatan \textit{device} berhasil. Hasil dapat dilihat pada lampiran \ref{fig:pengujian-p12-success}.

Pengujian dengan ID P13 dilakukan skenario \textit{user} yang telah terautentikasi ingin menambahkan \textit{device} yang dia miliki ke dalam sistem dengan \textit{node name} yang tidak terdapat pada \textit{cluster}. Langkah langkah yang dilakukan:
\begin{enumerate}
  \item Mengunjungi halaman /devices
  \item Menekan tombol "Add Device"
  \item Mengisi \textit{field name} dengan "raspi-device-1"
  \item Mengisi \textit{field node name} dengan "testing-cluster-two-nodes-control-plan"
  \item Mengisi \textit{field type} dengan nilai "raspi"
  \item Mengisi label dengan "testing=true"
\end{enumerate}

Setelah pengujian dilakukan, muncul sebuah modal pada bagian kanan bawah juga terdapat modal yang menunjukan bahwa pembuatan \textit{device} gagal dengan pesan "node not found". Hasil dapat dilihat pada lampiran \ref{fig:pengujian-p12-failed}.

Pengujian dengan ID P14 dilakukan dengan skenario \textit{user} ingin melihat seluruh \textit{device} yang \textit{company} miliki. Pengujian ini dilakukan dengan cara mengunjungi halaman /devices dan melihat isi dari tabel yang tersedia. Berdasarkan pengujian P11 dapat dilihat bahwa tabel terisi dengan nilai \textit{device} yang telah dibuat, hal ini menunjukan bahwa \textit{user} dapat melihat perangkat yang \textit{company} miliki.

Pengujian dengan ID P15 dilakukan dengan skenario \textit{user} ingin menghapus salah satu \textit{device} yang ia miliki. Langkah langkah yang dilakukan:
\begin{enumerate}
  \item Mengunjungi halaman /devices
  \item Menekan tombol "titik tiga (\textit{elipsis horizontal})" yang berada pada ujung tabel
  \item Menekan tombol "delete" dari popup modal
\end{enumerate}

Setelah melakukan langkah langkah tersebut dapat dilhat bahwa tidak ada data pada tabel. Hal ini menunjukan bahwa \textit{device} berhasil dihapus dari sistem. Hasil pengujian dapat dilihat pada lampiran \ref{fig:pengujian-p15}.

Seluruh rekap pengujian pada domain \textit{devices} dapat dilihat pada lampiran \ref{tab:pengujian-domain-device}. Berdasarkan hasil yang diperoleh, terbukti bahwa kebutuhan fungsional dengan ID F10 hingga F15 telah terimplementasi dengan baik.

\subsubsection{Fungsionalitas pada Domain \textit{Groups}}

Pengujian dengan ID P16 dilakukan dengan skenario \textit{user} ingin membuat \textit{group} baru dengan nama "group-baru" pada sistem. Langkah langkah yang dilakukan:
\begin{enumerate}
  \item Mengunjungi halaman /groups
  \item Menekan tombol "Add Group" pada bagian kanan bawah
  \item Mengisi \textit{field name} dengan nilai "group-baru"
  \item Menekan tombol "submit"
\end{enumerate}

Setelah pengujian dilakukan, tabel yang awalnya kosong terisi dengan deskripsi \textit{group} yang telah dibuat. Selain itu, terdapat modal pada bagian bawah kanan yang menunjukan bahwa pembuatan \textit{group} telah berhasil. Hasil dari pengujian dapat dilihat pada gambar \ref{fig:pengujian-p16}

Pengujian dengan ID P17 dilakukan dengan skenario \textit{user} ingin membuat \textit{group} baru dengan nama yang duplikat pada sistem. Langkah langkah yang dilakukan:
\begin{enumerate}
  \item Mengunjungi halaman /groups
  \item Menekan tombol "Add Group" pada bagian kanan bawah
  \item Mengisi \textit{field name} dengan nilai "group-baru"
  \item Menekan tombol "submit"
\end{enumerate}

Setelah pengujian dilakukan, muncul modal pada bagian bawah kanan yang menunjukan bahwa pembuatan \textit{group} gagal. Pesan yang tidunjukan yaitu "group-baru is exists please another name". Hasil dari pengujian dapat dilihat pada gambar \ref{fig:pengujian-p17}

Pengujian dengan ID P18 dilakukan dengan skenario \textit{user} ingin melihat daftar \textit{group} yang tersedia pada sistem. Pengujian ini dilakukan dengan cara mengunjungi halaman /groups. Dapat dilihat dari pengujian P17 bahwa tabel \textit{groups} terisi dengan nilai yang sudah dibuat sebelumnya. Hal ini menunjukan bawha \textit{user} dapat melihat daftar \textit{groups} pada sistem.

Pengujian dengan ID P19 dilakukan dengan skenario \textit{user} ingin menghapus salah satu \textit{group} yang dimiliki. Langkah langkah yang dilakukan:
\begin{enumerate}
  \item Mengunjungi halaman /groups
  \item Menekan tombol "titik tiga (\textit{elipsis horizontal})" yang berada pada ujung tabel
  \item Menekan tombol "delete" dari popup modal
\end{enumerate}

Setelah melakukan langkah langkah tersebut dapat dilhat bahwa tidak ada data pada tabel. Hal ini menunjukan bahwa \textit{device} berhasil dihapus dari sistem. Hasil pengujian dapat dilihat pada gambar \ref{fig:pengujian-p19}.
\input{chapters/chapter-4/pengujian/02-05-domain-deployment.tex}
\input{chapters/chapter-4/pengujian/02-06-domain-external.tex}

\subsection{Pengujian Sistem}

\subsubsection{Pengujian LED Blink pada RaspberryPi}

Pengujian sistem ini mencakup pengujian dengan ID P20, P21, P22, P23, P24, P25, P26, P27, P28, P29, P30 dan P31. Pada pengujian sistem ini, dilakukan proses \textit{remote deployment} untuk menyalakan lampu blink pada RaspberryPi dengan cluster "cluster-raspi". Dibuat sebuah script python yang berinteraksi dengan GPIO pada pin 2 untuk menyalakan lampu LED. Script dapat dilihat pada lampiran \ref{fig:raspi-python-led-script}

\begin{figure}[ht]
  \centering
  \includegraphics[width=0.8\textwidth]{resources/chapter-4/pengujian/pengujian-sistem-raspi-09-led.jpg}
  \caption{Python Raspi LED Script}
  \label{fig:raspi-python-led-script}
\end{figure}

Dari script tersebut, dibuat docker image dengan base image armv7 yang membungkus python tersebut agar dapat berjalan di sistem IoT. Docker image yang telah dibuat diunggah ke dockerhub dengan nama {gawrgare/led\textunderscore blink}. Isi dari dockerfile dapat dilihat pada lampiran \ref{fig:raspi-docker-led-script}. Setelah docker image tersedia, selanjutnya terdapat beberapa langkah yang harus disiapkan sebelum melakukan \textit{deployment} pada lingkungan raspberrypi.

\begin{enumerate}
  \item Membuat \textit{company} dengan nama "raspi-company" dan \textit{cluster\textunderscore name} "cluster-raspi". Hasil dapat dilihat pada lampiran \ref{fig:pengujian-sistem-raspi-01}
  \item Membuat user pada \textit{company} tersebut dengan email \textit{raspi@gmail.com}. Hasil dapat dilihat pada lampiran \ref{fig:pengujian-sistem-raspi-02}
  \item Login dengan kredensial yang telah dibuat
  \item Mengunjungi halaman /devices dan melakukan registrasi \textit{devices} untuk kedua RaspberryPi dengan daftar nama nodes yang dapat dilihat pada lampiran \ref{fig:pengujian-sistem-raspi-04}
        \begin{enumerate}
          \item \textit{Device} "raspi-master" untuk \textit{hostname masterpi} dengan nama \textit{cluster} yaitu "master-node-raspi"
          \item  \textit{Device} "raspi-worker" untuk \textit{hostname raspberrypi} dengan nama \textit{cluster} yaitu "worker-node-raspi"
        \end{enumerate}
  \item Mengunjungi halaman /groups dan membuat \textit{group} "raspi-group-blink". Hasil dapat dilihat pada lampiran \ref{fig:pengujian-sistem-raspi-05}
  \item Mengunjungi halaman group detail "raspi-group-blink" lalu menambahkan kedua device ke dalam group.
  \item Mengunjungi halaman \textit{deployment} lalu membuat repository dengan nama "raspi-image-blink" dan image\textit{ gawrgare/led\textunderscore blink}.
  \item Membuat deployment dengan nama "raspi-deployment-blink" dan mengisi target dengan "node=master" pada halaman \textit{deployment}.
\end{enumerate}

Layout dari RaspberryPi serta letak kabel dan LED dapat dilihat pada lampiran \ref{fig:pengujian-sistem-raspi-layout}. Setelah semua persiapan dilakukan, dilakukan pengujian dengan dua jenis \textit{deployment} yaitu \textit{target} dan \textit{custom} dengan versi \textit{group}.
\begin{enumerate}
  \item Target deployment merupakan deployment yang dilakukan sesuai dengan nilai target pada deployment object, dalam kasus ini target nya ialah node dengan attribut "node=master". Pengujian tipe ini hanya melakukan deployment dengan \textit{hostname masterpi} yang memiliki nama node "master-node-raspi". Setelah deployment dilakukan, LED yang terhubung dengan pin pada "master-node-raspi" berkedip, hal ini menunjukan bahwa proses deployment telah berhasil dilakukan. Hasil dapat dilihat pada lampiran \ref{fig:hasil-pengujian-sistem-raspi-target}.
  \item Custom deployment mengabaikan nilai target yang terdapat pada objek deployment dan melakukan deployment sesuai jenis serta id yang diinginkan. Terdapat dua jenis yaitu \textit{group} dan \textit{device}. Dalam kasus groups, Deployment dilakukan ke dua \textit{device} yang terhubung ke \textit{group} "raspi-group-blink". Setelah deployment dilakukan, kedua LED berkedip menunjukan bahwa script berhasil dijalankan untuk setiap \textit{device} yang terhubung pada group. Hasil dapat dilihat pada lampiran \ref{fig:hasil-pengujian-sistem-raspi-custom}.
\end{enumerate}

Setelah deployment berhasil dilakukan, pengujian dilanjutkan dengan mengunjungi halaman detail dari "raspi-deployment-blink" untuk melihat riwayat deployment. Hasil dapat dilhat pada lampiran \ref{fig:pengujian-sistem-raspi-10}. Dapat dilihat bahwa pengujian dengan ID P20 hingga P31 menunjukan hasil yang sesuai dengan kebutuhan fungsional yang telah didefinisikan. Seluruh rekap pengujian sistem ini dapat dilihat pada lampiran \ref{tab:pengujian-sistem-raspi}.

\subsection{Pengujian Non Fungsional}
Pada bagian ini, dilakukan pengujian terhadap kebutuhan non-fungsional sistem. Terdapat tiga kebutuhan nonfungsional mulai dari \textit{Security dan Portability}.

\subsubsection{Security}
Pengujian kebutuhan non-fungsional ini dilakukan dengan cara mencoba mengakses \textit{dashboard} dan \textit{service} tanpa meletakan token authentikasi. Bilang yang mau diuji dua aspek ini muali dari auth dan authorization

\begin{enumerate}
  \item Mengakses \textit{dashboard} tanpa kredensial

        Ketika mengakses \textit{dashboard} tanpa memiliki kredensial, \textit{dashboard} memiliki \textit{middleware} yang akan mengecek token yang disimpan pada \textit{client}. Jika token tidak valid maka sistem akan langsung melakukan \textit{redirect} ke halaman login.

  \item Mengakses user \textit{service} tanpa kredensial

        Ketika mencoba untuk mengakses \textit{service} pada endpoint apapun, \textit{service} memiliki \textit{middleware}
        yang akan mengecek header authentikasi yang dikirimkan oleh client. Jika kredensial pada header tidak dicantumkan maka akan mengembalikan \textit{401 Unauthorized} \ref{fig:akses-service-user}

  \item Mengakses admin \textit{service} tanpa kredensial

        Admin endpoint memiliki \textit{middleware} validateAdminJWTKey. Ketika mencoba untuk mengakses \textit{endpoint} admin tanpa memberikan header X-Admin-Api-Key yang sesuai maka \textit{service} akan mengembalikan \textit{401 Unauthorized} seperti pada lampiran \ref{fig:akses-service-admin}

  \item Mengakses \textit{url kubernetes} tanpa kredensial

        Seluruh cluster kubernetes akan mengexpose endpoint pada port 6443. Pengujian ini akan mengakses url kubernetes yaitu https://34.101.95.240:6443/. Ketika diakses, hasilnya akan menunjukan status \textit{401 Unauthorized} seperti pada lampiran \ref{fig:akses-service-kubernetes}

\end{enumerate}

\subsubsection{Portability}
Pengujian kebutuhan non-fungsional ini dialkukan dengan dua skenario yaitu mengakses \textit{dashboard} dari mobile serta mengakses dari berbagai \textit{browser}.

Portability tidak

\begin{enumerate}
  \item Mengakses \textit{dashboard} dari perangkat mobile

        \textit{dashboard} berhasil diakses melalui perangkat mobile dan dapat dilihat hasil dapat dilihat pada lampiran \ref{fig:akses-dashboard-mobile}.

  \item Mengakses \textit{dashboard} dari Chromium based browser

        \textit{dashboard} berhasil diakses melalui browser safari yang dapat dilhat pada lampiran \ref{fig:akses-dashboard-chromium}.

  \item Mengakses \textit{dashboard} dari browser safari

        \textit{dashboard} berhasil diakses melalui browser chromium based yaitu Arc yang dapat dilhat pada lampiran \ref{fig:akses-dashboard-safari}.

\end{enumerate}
