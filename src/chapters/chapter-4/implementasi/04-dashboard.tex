\subsection{Implementasi \textit{dashboard}}
komponen \textit{dashboard} dibuat dengan menggunakan bahasa pemrogramman \textit{typescript} dan \textit{vue}. Framework yang digunakan dalam membuat \textit{Dashboard} adalah \textit{Nuxt}. \textit{Nuxt} menjadi pilihan karena memiliki \textit{developer experience} yang bagus serta fitur yang cukup lengkap. \textit{Nuxt} juga memiliki \textit{UI library} yaitu \textit{NuxtUI} yang memudahkan pembuatan \textit{UI}. Pada komponen ini, terdapat 13 halaman yang dapat dikunjungi oleh \textit{user}. Detail dari penjelasan setiap halaman dapat ditemukan pada subbab dibawah ini.

\subsubsection{Halaman \textit{Login}}
Halaman ini berada pada \textit{route} \textbf{/login}. Halaman ini yang berfungsi sebagai \textit{entrypoint} dari komponen \textit{dashboard}. Pada halaman ini terdapat dua input yang di \textit{wrap} oleh sebuah \textit{form}. Input berupa \textit{email} dan \textit{password} \textit{user}. Setelah \textit{user} memasukan \textit{email} dan \textit{password} yang sesuai, maka akan dilakukan \textit{redirect} ke laman utama untuk menunjukan bahwa \textit{user} berhasil terautentikasi dan menggunakan fungsionalitas \textit{dashboard}.

\begin{figure}[h]
  \centering
  \includegraphics[width=1\textwidth]{resources/chapter-4/dashboard/login-page.jpg}
  \caption{Halaman Login}
  \label{fig:halaman-login}
\end{figure}

\subsubsection{Halaman utama}
Halaman ini merupakan halaman utama dari komponen \textit{dashboard}. Pada halaman ini \textit{user} dapat melihat status dari masing masing objek mulai dari \textit{deployment}, \textit{device}, \textit{group}, serta \textit{quick actions} untuk menuju halaman terkait.

\subsubsection{Halaman \textit{Account}}
Halaman ini berada pada \textit{route \textbf{/account}}. Halaman ini berfungsi untuk memberikan detail mengenai \textit{company} dari \textit{user}. Informasi \textit{company} ditampilkan pada sebuah \textit{card} yang berada pada tengah halaman. Pada halaman ini juga ditampilkan informasi mengenai daftar \textit{user} yang berada pada \textit{company} yang sama dan ditampilkan dengan sebuah tabel. Pada bagian akhir tabel terdapat tombol dengan \textit{icon} elipsis, tombol ini berfungsi sebagai \textit{action} yang dapat dilakukan untuk masing masing \textit{row}

\begin{figure}[h]
  \centering
  \includegraphics[width=1\textwidth]{resources/chapter-4/dashboard/account-page.jpg}
  \caption{Halaman \textit{account}}
  \label{fig:halaman-account}
\end{figure}

\subsubsection{Halaman \textit{Device}}
Halaman ini berada pada \textit{route \textbf{/devices}}. Halaman ini berfungsi untuk melakukan manajemen \textit{device} pada satu perusahaan. Halaman ini menunjukan informasi seluruh \textit{device} yang terdaftar pada \textit{company}.

\begin{figure}[h]
  \centering
  \includegraphics[width=1\textwidth]{resources/chapter-4/dashboard/device-page.jpg}
  \caption{Halaman \textit{device}}
  \label{fig:halaman-device}
\end{figure}

Sama seperti halaman \textit{account}, terdapat tombol dengan icon elipsis pada bagian kanan untuk melakukan fungsi seperti melihat detail ataupun menghapus \textit{device} terkait. Apabila \textit{user} menekan tombol \textit{detail} maka user akan diarahkan ke halaman \textbf{/device/:id} sesuai dengan id \textit{device} yang dipilih

\begin{figure}[h]
  \centering
  \includegraphics[width=1\textwidth]{resources/chapter-4/dashboard/device-page-actions.jpg}
  \caption{Actions pada table \textit{device}}
  \label{fig:halaman-device-actions}
\end{figure}

Pada halaman ini terdapat tombol yang dapat ditekan untuk menambahkan \textit{device}. Apabila ditekan akan muncul sebuah modal. Modal ini berisi input yang dapat \textit{user} isi untuk membuat sebuah \textit{device} baru pada sistem. Terdapat validasi pada setiap input, setelah semua validasi dilewati, tombol \textit{submit} akan mengirimkan \textit{request} ke \textit{server} untuk di proses.

Akan muncul sebuah notifikasi pada bagian kanan bawah tergantung \textit{response} yang diberikan oleh \textit{server}. Warna hijau menandakan bahwa \textit{response} sukses dan warna \textit{merah} menandakan bahwa terdapat masalah ketika memproses \textit{request}.

\begin{figure}[h]
  \centering
  \includegraphics[width=1\textwidth]{resources/chapter-4/dashboard/device-page-add.jpg}
  \caption{Modal menambahkan \textit{device}}
  \label{fig:halaman-device-add}
\end{figure}



\pagebreak

\subsubsection{Halaman \textit{Device detail}}
Halaman ini dapat diakses dengan cara mengunjungi \textbf{/devices/:id} dari tombol \textit{detail} pada \textit{actions} yang berada pada halaman \textbf{/devices}. Pada halaman ini \textit{user} dapat menambahkan \textit{device} ke dalam sebuah group dengan menekan tombol \textit{add group}.

Modal \textit{add group} akan muncul jika tombol ditekan, modal ini memliki sebuah dropdown yang telah berisi \textit{group} yang tersedia pada sistem. Akan muncul sebuah notifikasi pada bagian bawah kanan untuk menandakan bahwa \textit{request} berhasil di proses oleh server. Warna hijau menandakan sukses dan warna merah menandakan bawhwa \textit{server} belum berhasil memproses permintaan tersebut.

Pada halaman ini juga terdapat tombol \textit{delete} yang berada pada pojok kanan atas. Jika \textit{user} memutuskan untuk menghapus \textit{device}, akan muncul sebuah modal konfirmasi sebelum aksi penghapusan dilakukan.
\begin{figure}[h]
  \centering
  \includegraphics[width=1\textwidth]{resources/chapter-4/dashboard/device-detail-page.jpg}
  \caption{Halaman \textit{device detail}}
  \label{fig:halaman-device-detail}
\end{figure}

\begin{figure}[h]
  \centering
  \includegraphics[width=1\textwidth]{resources/chapter-4/dashboard/device-detail-add-group.jpg}
  \caption{Modal menambahkan group pada \textit{device detail}}
  \label{fig:halaman-device-detail-add-group}
\end{figure}

\begin{figure}[h]
  \centering
  \includegraphics[width=1\textwidth]{resources/chapter-4/dashboard/device-detail-delete.jpg}
  \caption{Modal menghapus device pada halaman \textit{device detail}}
  \label{fig:halaman-device-detail-delete}
\end{figure}

\pagebreak

\subsubsection{Halaman \textit{Groups}}
Halaman ini berada pada \textit{route \textbf{/groups}}. Halaman ini menunjukan informasi mengenai \textit{groups} yang telah terdaftar pada sistem. Informasi ditampilkan dalam bentuk tabel yang berisi detail dari objek \textit{groups}.

Pada tabel terdapat tombol elipsis yang terletak pada bagian kanan dari masing masing \textit{row}. Sama seperti tabel lainnya, tombol ini berfungsi untuk melakukan \textit{action} pada \textit{row}. Aksi yang dapat dilakukan berupa mengakses halaman detail atau menghapus \textit{item} tersebut.

Apabila tombol \textit{detail} dipilh maka \textit{user} akan menuju halaman \textit{group detail}. Apabila tombol \textit{delete} dipilih maka \textit{item} akan dihapus dari sistem dan muncul notifikasi pada bagian bawah yang menandkana bahwa proses berhasil dilakukan.

Pada halaman ini juga \textit{user} dapat menambahkan \textit{groups} dengan menekan tombol \textit{add group}. Ketika tombol ini ditekan akan muncul modal yang memiliki input nama. Input ini memiliki validasi berupa panjang karakter yang dimasukan haruslah memiliki panjang minimal 8 karakter. Setelah validasi berhasil di lewati, maka tombol submit akan berfungsi untuk mengirimkan \textit{request} ke server. Akan muncul notifikasi pada bagian bawah kanan untuk menandakan bahwa \textit{request} berhasil di proses. Warna hijau menandakan bahwa \textit{request} berhasil di proses dan warna merah menandakan sebaliknya.

\begin{figure}[h]
  \centering
  \includegraphics[width=1\textwidth]{resources/chapter-4/dashboard/groups-page.jpg}
  \caption{Halaman \textit{groups}}
  \label{fig:halaman-groups}
\end{figure}

\begin{figure}[h]
  \centering
  \includegraphics[width=1\textwidth]{resources/chapter-4/dashboard/groups-page-actions.jpg}
  \caption{Actions pada table \textit{groups}}
  \label{fig:halaman-groups-actions}
\end{figure}

\begin{figure}[h]
  \centering
  \includegraphics[width=1\textwidth]{resources/chapter-4/dashboard/groups-page-add.jpg}
  \caption{Modal menambahkan \textit{groups}}
  \label{fig:halaman-groups-add}
\end{figure}

\pagebreak

\subsubsection{Halaman \textit{Groups detail}}
Halaman ini dapat diakses oleh \textit{pengguna} melalui tombol \textit{detail} yang ada pada tabel di halaman \textit{groups}. Halaman ini memiliki \textit{route \textbf{/groups/:id}}. Pada halaman ini \textit{user} dapat melihat informasi mengenai detail dari \textit{groups} yang dipilih mulai dari nama, deskripsi dan \textit{device} apa saja yang telah terhubung pada \textit{group tersebut}.

\textit{user} juga dapat menambahkan \textit{device} dengan cara menekan tombol \textit{add devices} yang akan memunculkan modal berisi dropdown seluruh perangkat yang belum memiliki \textit{groups}.

\begin{figure}[h]
  \centering
  \includegraphics[width=1\textwidth]{resources/chapter-4/dashboard/groups-detail-page.jpg}
  \caption{Halaman \textit{groups detail}}
  \label{fig:halaman-groups-detail}
\end{figure}

\begin{figure}[h]
  \centering
  \includegraphics[width=1\textwidth]{resources/chapter-4/dashboard/groups-detail-add-device.jpg}
  \caption{Modal menambahkan group pada \textit{groups detail}}
  \label{fig:halaman-groups-detail-add-group}
\end{figure}

\begin{figure}[h]
  \centering
  \includegraphics[width=1\textwidth]{resources/chapter-4/dashboard/groups-detail-delete.jpg}
  \caption{Modal menghapus groups pada halaman \textit{groups detail}}
  \label{fig:halaman-groups-detail-delete}
\end{figure}

\pagebreak

\subsubsection{Halaman \textit{Deployment}}
Halaman ini merupakan fungsionalitas utama dari sistem \textit{remote deployment}. Halaman ini dapat diakses melalui \textit{sidebar} dan memiliki \textit{route \textbf{/deployments}}. Pada halaman ini \textit{user} dapat melihat informasi mengenai \textit{deployment plan} yang terdaftar pada sistem. Selain itu juga terdapat informasi mengenai \textit{deployment images} yang tersedia dalam sistem. Kedua informasi ditampilkan dalam bentuk tabel yang masing masing memiliki tombol \textit{add} pada bagian kanan bawah tabel.

Tombol add tersebut akan memunculkan modal yang berisi input yang harus diisi jika ingin membuat \textit{item} baru. Sama seperti tabel lainnya, pada masing masing tabel terdapat tombol elipsis pada bagian kanan untuk melakukan aksi berupa melihat \textit{detail} ataupun menghapus \textit{item} yang bersesuaian.

Tombol \textit{detail} akan mengarahkan \textit{user} ke halaman detail yang bersesuaian.

\begin{figure}[h]
  \centering
  \includegraphics[width=1\textwidth]{resources/chapter-4/dashboard/deployment-page.jpg}
  \caption{Halaman \textit{deployment}}
  \label{fig:halaman-deployment}
\end{figure}

\begin{figure}[h]
  \centering
  \includegraphics[width=1\textwidth]{resources/chapter-4/dashboard/deployment-page-add-deployment.jpg}
  \caption{Modal menambahkan \textit{deployment}}
  \label{fig:halaman-deployment-add-deployment}
\end{figure}

\begin{figure}[h]
  \centering
  \includegraphics[width=1\textwidth]{resources/chapter-4/dashboard/deployment-page-add-repostory.jpg}
  \caption{Modal menambahkan \textit{image} pada halaman \textit{deployment}}
  \label{fig:halaman-deployment-add-repostory}
\end{figure}

\pagebreak

\subsubsection{Halaman \textit{deployments detail}}
Halaman ini dapat diakses melalui tombol \textit{detail} di tabel \textit{deployment} pada halaman \textit{deployment}. Halaman ini memiliki \textit{route \textbf{/deployments/:id}}. Halaman ini menunjukan informasi mengenai \textit{deployment} yang telah dilakukan, status dari \textit{deployment}, serta target dari \textit{deployment} tersebut.

\begin{figure}[h]
  \centering
  \includegraphics[width=1\textwidth]{resources/chapter-4/dashboard/deployment-detail-page.jpg}
  \caption{Halaman \textit{deployment detail}}
  \label{fig:halaman-deployment-detail}
\end{figure}

\begin{figure}[h]
  \centering
  \includegraphics[width=1\textwidth]{resources/chapter-4/dashboard/deployment-detail-delete.jpg}
  \caption{Modal menghapus deployment pada halaman \textit{deployment detail}}
  \label{fig:halaman-deployment-detail-delete}
\end{figure}

\pagebreak

\subsubsection{Halaman \textit{FAQ}}
Halaman ini berada pada \textit{route \textbf{/faq}}. Halaman ini bertujuan untuk memberikan informasi mengenai tata cara hal yang perlu dilakukan sebelum mendaftarkan \textit{device} ke sistem. Terdapat dua bagian yang dibedakan dari banyaknya \textit{device} yang terhubung ke dalam sistem. Dua kategori tersebut yaitu jika belum memiliki \textit{device} sama sekali dan memiliki setidaknya 1 \textit{device} yang telah terhubung dengan sistem.
