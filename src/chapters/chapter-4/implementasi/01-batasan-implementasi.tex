\subsection{Batasan Implementasi}
Berikut adalah batasan yang ditetapkan dalam melakukan implementasi \textit{sistem remote deployment}.

\begin{enumerate}
  \item Semua batasan masalah dan konfigurasi yang telah dibahas pada bagian \ref{sec:batasan-masalah}.
  \item \textit{Platform agnostic} berarti PERISAI dapat dijalankan pada berbagai perangkat yang dapat menjalankan kubernetes.
  \item Kubernetes cluster berjalan di lokal dengan menggunakan kakas \textit{kind} dan hanya dibatasi menjadi 4 node dengan spesifikasi \textit{1 master} dan \textit{3 worker}
  \item \textit{Device} sudah terkoneksi sebelumnya sehingga tidak perlu register perangkat sebagai \textit{node} dan menghubungkannya ke dalam \textit{cluster}.
  \item \textit{Dashboard} hanya memilki fungsionalitas untuk \textit{user}
  \item Kubernetes memiliki batasan cluster sebagai berikut:
        \begin{enumerate}
          \item Maksimal 5000 nodes
          \item Maksimal 110 pods untuk setiap node
          \item Maksimal 150,000 total pods
          \item Maksimal 300,000 total containers
        \end{enumerate}
\end{enumerate}