\subsection{Kakas yang Digunakan}
Dalam melakukan implementasi ini diperlukan beberapa kakas, diantaranya adalah sebagai berikut.
\begin{enumerate}
  \item \textit{Docker}, \textit{Docker Desktop} dan \textit{Docker Desktop Kubernetes} untuk dipakai sebagai \textit{containerization} dan \textit{cluster} kubernetes lokal.
  \item Pandas dan Numpy untuk keperluan \textit{data processing} serta bentuk data untuk dikirimkan ke komponen lain serta model prediksi ARIMA.
  \item \textit{Kubernetes Python Client} untuk mengontrol \textit{cluster} kubernetes melalui kode Python.
  \item \textit{Pickle} untuk menyimpan model ARIMA sehingga persisten meskipun sistem di-\textit{restart}.
  \item \textit{Statsmodels} dan \textit{pmdarima} untuk membangun model ARIMA serta melakukan otomasi pencarian orde atau lebih dikenal sebagai Auto-ARIMA.
\end{enumerate}