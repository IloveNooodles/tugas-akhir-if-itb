\subsubsection{Pengujian Blink pada RaspberryPi}

Pada pengujian sistem ini, akan dilakukan proses \textit{remote deployment} untuk menyalakan lampu blink pada RaspberryPi. Dibuat sebuah docker image yang akan berinteraksi dengan GPIO pin pada raspi. Docker image yang telah dibuat diunggah ke dockerhub dengan nama {gawrgare/led\textunderscore blink}. Isi dari dockerfile dapat dilihat pada lampiran Berikut merupakan persiapan pengujian pada lingkungan ini.
\begin{enumerate}
  \item Membuat \textit{company} dengan nama "raspi-company" dan \textit{cluster\textunderscore name} "cluster-raspi". Hasil dapat dilihat pada gambar
  \item Membuat user pada \textit{company} tersebut dengan email \textit{raspi@gmail.com}. Hasil dapat dilihat pada gambar
  \item Login dengan kredensial yang telah dibuat. Hasil dapat dilihat pada gambar
  \item Melakukan registrasi \textit{devices} untuk kedua raspi dengan daftar nama nodes yang dapat dilihat pada gambar
        \begin{enumerate}
          \item \textit{Device} "raspi-master" untuk \textit{hostname masterpi} dengan nama \textit{cluster} yaitu "master-node-raspi"
          \item  \textit{Device} "raspi-worker" untuk \textit{hostname raspberrypi} dengan nama \textit{cluster} yaitu "worker-node-raspi"
        \end{enumerate}
  \item Membuat groups dengan nama "raspi-group-blink". Hasil dapat dilihat pada gambar
  \item Menambahkan kedua device ke dalam group.
  \item Membuat repository pada halaman \textit{deployment} dengan nama "raspi-image-blink" dan image\textit{ gawrgare/led\textunderscore blink}.
  \item Membuat deployment dengan nama "raspi-deployment-blink" dan mengisi target dengan "node=master".
\end{enumerate}

Setelah semua persiapan dilakukan, akan dilakukan pengujian dengan dua tipe \textit{deployment}.
\begin{enumerate}
  \item Targeted deployment dengan target "node=master". Pengujian tipe ini hanya akan melakukan deployment ke \textit{device} dengan \textit{hostname masterpi} yang memiliki nama node "master-node-raspi". Hasil dapat dilihat pada gambar
  \item Custom deployment dengan tipe GROUP. Deployment ini akan melakukan deployment ke dua \textit{device}.
\end{enumerate}

