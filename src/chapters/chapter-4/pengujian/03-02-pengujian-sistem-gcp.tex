\subsubsection{Penguijan Deployment NGINX pada \textit{Clutser GCP}}

Pada pengujian sistem ini, dilakukan proses \textit{remote deployment} NGINX pada \textit{cluster GCP} dengan cluster name "prod-cluster-example". Pengujian ini dilakukan untuk membuktikan \textit{compatibility} dari sistem yang tidak terbatas pada IoT namun seluruh perangkat yang berbasis UNIX. Proses \textit{remote deployment} dilakukan dengan menggunakan image NGINX yang ada pada \textit{dockerhub}. Pengujian dilakukan dengan mengikuti langkah langkah berikut:
\begin{enumerate}
  \item Membuat company dengan nama "test-semhas" dan nama cluster "prod-cluster-example". Hasil dapat dilihat pada gambar \ref{fig:pengujian-sistem-gcp-01}
  \item Membuat users dengan email "test@gmail.com", nama "test-user-semhas". Hasil dapat dilihat pada gambar \ref{fig:pengujian-sistem-gcp-02}
  \item Login dengan menggunakan kredensial "test@gmail.com". Hasil dapat dilihat pada gambar \ref{fig:pengujian-sistem-gcp-03}
  \item Membuat \textit{device} dengan nama "raspberrypi-pi-1" dan nama \textit{node} "master-cluster" serta memiliki label "sukses=aamiin". Hasil dapat dilihat pada gambar \ref{fig:pengujian-sistem-gcp-04}
  \item Membuat \textit{deployment plan v1} dengan mengambil repository "nginx" serta memiliki nama "deploy-nginx-to-system". Hasil dapat dilihat pada gambar \ref{fig:pengujian-sistem-gcp-05}
  \item Melakukan \textit{remote deployment} dengan tipe "TARGET". Hasil dapat dilihat pada gambar \ref{fig:pengujian-sistem-gcp-06}
  \item Setelah deployment berhasil dilakukan, hapus \textit{deployment} yang telah dibuat. Hasil dapat dilihat pada gambar \ref{fig:pengujian-sistem-gcp-07}.
\end{enumerate}