\subsubsection{Fungsionalitas pada Domain User}

Pengujian dengan ID P05 dilakukan dengan skenario yaitu admin berhasil membuat \textit{user} baru dengan nama, email, password, serta companyId yang valid. Admin akan membuat request dengan Postman kepada \textit{server} dengan request seperti berikut.

\begin{enumerate}
  \item Mengisi \textit{field name} dengan nilai "new user"
  \item Mengisi \textit{field email} dengan nilai "newuser@gmail.com"
  \item Mengisi \textit{field password} dengan nilai "inicontohpasswordges"
  \item Mengisi \textit{field company\textunderscore id} dengan nilai yang telah didapat sebelumnya yaitu "d6c03902-3758-47bd-994d-616e1917cc61"
\end{enumerate}

\textit{Request} dibuat dengan membuat request menggunakan Postman pada url /admin-api/v1/users dengan metode POST. \textit{Request dan response} dapat dilihat pada gambar \ref{fig:pengujian-p05}


Pengujian dengan ID P06 dilakukan dengan skenario yaitu admin ingin menghapus \textit{user} dengan id tertentu dari daftar \textit{user} pada sistem. Admin akan membuat DELETE request dengan Postman ke url /admin-api/v1/users/:id. Admin menggunakan id user yang berhasil dibuat sesuai dengan gambar \ref{fig:pengujian-p05} sebagai parameter untuk \textit{user} yang dihapus. \textit{Request dan Response} dapat dilihat pada gambar \ref{fig:pengujian-p06}.

Pengujian dengan ID P07 dilakukan dengan skenario yaitu user ingin login ke dalam sistem. Langkah langkah yang dilakukan:

\begin{enumerate}
  \item Mengunjungi halaman /login
  \item Memasukan \textit{field email} dengan "raspi@gmail.com"
  \item Memasukan \textit{field password} dengan "inicontohpasswordges"
  \item Kilk tombol "login"
\end{enumerate}

Setelah seluruh langkah dilakukan, muncul sebuah modal pada kanan bawah dengan pesan "sucess login, redirecting" serta redirect ke halaman / yang menunjukan bahwa proses login telah berhasil. Hasil dapat dilihat pada gambar \ref{fig:pengujian-p07}

Pengujian dengan ID P08 dilakukan dengan skenario yaitu user ingin login ke dalam sistem dengan kredensial yang tidak valid. Langkah langkah yang dilakukan:

\begin{enumerate}
  \item Mengunjungi halaman /login
  \item Memasukan \textit{field email} dengan "raspi@gmail.co"
  \item Memasukan \textit{field password} dengan "inicontohpasswordge"
  \item Kilk tombol "login"
\end{enumerate}

Setelah seluruh langkah dilakukan, muncul sebuah modal pada kanan bawah dengan pesan "invalid combination email or password". Dengan adanya modal ini \textit{user} menjadi \textit{aware} bahwa proses login yang dilakukan gagal dan perlu melakukan pengecekan kembali input yang dimasukkan. Hasil dapat dilihat pada gambar \ref{fig:pengujian-p08}

Pengujian dengan ID P09 dilakukan dengan skenario yaitu user ingin logout dari sistem. Pengujian dilakukan dengan cara menekan tombol logout pada bagian kiri bawah \textit{sidebar}. Setelah itu cookie akan dihapus dan sistem akan melakukan \textit{redirect} ke /login page.

Pengujian dengan ID P10, P11 dilakukan dengan skenario yaitu setelah \textit{user} berhasil login, \textit{user} ingin mendapatkan informasi detail \textit{company} serta seluruh \textit{user} pada \textit{company} tersebut dengan mengunjungi halaman /account. Halaman ini dapat dikunjungi dengan cara menekan tombol "account" pada sidebar. Hasil dapat dilihat pada gambar \ref{fig:pengujian-p08-09}.

Seluruh rekap pengujian pada domain \textit{user} dapat dilihat pada tabel \ref{tab:pengujian-domain-user}. Berdasarkan hasil yang diperoleh, terbukti bahwa kebutuhan fungsional dengan ID F04 hingga F09 telah terimplementasi dengan baik.

