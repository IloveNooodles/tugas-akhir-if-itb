\subsection{Batasan Pengujian}
\label{subsec:batasan-pengujian}
Berikut adalah batasan yang ditetapkan dalam melakukan pengujian \textit{sistem remote deployment}.

\begin{enumerate}
  \item Pengujian dilakukan di tiga kluster yang berbeda
        \begin{enumerate}
          \item Kubernetes lokal \textit{cluster}
          \item \textit{Google Cloud Platform (GCP) Compute Engine} Kubernetes \textit{cluster}
          \item RaspberryPi Cluster
        \end{enumerate}
  \item Setiap \textit{cluster} memiliki jumlah node yang sama yaitu 2.
  \item \textit{Cluster} dibuat dengan distribusi kubernetes k3s.
  \item Sistem \textit{remote deployment} dijalankan pada komputer lokal yang memilki spesifikasi yang telah dijelaskan pada bagian \ref{sec:lingkungan-implementasi}.
  \item Untuk beberapa fungsionalitas admin digunakan \textit{HTTP Client} yaitu Postman untuk membuat \textit{request} kepada \textit{service}
  \item Cluster sudah tersedia dan siap diakses.
  \item Setiap \textit{request} memiliki header X-Api-Key.
  \item Setiap \textit{request} yang mengarah ke /admin-api/ memiliki \textit{header} berupa X-Admin-API-Key.
  \item \textit{Database} pada bagian "deployment images" sudah terisi sebagian untuk memudahkan proses pengujian.
\end{enumerate}