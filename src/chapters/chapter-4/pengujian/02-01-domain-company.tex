\subsubsection{Fungsionalitas pada Domain Company}

Pengujian dengan ID P01 dilakukan dengan skenario yaitu admin berhasil membuat \textit{company} baru dengan nama \textit{cluster} yang tersedia pada sistem. Tersedia artinya konfigurasi kubernetes \textit{cluster} terdapat pada kubernetes \textit{config server}. Daftar \textit{cluster name} yang tersedia dapat dilihat pada gambar \ref{fig:list-cluster-tersedia}. Admin akan membuat request dengan Postman kepada \textit{server} dengan request seperti berikut.

\begin{enumerate}
  \item Mengisi \textit{field name} dengan nilai "new company"
  \item Mengisi \textit{field cluster\textunderscore name} dengan nilai "kind-testing-cluster-two-nodes"
\end{enumerate}

\textit{Request} dibuat dengan membuat request menggunakan Postman pada url /admin-api/v1/companies dengan metode POST. \textit{Request dan Response} dapat dilihat pada gambar \ref{fig:pengujian-p01}

Pengujian dengan ID P02 dilakukan dengan skenario yaitu admin tidak berhasil membuat \textit{company} karena nama cluster yang tidak tersedia pada server. Tersedia berarti, konfigurasi kubernetes cluster terdapat pada kubernetes \textit{config server} Admin akan membuat request dengan Postman kepada server dengan request seperti berikut.

\begin{enumerate}
  \item Mengisi \textit{field name} dengan nilai "new company"
  \item Mengisi \textit{field cluster\textunderscore name} dengan nilai "kind-testing-cluster-two-node"
\end{enumerate}

\textit{Request} dibuat dengan membuat request menggunakan Postman pada url /admin-api/v1/companies dengan metode POST. \textit{request dan response} dapat dilihat pada gambar


Pengujian dengan ID P03 dilakukan dengan skenario yaitu admin ingin mendapatkan seluruh \textit{company} yang terdaftar pada sistem. Admin akan membuat GET request dengan Postman ke url /admin-api/v1/companies. \textit{Response} balikan dari \textit{request} dapat dilihat pada gambar \ref{fig:pengujian-p03}

Pengujian dengan ID P04 dilakukan dengan skenario yaitu admin ingin menghapus \textit{company} dengan id tertentu dari daftar \textit{company} pada sistem. Admin akan membuat DELETE request dengan Postman ke url /admin-api/v1/companies/:id. Untuk itu dibuat sebuah \textit{company} baru dengan spesifikasi sebagai berikut.

\begin{enumerate}
  \item Mengisi \textit{field name} dengan nilai "new company"
  \item Mengisi \textit{field cluster\textunderscore name} dengan nilai "kind-psa-with-cluster-pss"
\end{enumerate}

Hasil pembuatan \textit{company} dapat dilihat pada gambar \ref{fig:pengujian-p04-1}. Admin menggunakan id company yang berhasil dibuat sesuai dengan gambar \ref{fig:pengujian-p04-1} sebagai parameter untuk \textit{company} yang dihapus. \textit{Request dan Response} dapat dilihat pada gambar \ref{fig:pengujian-p04}.

Seluruh rekap pengujian pada domain \textit{company} dapat dilihat pada tabel \ref{tab:pengujian-domain-company}. Berdasarkan hasil yang diperoleh, terbukti bahwa kebutuhan fungsional dengan ID F01, F02, dan F03 telah terimplementasi dengan baik.
