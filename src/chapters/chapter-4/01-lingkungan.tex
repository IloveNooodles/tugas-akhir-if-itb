\section{Lingkungan}
\label{sec:lingkungan-implementasi}
Pengembangan sistem \textit{remote deployment} diimplementasikan di lingkungan komputer lokal. Berikut merupakan penjelasan implementasi dari masing masing bagian secara terperinci yang terbagi menjadi dua bagian yaitu perangkat lunak dan perangkat keras.

Implementasi sistem tugas akhir dilakukan dengan mengimplementasikan dengan bantuan beberapa kakas pada bahasa \textit{golang},\textit{vue}, dan \textit{typescript}. Sistem hidup di luar \textit{kubernetes cluster} dan mengakses Kubernetes beserta \textit{pods}-nya melalui \textit{Kubernetes Client} dan \textit{dashboard} yang dapat diakses dari luar \textit{cluster} oleh pengguna. \textit{kubernetes cluster} dibuat dengan menggunakan \textit{kind}. Cluster disimulasikan memiliki 4 \textit{node} dengan spesifikasi 1 \textit{master node} dan 3 \textit{worker node}. Adapun spesifikasi dari komputer yang dipakai untuk pengembangan adalah sebagai berikut.
\begin{enumerate}
  \item \textbf{Perangkat Keras}

        \begin{enumerate}
          \item CPU: \textit{Apple M1 Chip}
          \item RAM: 8 GB
        \end{enumerate}

  \item \textbf{Perangkat Lunak}

        \begin{enumerate}
          \item Platform dan Sistem Operasi: Darwin Arm 64, MacOS Ventura 13.3
          \item \textit{Database}: Postgres 16.2-alpine3.19
          \item \textit{Containerization}: Docker desktop v26.0.0
          \item \textit{Kubernetes Cluster}:
                \begin{enumerate}
                  \item Kubernetes Client v1.30.0
                  \item K3s v1.29.4+k3s1
                  \item kind v0.22.0 go1.21.7 darwin/arm64
                \end{enumerate}
          \item Bahasa pemrogramman:
                \begin{enumerate}
                  \item Go 1.22.2 darwin/arm64
                  \item Vue 3
                  \item Typescript
                \end{enumerate}
          \item Dependensi Lain:
                \begin{enumerate}
                  \item \textit{Kubernetes Go-Client}
                  \item \textit{Echo, Logrus, Cobra, Viper, PQ, Sqlx, Go-migrate, Validator}
                  \item \textit{Nuxt, NuxtUI, pinia, Zod}
                \end{enumerate}
        \end{enumerate}
\end{enumerate}