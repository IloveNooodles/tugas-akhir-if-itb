\section{Implementasi}

Bagian ini akan menjelaskan tentang implementasi sistem \textit{remote deployment} secara terperinci. Seperti yang telah dijelaskan pada  \textbf{Bagian \ref{sec:rancangan-dashboard}} dan \textbf{\ref{sec:rancangan-service}} terdapat dua komponen utama yaitu \textit{dashboard} dan \textit{service}. Penjelasan bagian ini akan dimulai dari batasan implementasi, dilanjutkan dengan kakas yang digunakan dalam proses pembuatan sistem dan diakhiri dengan penjelasan mengenai implementasi dari \textit{dashboard} dan \textit{service}

\subsection{Batasan Implementasi}
Berikut adalah batasan yang ditetapkan dalam melakukan implementasi \textit{sistem remote deployment}.

\begin{enumerate}
  \item Semua batasan masalah dan konfigurasi yang telah dibahas pada bagian \ref{sec:batasan-masalah}.
  \item Kubernetes cluster berjalan di lokal dengan menggunakan kakas \textit{kind} dan hanya dibatasi menjadi 4 node dengan spesifikasi \textit{1 master} dan \textit{3 worker}
  \item \textit{Device} sudah terkoneksi sebelumnya sehingga tidak perlu register \textit{device} dan menghubungkannya ke dalam \textit{cluster}.
  \item \textit{Dashboard} hanya memilki fungsionalitas untuk \textit{user}
\end{enumerate}

\subsection{Kakas yang Digunakan}
Dalam melakukan implementasi ini diperlukan beberapa kakas, diantaranya adalah sebagai berikut.
\begin{enumerate}
  \item \textit{Docker}, \textit{Docker Desktop} dan \textit{Docker Desktop Kubernetes} untuk dipakai sebagai \textit{containerization} dan \textit{cluster} kubernetes lokal.
  \item Pandas dan Numpy untuk keperluan \textit{data processing} serta bentuk data untuk dikirimkan ke komponen lain serta model prediksi ARIMA.
  \item \textit{Kubernetes Python Client} untuk mengontrol \textit{cluster} kubernetes melalui kode Python.
  \item \textit{Pickle} untuk menyimpan model ARIMA sehingga persisten meskipun sistem di-\textit{restart}.
  \item \textit{Statsmodels} dan \textit{pmdarima} untuk membangun model ARIMA serta melakukan otomasi pencarian orde atau lebih dikenal sebagai Auto-ARIMA.
\end{enumerate}

\subsection{Persiapan \textit{kubernetes cluster}}

Tahapan ini merupakan tahapan persiaspan sebelum proses \textit{development}. Pada tahapan ini akan dibuat kubernetes \textit{cluster} pada komputer lokal dengan kakas \textit{kind}. \textit{Cluster} yang dibuat akan memiliki 4 nodes dengan spesifikasi 1 \textit{master node} dan 3 \textit{worker node}. Digunakan \textit{command} \textbf{kind create cluster --config cluster.yaml} dengan file konfigurasi yang dapat dilihat dibawah ini.

\begin{figure}[h]
  \centering
  \includegraphics[width=1\textwidth]{resources/appendix/pembuatan-cluster.jpg}
  \caption{konfigurasi pembuatan \textit{cluster} dengan kakas \textit{kind}}
  \label{fig:konfigurasi-pembuatan-cluster}
\end{figure}

\begin{figure}[h]
  \centering
  \includegraphics[width=1\textwidth]{resources/chapter-4/cluster-kind.jpg}
  \caption{Hasil \textit{cluster} dengan kakas \textit{kind}}
  \label{fig:hasil-cluster-kind}
\end{figure}

\pagebreak

\subsection{Implementasi \textit{dashboard}}
Penjelasan dashboard lalallaa

\subsubsection{Halaman \textit{Login}}
penjelasan halaman login

\begin{figure}[ht]
  \centering
  \includegraphics[width=1\textwidth]{resources/chapter-4/dashboard/login-page.jpg}
  \caption{Halaman Login}
  \label{fig:halaman-login}
\end{figure}

\subsubsection{Halaman utama}
penjelasan halaman utama

\subsubsection{Halaman \textit{Account}}
penjelasan halaman account

\begin{figure}
  \centering
  \includegraphics[width=1\textwidth]{resources/chapter-4/dashboard/account-page.jpg}
  \caption{Halaman \textit{account}}
  \label{fig:halaman-account}
\end{figure}

\subsubsection{Halaman \textit{Device}}

\begin{figure}
  \centering
  \includegraphics[width=1\textwidth]{resources/chapter-4/dashboard/device-page.jpg}
  \caption{Halaman \textit{device}}
  \label{fig:halaman-device}
\end{figure}

\begin{figure}
  \centering
  \includegraphics[width=1\textwidth]{resources/chapter-4/dashboard/device-page-add.jpg}
  \caption{Modal menambahkan \textit{device}}
  \label{fig:halaman-device-add}
\end{figure}

\subsubsection{Halaman \textit{Device detail}}

\begin{figure}
  \centering
  \includegraphics[width=1\textwidth]{resources/chapter-4/dashboard/device-detail-page.jpg}
  \caption{Halaman \textit{device detail}}
  \label{fig:halaman-device-detail}
\end{figure}

\begin{figure}
  \centering
  \includegraphics[width=1\textwidth]{resources/chapter-4/dashboard/device-detail-add-group.jpg}
  \caption{Modal menambahkan group pada \textit{device detail}}
  \label{fig:halaman-device-detail-add-group}
\end{figure}

\begin{figure}
  \centering
  \includegraphics[width=1\textwidth]{resources/chapter-4/dashboard/device-detail-delete.jpg}
  \caption{Modal menghapus device pada halaman \textit{device detail}}
  \label{fig:halaman-device-detail-delete}
\end{figure}

\pagebreak

\subsubsection{Halaman \textit{Groups}}

\begin{figure}
  \centering
  \includegraphics[width=1\textwidth]{resources/chapter-4/dashboard/groups-page.jpg}
  \caption{Halaman \textit{groups}}
  \label{fig:halaman-groups}
\end{figure}

\begin{figure}
  \centering
  \includegraphics[width=1\textwidth]{resources/chapter-4/dashboard/groups-page-add.jpg}
  \caption{Modal menambahkan \textit{groups}}
  \label{fig:halaman-groups-add}
\end{figure}

\pagebreak

\subsubsection{Halaman \textit{Groups detail}}

\begin{figure}
  \centering
  \includegraphics[width=1\textwidth]{resources/chapter-4/dashboard/groups-detail-page.jpg}
  \caption{Halaman \textit{groups detail}}
  \label{fig:halaman-groups-detail}
\end{figure}

\begin{figure}
  \centering
  \includegraphics[width=1\textwidth]{resources/chapter-4/dashboard/groups-detail-add-device.jpg}
  \caption{Modal menambahkan group pada \textit{groups detail}}
  \label{fig:halaman-groups-detail-add-group}
\end{figure}

\begin{figure}
  \centering
  \includegraphics[width=1\textwidth]{resources/chapter-4/dashboard/groups-detail-delete.jpg}
  \caption{Modal menghapus groups pada halaman \textit{groups detail}}
  \label{fig:halaman-groups-detail-delete}
\end{figure}

\pagebreak

\subsubsection{Halaman \textit{Deployment}}

\begin{figure}
  \centering
  \includegraphics[width=1\textwidth]{resources/chapter-4/dashboard/deployment-page.jpg}
  \caption{Halaman \textit{deployment}}
  \label{fig:halaman-deployment}
\end{figure}


\begin{figure}
  \centering
  \includegraphics[width=1\textwidth]{resources/chapter-4/dashboard/deployment-page-add-deployment.jpg}
  \caption{Modal menambahkan \textit{deployment}}
  \label{fig:halaman-deployment-add-deployment}
\end{figure}

\begin{figure}
  \centering
  \includegraphics[width=1\textwidth]{resources/chapter-4/dashboard/deployment-page-add-repostory.jpg}
  \caption{Modal menambahkan \textit{image} pada halaman \textit{deployment}}
  \label{fig:halaman-deployment-add-repostory}
\end{figure}

\pagebreak

\subsubsection{Halaman \textit{deployments detail}}

\begin{figure}
  \centering
  \includegraphics[width=1\textwidth]{resources/chapter-4/dashboard/deployment-detail-page.jpg}
  \caption{Halaman \textit{deployment detail}}
  \label{fig:halaman-deployment-detail}
\end{figure}

\begin{figure}
  \centering
  \includegraphics[width=1\textwidth]{resources/chapter-4/dashboard/deployment-detail-delete.jpg}
  \caption{Modal menghapus deployment pada halaman \textit{deployment detail}}
  \label{fig:halaman-deployment-detail-delete}
\end{figure}

\pagebreak

\subsubsection{Halaman \textit{history detail}}
Penjelasan halaman history detail

\subsubsection{Halaman \textit{FAQ}}
Penjelasan halaman faq

\subsection{Implementasi \textit{Service}}

Implementasi \textit{service} dibuat dengan menggunakan bahasa pemrogramman golang dan framework \textit{Echo}. Arsitektur kode yang dibuat memiliki tiga lapisan dimulai dari \textit{handler}, \textit{usecase}, dan \textit{repository}. Handler bertujuan membaca permintaan pengguna dan dapat disebut sebagai entrypoint. Data dari handler akan diberikan kepada \textit{usecase} untuk diproses. \textit{Usecase} merupakan lapisan yang hanya memiliki \textit{logic} proses bisnis. Setelah data berhasil melewati lapisan \textit{usecase}, data siap untuk dimasukkan ke database. Proses hubungan antara \textit{service} dengan \textit{database} diletakan pada lapisan \textit{repository}. 

Pemisahan lapisan ini mengikuti design pattern yaitu \textit{dependency injection}. Selain itu, pemisahan ini juga bertujuan memudahkan testing dan meningkatkan \textit{maintanability} karena mudah untuk dibaca dan dipahami.

\subsubsection{Domain \textit{company}}
\subsubsection{Domain \textit{user}}
\subsubsection{Domain \textit{devices}}
\subsubsection{Domain \textit{groups}}
\subsubsection{Domain \textit{deployment}}
\subsubsection{Domain \textit{external services}}




% \subsection{Persiapan \textit{Pods Elastic Search}}

% Sebelum melakukan implementasi, diperlukan untuk menyalakan \textit{Pods Elastic Search}. Konfigurasi ini dilakukan dengan cara membuat \textit{file deployment} untuk \textit{pods Elastic Search} serta sebuah \textit{persistent volume claim} untuk tempat penyimpanan data. Sebagai contoh dan konfigurasi yang dipakai dalam membuat tugas akhir ini dapat dilihat pada lampiran \ref{appendix:cth-konfigurasi-es-pods}.

% \subsection{Pengujian Komponen \textit{Metrics Fetcher}}

Pada bagian ini akan dijelaskan tentang tujuan, skenario, hasil, dan analisis dari pengujian komponen \textbf{\textit{Metrics Fetcher}}.

\subsubsection{Tujuan Pengujian}

Tujuan pengujian ini memastikan komponen \textbf{\textit{Metrics Fetcher}} dapat berjalan dengan baik dan menghasilkan data yang sesuai dengan ekspektasi.

\subsubsection{Skenario Pengujian}

Pengujian terhadap komponen \textbf{\textit{Metrics Fetcher}} dilakukan dengan beberapa skenario sebagai berikut serta ekspektasi dari pengujian yang dilakukan.
\begin{enumerate}
  \item \bfseries\textit{Elastic Search} sedang \textit{idle}.\normalfont

        Data yang diminta dari \textit{Node Stats API} diekspektasikan relatif statis dan berhasil diletakkan pada \textit{stream file}.
  \item \bfseries\textit{Elastic Search} sedang digunakan untuk melakukan operasi penambahan data.\normalfont

        Data yang diminta dari \textit{Node Stats API} seharusnya relatif berubah terutama pada aspek \textit{throughput} operasi \textit{index} dan \textit{bulk}. Lalu, data tersebut diekspektasikan berhasil diletakkan pada \textit{stream file}.

  \item \bfseries\textit{Elastic Search} sedang digunakan untuk melakukan operasi pencarian data.\normalfont

        Data yang diminta dari \textit{Node Stats API} seharusnya relatif berubah terutama pada aspek \textit{throughput} operasi \textit{query} dan \textit{fetch}. Lalu, data tersebut diekspektasikan berhasil diletakkan pada \textit{stream file}.
\end{enumerate}

\subsubsection{Hasil Pengujian dan Analisis}

Hasil untuk skenario 1 dapat dilihat pada gambar \ref{fig:mf-1}. Data yang ditarik sudah relatif statis untuk semua aspek dan berhasil diletakkan pada \textit{stream file}. Untuk skenario 2, dapat dilihat pada gambar \ref{fig:mf-2}. Data yang ditarik sudah mengalami perubahan pada operasi \textit{index} dan \textit{bulk} serta berhasil diletakkan pada \textit{stream file}. Terakhir, skenario 3, dapat dilihat pada gambar \ref{fig:mf-3}. Data yang ditarik sudah mengalami perubahan pada operasi \textit{query} dan \textit{fetch} serta berhasil diletakkan pada \textit{stream file}.

\begin{figure}[ht]
  \centering
  \includegraphics[width=0.8\textwidth]{chapter-4/mf-1.png}
  \caption{Hasil Pengujian Komponen \textit{Metrics Fetcher} Skenario 1}
  \label{fig:mf-1}
\end{figure}

\begin{figure}[ht]
  \centering
  \includegraphics[width=0.8\textwidth]{chapter-4/mf-2.png}
  \caption{Hasil Pengujian Komponen \textit{Metrics Fetcher} Skenario 2}
  \label{fig:mf-2}
\end{figure}

\begin{figure}[ht]
  \centering
  \includegraphics[width=0.8\textwidth]{chapter-4/mf-3.png}
  \caption{Hasil Pengujian Komponen \textit{Metrics Fetcher} Skenario 3}
  \label{fig:mf-3}
\end{figure}

Pengujian komponen \textbf{\textit{Metrics Fetcher}} sudah sesuai ekspektasi dan dapat dilanjutkan ke pengujian komponen lainnya.
% \subsection{Komponen \textit{Predictor}}

Selama program berjalan, data akan terus masuk karena \textbf{\textit{Metrics Fetcher}} terus menarik data setiap satuan waktu tertentu, oleh karena itu, komponen ini akan menyesuaikan model dengan data terbaru atau dengan kata lain melakukan \textit{retraining}. Pendekatan yang dibuat adalah dengan membuat sebuah \textit{offset} yang dapat dikonfigurasi, lihat bagian \ref{sec:komponen-pendukung}. Untuk menentukan waktu melakukan \textit{retraining}, maka akan dilakukan perhitungan dengan persamaan \ref{eq:retrain-time}. $\lambda(x)$ bisa disebut sebagai panjang data historis saat training untuk ke-$x$ kalinya dengan $c$ sebagai sebuah konstanta dan $p$ sebagai sebuah persentase. $c$ dan $p$ dapat diubah oleh pengguna sesuai kebutuhan melalui konfigurasi. Dengan kata lain, untuk melakukan training ke-$x$ kalinya, panjang data historis saat itu harus melebihi atau sama dengan $\lambda(x)$. Sebagai contoh, apabila \textit{training} pertama kali membutuhkan 100 data, dengan $c$ sejumlah $50$ dan $p$ sebesar $0.7$, maka \textit{training} kedua kali akan dilakukan ketika data historis sudah mencapai 120 data yang didapat dari $50+0.7*100$. Sebagai catatan, pada saat \textit{training} pertama kali, maka $\lambda(x)$ akan bernilai $c$, dan fitur prediksi tidak akan berjalan sampai \textit{training} dilakukan.

\begin{equation}
    \label{eq:retrain-time}
    \lambda(x) =
        \left\{
        \begin{array}{cl}
            c + p*\lambda(x-1) & : \ x \geq 1 \\
            0 & : \ x < 1
        \end{array}
        \right.
\end{equation}

Selain dari skema \textit{retraining} yang disebutkan sebelumnya, \textbf{\textit{Predict Component}} juga memiliki proses khusus terhadap data yang masuk dan berdampak ketika mengembalikan angka prediksi. Sebagai contoh kasusnya adalah sebagai berikut.
\begin{enumerate}
    \item Ketika mean dari data historis bernilai nol dan standar deviasi bernilai nol.
    \item Ketika data stasioner, yang berarti mean tidak nol namun standar deviasi bernilai nol.
    \item Kondisi normal, mean tidak nol dan standar deviasi tidak nol.
\end{enumerate}

Hal ini dilakukan untuk menghindari \textit{error} yang terjadi ketika melakukan \textit{training} dengan data yang bernilai nol ataupun yang bersifat statis. Proses khusus ini dapat direpresentasikan dengan persamaan yang dapat dilihat pada persamaan \ref{eq:predict-function}. Pada persamaan tersebut, $y(t)$ adalah hasil prediksi di waktu $t$, $ARIMA(t)$ adalah hasil prediksi dari model ARIMA di waktu $t$, $mean$ adalah nilai rata-rata dari data historis, dan $std$ adalah nilai standar deviasi dari data historis di. Pada persamaan tersebut, $mean$ akan selalu bernilai positif atau nol dikarenakan data yang digunakan adalah data yang bernilai positif seperti \textit{request time}, \textit{process time}, \textit{memory used percent}, \textit{cpu percent} dan \textit{load average}. $mean$ dan $std$ akan selalu diperbaharui setiap kali melakukan \textit{training}.

\begin{equation}
    \label{eq:predict-function}
    y(t) = \left\{
        \begin{array}{cl}
            ARIMA(t) & : \ mean > 0, std > 0 \\
            mean & : \ mean > 0, std = 0 \\
            NaN & : \ mean = 0, std = 0 \\
        \end{array}
    \right.
\end{equation}
% \subsection{Pengujian Komponen \textit{Rule Manager}}

Pada bagian ini akan dijelaskan tentang tujuan, skenario, hasil, dan analisis dari pengujian komponen \textbf{\textit{Rule Manager}}.

\subsubsection{Tujuan Pengujian}

Tujuan pengujian ini memastikan komponen \textbf{\textit{Rule Manager}} dapat berjalan dengan baik dan menghasilkan data yang sesuai dengan ekspektasi.

\subsubsection{Skenario Pengujian}

\textbf{\textit{Rule Manager}} adalah komponen \textit{low-level} yang hanya akan digunakan oleh komponen lain. Sehingga, untuk melakukan pengujian ini, akan dilakukan dengan pendekatan pembuatan \textit{test driver} khusus. Pengujian terhadap komponen \textbf{\textit{Rule Manager}} dilakukan dengan beberapa skenario sebagai berikut.
\begin{enumerate}
  \item \bfseries Pembuatan \textit{Rule}\normalfont

        Skenario bertujuan untuk memastikan \textit{parser} bekerja secara benar dan menghasilkan \textit{rule} yang sesuai dengan ekspektasi. Diekspektasikan \textit{rule} terkait inisiasi, prosesor, dan memori berhasil di\textit{parse} dan program berjalan tanpa rusak serta parameter yang dimasukkan sesuai dengan data uji coba.

  \item \bfseries Kebenaran \textit{Rule}\normalfont

        Skenario ditujukan untuk membuktikan bahwa \textit{rule} tetap berjalan pada kondisi simpleks maupun kompleks dan mengeluarkan \textit{output} yang sesuai saat di panggil.

  \item \bfseries Deklarasi Variabel dan Pengaksesan Variabel\normalfont

        Skenario mencakup fungsi \textit{rule} yang dapat mendeklarasikan variabel baru dan mengaksesnya pada \textit{rule} lain dan juga mengubahnya.
\end{enumerate}

\subsubsection{Hasil Pengujian dan Analisis}

Pengujian dilakukan dengan dua buah program \textit{test driver} dan dua buah \textit{file rule}. \textit{Driver} pertama, lihat gambar \ref{fig:rm-driver-1}, berguna untuk memastikan tidak ada eror saat melakukan parse. Untuk kedua buah file \textit{rule} bisa dilihat pada gambar \ref{fig:rm-rule-1} dan \ref{fig:rm-rule-2}. \textit{Rule} pertama berfungsi untuk kasus normal sedangkan yang kedua untuk kasus tidak valid. Hasil dari percobaan ini dapat dilihat pada gambar \ref{fig:rm-1} dan \ref{fig:rm-2}. Hasil dari percobaan ini sesuai dengan ekspektasi dan memenuhi percobaan skenario 1. Sedangkan, untuk menguji percobaan ke 2 dan 3, akan digunakan \textit{driver} ke dua, lihat gambar \ref{fig:rm-driver-2}. Karena komponen ini bergantung pada data prediksi, maka pada program tersebut, data di-\textit{mock} melalui \textit{input} pengguna. \textit{File rule} yang dipakai dapat dilihat pada gambar \ref{fig:rm-rule-1}. Hasil dari percobaan ini dapat dilihat pada gambar \ref{fig:rm-3-1} dilanjutkan dengan gambar \ref{fig:rm-3-2}. Hasil dari percobaan ini sesuai dengan ekspektasi dan memenuhi percobaan skenario 2 dan 3.

\begin{figure}[h]
  \centering
  \includegraphics[width=0.5\textwidth]{chapter-4/rm-driver-1.png}
  \caption{Program \textit{Test Driver 1} untuk Menguji \textit{Rule Manager}}
  \label{fig:rm-driver-1}
\end{figure}

\begin{figure}[h]
  \centering
  \includegraphics[width=0.8\textwidth]{chapter-4/rm-driver-2.png}
  \caption{Program \textit{Test Driver 2} untuk Menguji \textit{Rule Manager}}
  \label{fig:rm-driver-2}
\end{figure}

\begin{figure}[h]
  \centering
  \includegraphics[width=1\textwidth]{chapter-4/rm-rule-1.png}
  \caption{\textit{File Rule 1} untuk Pengujian \textit{Rule Manager}}
  \label{fig:rm-rule-1}
\end{figure}

\begin{figure}[h]
  \centering
  \includegraphics[width=1\textwidth]{chapter-4/rm-rule-2.png}
  \caption{\textit{File Rule 2} untuk Pengujian \textit{Rule Manager}}
  \label{fig:rm-rule-2}
\end{figure}

\begin{figure}[h]
  \centering
  \includegraphics[width=1\textwidth]{chapter-4/rm-1.png}
  \caption{Hasil Pengujian Skenario 1 (1/2)}
  \label{fig:rm-1}
\end{figure}

\begin{figure}[h]
  \centering
  \includegraphics[width=1\textwidth]{chapter-4/rm-2.png}
  \caption{Hasil Pengujian Skenario 1 (2/2)}
  \label{fig:rm-2}
\end{figure}

\begin{figure}[h]
  \centering
  \includegraphics[width=1\textwidth]{chapter-4/rm-3-1.png}
  \caption{Hasil Pengujian Skenario 2 dan 3 (1/2)}
  \label{fig:rm-3-1}
\end{figure}

\begin{figure}[h]
  \centering
  \includegraphics[width=1\textwidth]{chapter-4/rm-3-2.png}
  \caption{Hasil Pengujian Skenario 2 dan 3 (2/2)}
  \label{fig:rm-3-2}
\end{figure}

Pengujian komponen \textbf{\textit{Rule Manager}} sudah sesuai ekspektasi dan dapat dilanjutkan ke pengujian komponen lainnya.
% \subsection{Komponen \textit{Resource Controller}}

Seperti yang sudah dirancangkan sebelumnya, kelas ini menggunakan \textit{Kubernetes Client API} untuk mengubah alokasi sumber daya. Diimplementasikan dengan sistem antrian, sehingga jika sejumlah rule aktif secara bersamaan, maka akan dijalankan secara berurutan. Terdapat sebuah fungsi \textit{tick} yang akan berfungsi untuk mengeksekusi antrian. Contoh simpanan file antrian dapat dilihat pada gambar \ref{fig:ex-queue-rc}. File tersebut menyimpan status alokasi sumber daya pada saat itu, kapan melakukan perubahan pada antrian berikutnya dalam waktu UNIX dan antrian yang akan dieksekusi satu per satu.

\begin{figure}[h]
    \centering
    \includegraphics[width=0.45\textwidth]{chapter-4/rc-queue-ex.png}
    \caption{Contoh File Antrian Pengubahan Alokasi}
    \label{fig:ex-queue-rc}
\end{figure}

% TODO CONTOH SISTEM ANTRIAN
% % \subsection{Pengujian Sistem \textit{Flexible Control}}

Pada bagian ini akan dijelaskan tentang tujuan, skenario, hasil, dan analisis dari pengujian sistem sekaligus komponen \textbf{\textit{Flexible Control}}.

\subsubsection{Tujuan Pengujian}

Tujuan pengujian ini memastikan sistem \textbf{\textit{Flexible Control}} dapat berjalan dengan baik dan menghasilkan perilaku yang sesuai.

\subsubsection{Skenario Pengujian}

Pengujian terhadap komponen \textbf{\textit{Flexible Control}} dilakukan dengan beberapa skenario sebagai berikut serta ekspektasi dari pengujian yang dilakukan.
\begin{enumerate}
    \item \bfseries Sebuah \textit{rule} memenuhi kondisi untuk mengubah alokasi prosesor.\normalfont
    
    Prosesor akan berubah jumlahnya sesuai dengan \textit{rule} yang memenuhi kondisi. Perubahan pada spesifikasi \textit{pods} juga diekspektasikan mengikuti.

    \item \bfseries Sebuah \textit{rule} memenuhi kondisi untuk mengubah alokasi memori.\normalfont
    
    Prosesor akan berubah jumlahnya sesuai dengan \textit{rule} yang memenuhi kondisi. Perubahan pada spesifikasi \textit{pods} juga diekspektasikan mengikuti. Memory Used Percent akan menurun karena penambahan yang terjadi.
\end{enumerate}

\subsubsection{Hasil Pengujian dan Analisis}

Pengujian akan dilakukan dengan \textit{file rule} yang dapat dilihat pada gambar \ref{fig:ac-rule}. Terdapat dua buah \textit{rule} yang akan diuraikan sebagai berikut.

\begin{enumerate}
    \item Jika \textit{load average 1m} pada 10 detik kedepan diprediksikan diatas 0 maka akan ditambah alokasi prosesor sebesar 1000m atau sejumlah 1. Kondisi dari \textit{rule} sengaja dibuat seperti itu agar rule pasti terpenuhi.
    \item Jika \textit{memory used percent} pada 5 dan 10 detik kedepan diprediksikan diatas 60 maka akan ditambah alokasi memori sebesar 2048 mebibyte atau sejumlah 2 gibibyte (Gi). Kondisi dari \textit{rule} sengaja dibuat seperti itu agar rule pasti terpenuhi.
\end{enumerate}

Hasil dari pengujian skenario kedua dapat dilihat pada gambar \ref{fig:ac-mem}. Dan perubahan terhadap spesifikasi pods dapat dilihat pada gambar \ref{fig:ac-mem-kube}. Perubahan juga terjadi pada \textit{memory used percent} pada \textit{stream file} atau data yang ditarik oleh komponen \textbf{\textit{Metrics Fetcher}} dapat dilihat pada gambar \ref{fig:ac-mf-turun}.
Diikuti dengan hasil dari pengujian skenario pertama dapat dilihat pada gambar \ref{fig:ac-cpu}. Dapat dilihat bahwa prosesor berubah sesuai dengan ekspektasi. Perubahan pada spesifikasi \textit{pods} juga mengikuti perubahan prosesor yang dapat dilihat pada gambar \ref{fig:ac-cpu-kube}.

\begin{figure}[h]
    \centering
    \includegraphics[width=0.8\textwidth]{chapter-4/ac-cpu.png}
    \caption{Hasil Pengujian Komponen \textit{Flexible Control} Skenario 1: Perubahan Prosesor}
    \label{fig:ac-cpu}
\end{figure}

\begin{figure}[h]
    \centering
    \includegraphics[width=0.8\textwidth]{chapter-4/ac-cpu-kube.png}
    \caption{Hasil Pengujian Komponen \textit{Flexible Control} Skenario 1: Perubahan Spesifikasi Kubernetes}
    \label{fig:ac-cpu-kube}
\end{figure}

\begin{figure}[h]
    \centering
    \includegraphics[width=0.8\textwidth]{chapter-4/ac-rule.png}
    \caption{File Rule untuk Pengujian Komponen \textit{Flexible Control}}
    \label{fig:ac-rule}
\end{figure}

\begin{figure}[h]
    \centering
    \includegraphics[width=1\textwidth]{chapter-4/ac-mem.png}
    \caption{Hasil Pengujian Komponen \textit{Flexible Control} Skenario 2: Perubahan Memori}
    \label{fig:ac-mem}
\end{figure}

\begin{figure}[h]
    \centering
    \includegraphics[width=0.8\textwidth]{chapter-4/ac-mem-kube.png}
    \caption{Hasil Pengujian Komponen \textit{Flexible Control} Skenario 2: Perubahan Spesifikasi Kubernetes}
    \label{fig:ac-mem-kube}
\end{figure}

\begin{figure}[h]
    \centering
    \includegraphics[width=0.8\textwidth]{chapter-4/ac-mf-turun.png}
    \caption{Hasil Pengujian Komponen \textit{Flexible Control} Skenario 2: Perubahan Memory Used Percent pada \textit{stream file}}
    \label{fig:ac-mf-turun}
\end{figure}

Pengujian komponen \textbf{\textit{Flexible Control}} sudah sesuai ekspektasi dan sistem dapat berjalan dengan baik.
% \subsection{Komponen Pendukung: Konfigurasi dan Utilitas}
\label{sec:komponen-pendukung}

Seperti yang sudah dijelaskan sebelumnya, terdapat konfigurasi yang dapat mengatur sistem. Konfigurasi yang dapat diatur dapat dilihat pada gambar \ref{fig:config-spek}.

\begin{figure}[h]
    \centering
    \includegraphics[width=0.8\textwidth]{chapter-4/config.png}
    \caption{Konfigurasi \textit{Flexible Control}}
    \label{fig:config-spek}
\end{figure}

Setiap konfigurasi tersebut mengatur perilaku dari sistem. Untuk setiap konfigurasinya, berikut adalah penjelasannya.

\begin{enumerate}
    \item \textbf{\textit{Debug}} dan \textbf{\textit{Warning}}
    
    Kedua \textit{flag} ini adalah untuk mematikan dan menyalakan pesan \textit{debug} dan \textit{warning}. Jika \textit{debug} dimatikan, maka program tidak akan mengirimkan pesan apapun selama berjalan.

    \item \textbf{\textit{Timezone}}
    
    Flag ini bertujuan untuk mengubah zona waktu yang digunakan oleh pandas karena data yang didapatkan dari \textit{Elastic Search} adalah berupa unix time sehingga akan dibaca secara \textit{default} menjadi UTC saat dikonversi.

    \item \textbf{\textit{Node Name}}, \textbf{\textit{Namespace}} dan \textbf{\textit{Deployment Name}}
    
    \textbf{\textit{Node Name}} adalah nama \textit{node} yang telah dikonfigurasi pada \textit{pods Elastic Search}. Nama harus sesuai karena \textbf{\textit{Metrics Fetcher}} akan mencari data untuk node dengan nama tersebut. Sedangkan, \textbf{\textit{Namespace}} dan \textbf{\textit{Deployment Name}} berkaitan dengan \textit{namespace} dan \textit{deployment Elasticsearch} dengan Kubernetes.

    \item \textbf{\textit{Elasticsearch Host}}
    
    \textit{Flag} ini berisikan target \textit{host} dari \textit{Elasticsearch}. Bertindak sebagai \textit{Base URL} untuk mengakses API \textit{Elastic Search}.

    \item \textbf{\textit{CPU Limit}} dan \textbf{\textit{Memory Limit}}
    
    Kedua limit ini digunakan untuk \textbf{\textit{Resource Controller}} mengubah alokasi sumber daya. \textit{Flag} ini berisikan \textit{tuple} dengan dua buah angka yang berguna sebagai batas bawah dan batas atas dari sumber daya bersangkutan. Satuan yang digunakan untuk prosesor adalah mili (m) sedangkan untuk memori adalah \textit{mebibyte} (MiB). Kedua batas ini bersifat inklusif.

    \item \textbf{\textit{File Path}}
    
    Seperti namanya, konfigurasi yang berkaitan dengan \textit{file path} berfungsi untuk mengatur tata letak file yang akan dibuat/dibaca oleh sistem.

    \item \textbf{\textit{Fetcher Interval}}, \textbf{\textit{Resource Change Cooldown}} dan \textbf{\textit{Data Update Tick Second}}
    
    \textbf{\textit{Fetcher Interval}} adalah interval komponen \textbf{\textit{Metrics Fetcher}} melakukan penarikan data. Lalu, \textbf{\textit{Resource Change Cooldown}} adalah waktu yang diperlukan oleh \textbf{\textit{Resource Controller}} untuk menunggu sebelum melakukan perubahan sumber daya. Terakhir, \textbf{\textit{Data Update Tick Second}} adalah interval yang digunakan oleh \textbf{\textit{Flexible Control}} untuk melakukan pembacaan data dari \textit{stream file}. \textbf{\textit{Data Update Tick Second}} harus lebih besar sama dengan \textbf{\textit{Fetcher Interval}} agar efisien. Satuan yang digunakan oleh ketiga \textit{flag} tersebut adalah detik.

    \item \textbf{\textit{Save Model}}
    
    \textit{Flag} ini berfungsi untuk mematikan penyimpanan model setiap kali model berubah. Jika \textit{flag} ini tidak dinyalakan, maka setiap kali sistem melakukan \textit{restart}, model prediksi akan diulang dari kosong.

    \item \textbf{\textit{Offset Treshold Retrain}} dan \textbf{\textit{Percent Retrain}}
    
    Dalam melakukan penambahan data, tidak setiap saat model akan di-\textit{retrain}. Saat tidak di-\textit{retrain}, model prediksi hanya melakukan update yang jauh lebih cepat namun tidak terlalu akurat. Terdapat sebuah angka yang akan menentukan kapan model harus di-\textit{retrain}. Hal ini diperlukan karena melakukan \textit{retrain} membutuhkan waktu yang lama terutama saat data sudah sangat besar. \textbf{\textit{Offset Treshold Retrain}} adalah angka yang menentukan kapan model harus di-\textit{retrain} berdasarkan jumlah data fixed. Sedangkan, \textbf{\textit{Percent Retrain}} adalah angka yang menentukan kapan model harus di-\textit{retrain} berdasarkan persentase jumlah data saat itu. Dengan persamaan \ref{eq:retrain-time}, \textbf{\textit{Offset Treshold Retrain}} adalah variable $c$ dan \textbf{\textit{Percent Retrain}} adalah variable $p$.
\end{enumerate}

Terdapat juga fungsi-fungsi utilitas yang akan membantu komponen-komponen yang telah dijelaskan sebelumnya, spesifikasi utilitas bisa dilihat pada gambar \ref{fig:util-spek}. Untuk setiap fungsinya, berikut adalah kegunaannya.

\begin{enumerate}
    \item \textbf{\textit{Save Model}}
    
    Fungsi ini akan menyimpan model yang telah dilatih ke dalam sebuah file. Digunakan kakas \textit{pickle} untuk melakukan hal ini.

    \item \textbf{\textit{Load Model}}
    
    Fungsi ini akan memuat model yang telah dilatih dari sebuah file. Digunakan kakas \textit{pickle} untuk melakukan hal ini.

    \item \textbf{\textit{Timings}}
    
    Fungsi adalah abstraksi untuk menghitung waktu eksekusi. Digunakan fungsi sebagai \textit{return value} agar lebih rapih ketika diperlukan banyak penghitungan waktu eksekusi.

    \item \textbf{\textit{Printd}}
    
    Fungsi ini hanyalah \textit{wrapper} dari fungsi \textit{print} pada Python untuk mengikuti aturan konfigurasi.

    \item \textbf{\textit{Read From File}}
    
    Fungsi ini digunakan untuk membaca \textit{stream file}. Fungsi ini digunakan oleh komponen \textbf{\textit{Flexible Control}} untuk membaca secara periodik, mentranformasikan dan mengirimkan data ke \textbf{\textit{Predict Component Storage}}.

    \item \textbf{\textit{To Vector}}
    
    Fungsi ini adalah fungsi transformasi format JSON (\textit{Java Syntax Object Notation}) yang ditulis ke \textit{stream file} menjadi sebuah \textit{numpy array} yang akan digunakan untuk membuat \textit{pandas dataframe}.

    \item \textbf{\textit{Create Dataframe}}
    
    Seperti namanya, fungsi ini membuat dataframe dari data yang telah dibaca dari \textit{stream file} dan sudah ditransformasikan dengan fungsi \textbf{\textit{to vector}}.

    \item \textbf{\textit{Extract Number From String}}
    
    Fungsi ini memanfaatkan regex untuk mengambil angka dari sebuah string.
\end{enumerate}

\begin{figure}[h]
    \centering
    \includegraphics[width=0.6\textwidth]{chapter-4/utils.png}
    \caption{Spesifikasi Fungsi Utilitas Pendukung}
    \label{fig:util-spek}
\end{figure}

% \subsection{Penggunaan untuk \textit{Pods} dengan Aplikasi Lain}

% Sistem yang diimplementasikan juga dapat digunakan untuk pods lainnya. Tentunya dengan mengubah konfigurasi serta membuat \textbf{\textit{Metrics Fetcher}} khusus untuk aplikasi tersebut. Berikut adalah \textit{requirement} untuk dapat digunakan pada \textit{pods} lain.

% \begin{enumerate}
%   \item Aplikasi tersebut harus memiliki \textit{metrics} yang dapat diambil melalui suatu API, dapat berbentuk HTTP, GRPC atau yang lainnya. \label{item:requirement-general-1}
%   \item Aplikasi harus bisa mempunyai komponen untuk menghitung \textit{throuhgput} untuk setiap operasinya. \label{item:requirement-general-2}
% \end{enumerate}

% Ketika poin \ref{item:requirement-general-1} dan \ref{item:requirement-general-2} sudah terpenuhi, maka langkah-langkah yang harus dilakukan adalah sebagai berikut.

% \begin{enumerate}
%   \item Membuat komponen \textbf{\textit{Metrics Fetcher}} yang baru untuk aplikasi tersebut.

%         Dalam melakukan hal ini, transformasi sebaiknya diusahakan disesuaikan dengan struktur data yang sudah ada. Namun, jika ingin mengubah, maka harus melakukan pengubahan ekstra pada fungsi pemetaan \textit{metrics} yang ada pada komponen \textbf{\textit{Predictor}}.
%   \item Menyesuaikan konfigurasi dan fungsi pemetaan komponen \textbf{\textit{Predictor}}.

%         Tentunya, nama operasi akan berbeda serta detil-detil utilisasi umum akan berbeda sehingga diperlukan untuk mengubah dan menyesuaikan pada file konfigurasi.

%   \item Mengubah \textit{label selector kubernetes} pada komponen \textbf{\textit{Resource Controller}}

%         Karena sebelumnya dispesifikan ke aplikasi \textit{Elastic Search}, akibatnya, untuk mengubah ke aplikasi lain, perlu melakukan pengubahan \textit{label selector}.
% \end{enumerate}

% Setelah melakukan pengubahan tersebut, sistem sudah dapat digunakan untuk aplikasi lainnya.