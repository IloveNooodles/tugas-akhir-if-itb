\section{Implementasi}

Bagian ini menjelaskan tentang implementasi PERISAI secara terperinci. Seperti yang telah dijelaskan pada bagian \ref{sec:rancangan-dashboard} dan \ref{sec:rancangan-service} terdapat dua komponen utama yaitu \textit{dashboard} dan \textit{service}. Penjelasan bagian ini dimulai dari batasan implementasi, dilanjutkan dengan kakas yang digunakan dalam proses pembuatan sistem dan diakhiri dengan penjelasan mengenai implementasi dari \textit{dashboard} dan \textit{service}.

Karena PERISAI berada pada dunia IoT dan Kubernetes, perlu dilakukan pencocokan terminologi untuk memudahkan pengembangan sistem. Istilah \textit{node} pada kuberntes akan berpadanan dengan perangkat di IoT pada bagian "thing". \textit{eployment} pada kubernetes berpadanan dengan aplikasi yang dijalankan pada "thing". Kapabilitas serta tipe dari perangkat pada IoT akan dibuat menjadi label pada kubernetes.

\subsection{Batasan Implementasi}
Berikut adalah batasan yang ditetapkan dalam melakukan implementasi \textit{sistem remote deployment}.

\begin{enumerate}
  \item Semua batasan masalah dan konfigurasi yang telah dibahas pada bagian \ref{sec:batasan-masalah}.
  \item Kubernetes cluster berjalan di lokal dengan menggunakan kakas \textit{kind} dan hanya dibatasi menjadi 4 node dengan spesifikasi \textit{1 master} dan \textit{3 worker}
  \item \textit{Device} sudah terkoneksi sebelumnya sehingga tidak perlu register \textit{device} dan menghubungkannya ke dalam \textit{cluster}.
  \item \textit{Dashboard} hanya memilki fungsionalitas untuk \textit{user}
\end{enumerate}

\subsection{Kakas yang Digunakan}
Dalam melakukan implementasi ini diperlukan beberapa kakas, diantaranya adalah sebagai berikut.
\begin{enumerate}
  \item \textit{Docker}, \textit{Docker Desktop} dan \textit{Docker Desktop Kubernetes} untuk dipakai sebagai \textit{containerization} dan \textit{cluster} kubernetes lokal.
  \item Pandas dan Numpy untuk keperluan \textit{data processing} serta bentuk data untuk dikirimkan ke komponen lain serta model prediksi ARIMA.
  \item \textit{Kubernetes Python Client} untuk mengontrol \textit{cluster} kubernetes melalui kode Python.
  \item \textit{Pickle} untuk menyimpan model ARIMA sehingga persisten meskipun sistem di-\textit{restart}.
  \item \textit{Statsmodels} dan \textit{pmdarima} untuk membangun model ARIMA serta melakukan otomasi pencarian orde atau lebih dikenal sebagai Auto-ARIMA.
\end{enumerate}

\subsection{Persiapan \textit{kubernetes cluster}}

Tahapan ini merupakan tahapan persiaspan sebelum proses \textit{development}. Pada tahapan ini akan dibuat kubernetes \textit{cluster} pada komputer lokal dengan kakas \textit{kind}. \textit{Cluster} yang dibuat akan memiliki 4 nodes dengan spesifikasi 1 \textit{master node} dan 3 \textit{worker node}. Digunakan \textit{command} \textbf{kind create cluster --config cluster.yaml} dengan file konfigurasi yang dapat dilihat dibawah ini.

\begin{figure}[h]
  \centering
  \includegraphics[width=1\textwidth]{resources/appendix/pembuatan-cluster.jpg}
  \caption{konfigurasi pembuatan \textit{cluster} dengan kakas \textit{kind}}
  \label{fig:konfigurasi-pembuatan-cluster}
\end{figure}

\begin{figure}[h]
  \centering
  \includegraphics[width=1\textwidth]{resources/chapter-4/cluster-kind.jpg}
  \caption{Hasil \textit{cluster} dengan kakas \textit{kind}}
  \label{fig:hasil-cluster-kind}
\end{figure}

\pagebreak

\subsection{Implementasi \textit{dashboard}}
Penjelasan dashboard lalallaa

\subsubsection{Halaman \textit{Login}}
penjelasan halaman login

\begin{figure}[ht]
  \centering
  \includegraphics[width=1\textwidth]{resources/chapter-4/dashboard/login-page.jpg}
  \caption{Halaman Login}
  \label{fig:halaman-login}
\end{figure}

\subsubsection{Halaman utama}
penjelasan halaman utama

\subsubsection{Halaman \textit{Account}}
penjelasan halaman account

\begin{figure}
  \centering
  \includegraphics[width=1\textwidth]{resources/chapter-4/dashboard/account-page.jpg}
  \caption{Halaman \textit{account}}
  \label{fig:halaman-account}
\end{figure}

\subsubsection{Halaman \textit{Device}}

\begin{figure}
  \centering
  \includegraphics[width=1\textwidth]{resources/chapter-4/dashboard/device-page.jpg}
  \caption{Halaman \textit{device}}
  \label{fig:halaman-device}
\end{figure}

\begin{figure}
  \centering
  \includegraphics[width=1\textwidth]{resources/chapter-4/dashboard/device-page-add.jpg}
  \caption{Modal menambahkan \textit{device}}
  \label{fig:halaman-device-add}
\end{figure}

\subsubsection{Halaman \textit{Device detail}}

\begin{figure}
  \centering
  \includegraphics[width=1\textwidth]{resources/chapter-4/dashboard/device-detail-page.jpg}
  \caption{Halaman \textit{device detail}}
  \label{fig:halaman-device-detail}
\end{figure}

\begin{figure}
  \centering
  \includegraphics[width=1\textwidth]{resources/chapter-4/dashboard/device-detail-add-group.jpg}
  \caption{Modal menambahkan group pada \textit{device detail}}
  \label{fig:halaman-device-detail-add-group}
\end{figure}

\begin{figure}
  \centering
  \includegraphics[width=1\textwidth]{resources/chapter-4/dashboard/device-detail-delete.jpg}
  \caption{Modal menghapus device pada halaman \textit{device detail}}
  \label{fig:halaman-device-detail-delete}
\end{figure}

\pagebreak

\subsubsection{Halaman \textit{Groups}}

\begin{figure}
  \centering
  \includegraphics[width=1\textwidth]{resources/chapter-4/dashboard/groups-page.jpg}
  \caption{Halaman \textit{groups}}
  \label{fig:halaman-groups}
\end{figure}

\begin{figure}
  \centering
  \includegraphics[width=1\textwidth]{resources/chapter-4/dashboard/groups-page-add.jpg}
  \caption{Modal menambahkan \textit{groups}}
  \label{fig:halaman-groups-add}
\end{figure}

\pagebreak

\subsubsection{Halaman \textit{Groups detail}}

\begin{figure}
  \centering
  \includegraphics[width=1\textwidth]{resources/chapter-4/dashboard/groups-detail-page.jpg}
  \caption{Halaman \textit{groups detail}}
  \label{fig:halaman-groups-detail}
\end{figure}

\begin{figure}
  \centering
  \includegraphics[width=1\textwidth]{resources/chapter-4/dashboard/groups-detail-add-device.jpg}
  \caption{Modal menambahkan group pada \textit{groups detail}}
  \label{fig:halaman-groups-detail-add-group}
\end{figure}

\begin{figure}
  \centering
  \includegraphics[width=1\textwidth]{resources/chapter-4/dashboard/groups-detail-delete.jpg}
  \caption{Modal menghapus groups pada halaman \textit{groups detail}}
  \label{fig:halaman-groups-detail-delete}
\end{figure}

\pagebreak

\subsubsection{Halaman \textit{Deployment}}

\begin{figure}
  \centering
  \includegraphics[width=1\textwidth]{resources/chapter-4/dashboard/deployment-page.jpg}
  \caption{Halaman \textit{deployment}}
  \label{fig:halaman-deployment}
\end{figure}


\begin{figure}
  \centering
  \includegraphics[width=1\textwidth]{resources/chapter-4/dashboard/deployment-page-add-deployment.jpg}
  \caption{Modal menambahkan \textit{deployment}}
  \label{fig:halaman-deployment-add-deployment}
\end{figure}

\begin{figure}
  \centering
  \includegraphics[width=1\textwidth]{resources/chapter-4/dashboard/deployment-page-add-repostory.jpg}
  \caption{Modal menambahkan \textit{image} pada halaman \textit{deployment}}
  \label{fig:halaman-deployment-add-repostory}
\end{figure}

\pagebreak

\subsubsection{Halaman \textit{deployments detail}}

\begin{figure}
  \centering
  \includegraphics[width=1\textwidth]{resources/chapter-4/dashboard/deployment-detail-page.jpg}
  \caption{Halaman \textit{deployment detail}}
  \label{fig:halaman-deployment-detail}
\end{figure}

\begin{figure}
  \centering
  \includegraphics[width=1\textwidth]{resources/chapter-4/dashboard/deployment-detail-delete.jpg}
  \caption{Modal menghapus deployment pada halaman \textit{deployment detail}}
  \label{fig:halaman-deployment-detail-delete}
\end{figure}

\pagebreak

\subsubsection{Halaman \textit{history detail}}
Penjelasan halaman history detail

\subsubsection{Halaman \textit{FAQ}}
Penjelasan halaman faq

\subsection{Implementasi \textit{Service}}

Implementasi \textit{service} dibuat dengan menggunakan bahasa pemrogramman golang dan framework \textit{Echo}. Arsitektur kode yang dibuat memiliki tiga lapisan dimulai dari \textit{handler}, \textit{usecase}, dan \textit{repository}. Handler bertujuan membaca permintaan pengguna dan dapat disebut sebagai entrypoint. Data dari handler akan diberikan kepada \textit{usecase} untuk diproses. \textit{Usecase} merupakan lapisan yang hanya memiliki \textit{logic} proses bisnis. Setelah data berhasil melewati lapisan \textit{usecase}, data siap untuk dimasukkan ke database. Proses hubungan antara \textit{service} dengan \textit{database} diletakan pada lapisan \textit{repository}. 

Pemisahan lapisan ini mengikuti design pattern yaitu \textit{dependency injection}. Selain itu, pemisahan ini juga bertujuan memudahkan testing dan meningkatkan \textit{maintanability} karena mudah untuk dibaca dan dipahami.

\subsubsection{Domain \textit{company}}
\subsubsection{Domain \textit{user}}
\subsubsection{Domain \textit{devices}}
\subsubsection{Domain \textit{groups}}
\subsubsection{Domain \textit{deployment}}
\subsubsection{Domain \textit{external services}}