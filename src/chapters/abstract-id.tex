\clearpage
\chapter*{ABSTRAK}
\addcontentsline{toc}{chapter}{Abstrak}
\begin{center}
  \center
  \begin{singlespace}
    \large\bfseries\MakeUppercase{\thetitle}
    
    \normalfont\normalsize
    Oleh:
    
    \bfseries \theauthor
  \end{singlespace}
\end{center}

\begin{singlespace}
  \small
  Dengan semakin meningkatnya jumlah pengguna perangkat \textit{Internet of Things (IoT)}, manajemen perangkat menjadi semakin kompleks dan menantang. Pengelolaan perangkat IoT yang efisien adalah kunci untuk memastikan keandalan dan kinerja optimal dari jaringan yang semakin berkembang. Awalnya, pembaruan perangkat lunak pada perangkat IoT dilakukan secara serial, yang sangat memakan waktu dan tidak efisien. Seiring dengan perkembangan teknologi, metode \textit{Over The Air} (OTA) mulai diperkenalkan, memungkinkan pembaruan perangkat lunak dilakukan secara \textit{remote}. Namun, meskipun metode OTA telah membawa banyak perbaikan, tantangan dalam manajemen perangkat yang skalabel dan efisien tetap ada.
  
  Penelitian ini bertujuan untuk mengeksplorasi penggunaan Kubernetes sebagai solusi untuk pengelolaan perangkat \textit{IoT} secara \textit{remote}. Kubernetes, yang telah terbukti efektif dalam manajemen aplikasi container di lingkungan \textit{cloud}, memiliki potensi besar untuk diterapkan dalam ekosistem \textit{IoT}. Namun, hingga saat ini, penelitian yang mengevaluasi penggunaan Kubernetes untuk manajemen perangkat \textit{IoT} masih sangat terbatas. 
  
  Implementasi \textit{remote deployment} dengan kubernetes berhasil dibuat dengan baik sehingga dapat meningkatkan proses pengelolaan perangkat \textit{IoT} menjadi lebih efisien serta dapat dilakukan dengan sangat mudah.
  
  \textbf{\textit{Kata kunci: IoT, Remote Deployment, Kubernetes }}
  
\end{singlespace}
\clearpage