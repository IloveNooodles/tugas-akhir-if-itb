\clearpage
\chapter*{ABSTRAK}
\addcontentsline{toc}{chapter}{Abstrak}
\begin{center}
  \center
  \begin{singlespace}
    \large\bfseries\MakeUppercase{\thetitle}
    
    \normalfont\normalsize
    Oleh:
    
    \bfseries \theauthor
  \end{singlespace}
\end{center}

\begin{singlespace}
  \small
  Dalam revolusi teknologi informasi yang terus berkembang, konsep "\textit{Smart Home}" telah menjadi pusat perhatian sebagai cara yang inovatif dan efisien untuk mengelola berbagai aspek pada rumah. Namun, dalam upaya untuk mencapai \textit{smart home} yang sepenuhnya terintegrasi, terdapat beberapa permasalahan. Salah satu permasalahan yang umum adalah interoperabilitas antara perangkat dari berbagai layanan yang sering kali menggunakan protokol dan antarmuka yang berbeda. Hal ini dapat menghambat integrasi yang mulus antara perangkat dan mempersulit manajemen smart home. Masalah lainnya adalah \textit{device discovery}. \textit{Device discovery} merupakan salah satu cara bagiamana untuk menemukan atau \textit{discover} perangkat \textit{IoT} yang terhubung dengan sistem \textit{smart home}. Masalah lainnya yang cukup penting yaitu masalah yang berkaitan dengan \textit{remote deployment}. \textit{Remote deployment} adalah bagaimana cara untuk membuat layanan IoT dapat dijalankan melalui satu \textit{entry point} saja tanpa harus melakukan konfigurasi dari setiap perangkat \textit{IoT}. 
  
  Oleh karena itu, diperlukan sebuah solusi yang dapat mengatasi ketiga masalah tersebut. Salah satu solusi yang dapat dibuat yaitu pembuatan \textit{Smart home system} berbasis \textit{service mesh}. Dengan menggunakan \textit{service mesh} proses \textit{device discovery} dapat dilakukan dengan mudah karena adanya komponen \textit{service mesh} yang tergabung dengan komponen \textit{IoT} sehingga membuat perangkat dapat ditemukan dengan mudah. Dengan adanya kubernetes, proses \textit{remote deployment} menjadi \textit{feasible} untuk dilakukan sehingga proses konfigurasi dari setiap perangkat dapat dipersingkat.
  
  Pada akhirnya, dengan adanya pembuatan \textit{smart home system} berbasis \textit{service mesh} dengan \textit{kubernetes} dapat membantu dalam meningkatkan layanan dan performa pada \textit{smart home system}.

  \textbf{\textit{Kata kunci: Smart Home System, Kubernetes, KubeEdge IoT, Service Mesh}}
  
\end{singlespace}
\clearpage