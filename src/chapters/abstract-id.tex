\clearpage
\chapter*{ABSTRAK}
\addcontentsline{toc}{chapter}{Abstrak}
\begin{center}
  \center
  \begin{singlespace}
    \large\bfseries\MakeUppercase{\thetitle}
    
    \normalfont\normalsize
    Oleh:
    
    \bfseries \theauthor
  \end{singlespace}
\end{center}

\begin{singlespace}
  \small
  Seiring bertambahnya jumlah pengguna perangkat \textit{Internet of Things (IoT)}, proses manajemen perangkat menjadi semakin lama dan kompleks. Salah satu proses yang penting dalam pengelolaan perangkat IoT yaitu proses pembaruan perangkat. Proses ini biasanya dilakukan secara serial dari satu perangkat ke perangkat lainnya. Hal ini sangat memakan waktu dan \textit{resource} yang sebenarnya dapat dimanfaatkan untuk proses lainnya pada manajemen perangkat. Pengelolaan proses pembaruan perangkat pada IoT adalah kunci untuk memastikan keandalan dan kinerja optimal dari jaringan yang semakin berkembang. Seiring dengan perkembangan teknologi \textit{IoT}, metode \textit{Over The Air} (OTA) mulai diperkenalkan, metode ini memungkinkan pembaruan perangkat dilakukan secara \textit{remote}. Meskipun metode OTA telah membawa banyak perbaikan, tantangan dalam manajemen perangkat yang skalabel dan efisien tetap ada.
  
  Penelitian ini bertujuan untuk mengeksplorasi penggunaan Kubernetes sebagai solusi untuk pengelolaan perangkat \textit{IoT} secara \textit{remote}. Kubernetes, yang telah terbukti efektif dalam manajemen aplikasi container di lingkungan \textit{cloud}, memiliki potensi besar untuk diterapkan dalam ekosistem \textit{IoT}. Namun, hingga saat ini, penelitian yang mengevaluasi penggunaan Kubernetes untuk manajemen perangkat \textit{IoT} masih sangat terbatas. Oleh karena itu, penelitian ini berfokus untuk merancang sistem \textit{remote deployment} pada lingkungan \textit{IoT} dengan menggunakan Kubernetes.
  
  Implementasi sistem \textit{remote deployment} pada lingkungan \textit{IoT} dengan kubernetes berhasil dilakukan pada dua buah \textit{RaspberryPi} yang berjalan sebagai kubernetes cluster.  Implementasi cluster dibuat dengan menggunakan K3s sebagai distribusi kubernetes untuk mendukung perangkat \textit{low resource}. Deployment yang dilakukan ialah untuk menyalakan dua buah lampu yang terhubung dengan masing masing \textit{RaspberryPi}. Alhasil, proses \textit{deployment pada IoT} dapat dilakukan secara \textit{remote}.
  
  \textbf{\textit{Kata kunci: IoT, Remote Deployment, Kubernetes }}
  
\end{singlespace}
\clearpage