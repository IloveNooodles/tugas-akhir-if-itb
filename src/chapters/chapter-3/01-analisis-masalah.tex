\subsection{Analisis Permasalahan}
\label{sec:analisis-permasalahan}

Pada lingkungan IoT, berbagai perangkat terhubung memainkan peran penting dalam mengumpulkan dan memproses data dari lingkungan sekitarnya. Namun, perangkat ini sering memiliki keterbatasan dalam \textit{resource}. Selain itu, aplikasi solusi ynag dibuat sering kali memiliki kompatibilitas yang rendah sehingga hanya bisa digunakan untuk beberapa \textit{device} saja. Oleh karena itu, terdapat kebutuhan untuk mengimplementasikan logika aplikasi yang kompleks secara langsung ke perangkat-perangkat ini. Implementasi ini berfungsi untuk meningkatkan efisiensi, mengurangi latensi, dan mendukung operasi \textit{offline} atau semi-\textit{offline}. Masalah utama yang muncul adalah bagaimana cara yang efisien untuk mengelola dan mengerahkan komponen aplikasi pada skala besar dalam lingkungan yang sangat heterogen dan memiliki keterbatasan sumber daya.

Berdasarkan latar belakang yang telah diuraikan pada bagian \ref{sec:latar-belakang}, masalah ini dapat diselesaikan dengan menggunakan \textit{remote deployment}. Namun, proses pembuatan sistem \textit{remote deployment} untuk perangkat yang heterogen menjadi hal yang cukup sulit. Sistem juga harus dapat berjalan pada perangkat yang memiliki \textit{low computaional power}  sehingga dapat digunakan di berbagai jenis \textit{device}. Sistem yang dibuat pun harus memiliki sebuah standar \textit{deployment}. Standar ini dapat disebut sebagai \textit{deployment plan}, yang terdiri dari model \textit{deployment} serta layanan orkestrasi untuk membantu proses pengelolaan. Layanan orkestrasi berperan cukup penting dalam sistem. Dengan adanya layanan ini, proses pengelolaan perangkat yang heterogen menjadi cukup mudah untuk dilakukan. Selain layanan orkestrasi, \textit{deployment plan} model pun perlu memiliki standar agar dapat digunakan oleh \textit{client} atau \textit{developer}.

Sistem dibuat dengan menyediakan infrastruktur yang berorientasi layanan untuk pengerahan komponen aplikasi yang elastis pada perangkat yang memiliki sumber daya yang terbatas pada IoT dengan skala besar. Sistem harus memiliki dukungan untuk melakukan \textit{deployment} berbasis push serta sistem yang dibuat perlu memiliki cara untuk membuat sebuah \textit{deployment plan} yang terdiri dari beberapa komponen iot serta
pemrosesan paket aplikasi yang disesuaikan dengan platform perangkat, dan mekanisme skalabilitas untuk mengelola penyebaran ke jumlah perangkat yang besar dengan efisien.

Untuk membuat implementasi dari sistem \textit{remote deployment}, terdapat beberapa rintangan yang perlu diatasi. Rintangan tersebut, dapat dirumuskan ke dalam poin-poin sebagai berikut.

\begin{enumerate}
  \item Saat ini, belum banyak sistem IoT yang dibuat dan diorkestrasi menggunakan Kubernetes karena Kubernetes membutuhkan \textit{resource} yang cukup besar.
  \item Sistem harus dapat berjalan di berbagai platform \textit{(platform agnostic)}
  \item Belum ada \textit{deployment plan} yang dapat mendefinisikan proses \textit{deployment} dengan Kubernetes pada IoT
  \item Sistem dapat melakukan \textit{deployment} dengan metode \textit{push}.
\end{enumerate}
