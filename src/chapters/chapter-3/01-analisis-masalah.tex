\section{Analisis Persoalan}

\textit{Autoscaler} merupakan salah satu teknologi yang bermanfaat untuk membantu pengelola infrastruktur dalam melakukan \textit{scaling} pada teknologi \textit{container orchestration} seperti Kubernetes. Dengan menggunakan \textit{autoscaler}, pengguna dapat mengatur dan mengontrol (\textit{scaling}) sumber daya dari sekumpulan \textit{pods} dengan otomatis. Pada teknologi \textit{autoscaler} yang paling sederhana, sistem akan secara otomatis melakukan \textit{scaling} apabila suatu metrik melewati ambang batas tertentu. Beberapa \textit{autoscaler} yang paling populer saat ini adalah \textit{horizontal autoscaling} dan \textit{vertical autoscaling} yang melakukan \textit{scaling} berdasarkan metrik sistem seperti utilisasi prosesor, memori serta beban permintaan dalam satuan waktu. Perkembangan dari metode \textit{autoscaling} telah menyediakan pengguna berbagai opsi untuk mengotomasi alokasi sumber daya. Namun, dari semua metode yang sekarang sudah ada, terdapat beberapa kekurangan yang dapat diperbaiki, diantaranya sebagai berikut.

\begin{enumerate}
    \item Terdapat keperluan pengelola untuk menambah dan mengurangi secara manual terhadap konfigurasi \textit{deployment autoscaler} yang beracuan pada \textit{throughput} yang sebenarnya bisa diotomasi dengan sistem melalui aturan yang dibuat oleh pengelola.
    \item \textit{Autoscaler} sederhana umumnya dibuat untuk sesuatu yang sangat umum seperti pemrosesan seperti \textit{web service}, \textit{CRON Job}, dan sebagainya. Hal ini sedikit berbeda jika berkaitan dengan sistem \textit{information retrieval}, karena sistem \textit{information retrieval} memiliki data yang berbeda sehingga kegunaan dan keperluannya menjadi bervariatif.
    \item \textit{Metrics} yang digunakan tidak melihat data waktu sebelumnya yang bisa menjadi acuan untuk melihat pola penggunaan. Ditambah dengan fakta bahwa \textit{autoscaler} sederhana berpatokan dengan sebuah ambang batas sehingga tidak efisien terhadap kinerja sistem yang fluktuatif. Apabila hanya mengandalkan angka ambang batas, ketika sistem melewati angka ambang batas untuk waktu yang sangat singkat, maka sistem akan mencoba untuk melakukan \textit{scaling} yang sebenarnya bisa dianggap tidak perlu jika hanya \textit{scaling} untuk waktu yang singkat, \parencite{riset1}. 
    \item Setiap pengelola infrastruktur memiliki variabel \textit{cost} dan target kinerja sistem yang berbeda-beda. Toleransi terhadap target kinerja sistem dan \textit{cost} yang berbeda memerlukan \textit{autoscaler} yang dapat dikonfigurasi dengan fleksibel.
\end{enumerate}

Pada tugas akhir ini, akan dilakukan penelitian untuk melakukan pengembangan metode \textit{autoscale} yang berjalan diatas Kubernetes yang spesifik untuk mengontrol alokasi sumber daya \textit{Elastic Search}. Dengan melakukan pengembangan tersebut, diharapkan penelitian ini dapat meningkatkan efisiensi \textit{autoscale} pada \textit{Elastic Search} dengan kontrol fleksibel berdasarkan model prediktif berbasis \textit{time series}. Pendekatan prediksi berbasis \textit{time series} dilakukan karena metrik sistem sangat dekat dengan data historis dan korelasi data dengan waktu. Melalui referensi studi literatur yang sudah dilakukan, banyak penelitian yang menggunakan model prediksi dengan \textit{time series} seperti ARIMA, LSTM, dan Bi-LSTM.

% Namun, saat ini kontrol adaptif milik Kubernetes memiliki beberapa kekurangan. Salah satu kekurangan dari kontrol adaptif kubernetes adalah \textit{trigger autoscale} yang didasarkan oleh metriks umum seperti utilisasi memori dan prosesor berdasarkan waktu saat itu. Sedangkan, tidak semua \textit{container} dinyatakan memakai sumber daya komputasi secara efisien jika hanya memakai faktor tersebut sebab tidak semua \textit{container} bergantung hanya pada utilisasi prosesor dan memori. Terkadang ada beberapa fitur pada sebuah \textit{container} yang terus berjalan namun tidak terpakai atau disimpan pada \textit{cache} namun tidak dipakai. Semuanya kembali lagi kepada isi dari sebuah \textit{container} serta penggunaannya. Selain dari hal tersebut, kubernetes juga hanya memakai data \textit{metrics} yang sedang berlangsung. Sehingga, kubernetes tidak dapat memprediksi kebutuhan \textit{resource} pada waktu yang akan datang melalui pola penggunaan. Hal ini akan menjadi acuan untuk meningkatkan efisiensi \textit{autoscale} dari kubernetes terutama pada \textit{Elastic Search} melalui adaptif kontrol dengan model prediktif berbasis \textit{time series}.