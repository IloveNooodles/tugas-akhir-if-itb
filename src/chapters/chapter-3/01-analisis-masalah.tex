\subsection{Analisis Permasalahan}
\label{sec:analisis-permasalahan}

Pada lingkungan IoT, berbagai perangkat terhubung memainkan peran penting dalam mengumpulkan dan memproses data dari lingkungan sekitarnya. Namun, perangkat ini sering memiliki keterbatasan dalam \textit{resource}. Selain itu, aplikasi yang dibuat sering kali memiliki kompatibilitas yang rendah sehingga hanya bisa digunakan untuk beberapa \textit{device} saja. Oleh karena itu, terdapat kebutuhan untuk mengimplementasikan logika aplikasi yang kompleks secara langsung ke perangkat perangkat ini. Implementasi ini berfungsi untuk membuat cara agar dapat mengelola dan mengerahkan komponen aplikasi pada skala besar dalam lingkungan yang heterogen dan memiliki keterbatasan sumber daya.

Berdasarkan latar belakang yang telah diuraikan pada bagian \ref{sec:latar-belakang}, masalah ini dapat diselesaikan dengan menggunakan PERISAI. Namun, proses pembuatan sistem \textit{remote deployment} untuk perangkat yang heterogen menjadi hal yang cukup sulit karena sistem juga harus dapat dijalankan pada perangkat yang memiliki \textit{low computaional power} sehingga dapat digunakan di berbagai jenis \textit{device}. PERISAI, memiliki cara untuk mendefinisikan proses \textit{deployment} yaitu dengan membuat \textit{deployment plan}, yang terdiri dari model \textit{deployment}, jenis aplikasi yang ingin digunakan, serta jenis dan kapabilitas perangkat yang dibuat serta layanan orkestrasi untuk membantu proses pengelolaan. Dengan adanya layanan orkestrasi ini, proses pengelolaan perangkat yang heterogen menjadi cukup mudah untuk dilakukan.

PERISAI dibuat dengan menyediakan infrastruktur yang berorientasi layanan untuk pengerahan komponen aplikasi yang elastis pada perangkat yang memiliki sumber daya yang terbatas pada IoT dengan skala besar. Sistem harus memiliki dukungan untuk melakukan \textit{deployment} berbasis push serta sistem yang dibuat perlu memiliki cara untuk membuat sebuah \textit{deployment plan} yang terdiri dari beberapa komponen IoT serta pemrosesan paket aplikasi yang disesuaikan dengan platform perangkat, dan mekanisme skalabilitas untuk mengelola penyebaran ke jumlah perangkat yang besar.

Untuk membuat PERISAI, terdapat beberapa rintangan yang perlu diatasi. Rintangan tersebut, dapat dirumuskan ke dalam poin-poin sebagai berikut.

\begin{enumerate}
  \item Sistem dapat berjalan pada perangkat yang memiliki daya komputasi rendah
  \item Sistem harus dapat berjalan di berbagai platform \textit{(platform agnostic)}
  \item Sistem dapat melakukan \textit{deployment} dengan metode \textit{push}.
  \item Sistem harus bisa melakukan \textit{deployment} pada perangkat tertentu (\textit{targeted deployment})
\end{enumerate}
