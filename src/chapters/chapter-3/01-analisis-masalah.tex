\section{Analisis Permasalahan}
\label{sec:analisis-permasalahan}

Pada lingkungan \textit{IoT}, berbagai perangkat terhubung memainkan peran penting dalam mengumpulkan dan memproses data dari lingkungan sekitarnya. Perangkat ini sering kali terbatas sumber daya, memiliki kapasitas pemrosesan, memori, dan penyimpanan yang minim. Meskipun begitu, ada kebutuhan yang meningkat untuk mengerahkan logika aplikasi yang kompleks secara langsung ke perangkat-perangkat ini untuk meningkatkan efisiensi, mengurangi latensi, dan mendukung operasi \textit{offline} atau semi-\textit{offline}. Masalah utama yang muncul adalah bagaimana secara efisien mengelola dan mengerahkan komponen aplikasi pada skala besar dalam lingkungan yang sangat heterogen dan terbatas sumber daya.

Berdasarkan latar belakang yang telah diuraikan pada \ref{sec:latar-belakang}, masalah ini dapat diselesaikan dengan menggunakan \textit{remote deployment}. Namun, proses pembuatan layanan agar dapat melakukan \textit{remote deplotment} untuk perangkat yang heterogen menjadi hal yang cukup sulit, perlu dibuat sebuah sistem yang mengatur seluruh proses agar menjadi konsisten dan \textit{well documented}. Sistem ini juga bermanfaat untuk pengguna yang ingin mencoba mengembangkan sistem untuk menambah berbagai fitur kedepannya. Standar ini akan terdiri dari model \textit{deployment} serta layanan orkestrasi untuk membantu proses pengelolaan. Layanan orkestrasi, berperan cukup penting dalam sistem, dengan adanya layanan ini proses pengelolaan perangkat yang heterogen menjadi cukup mudah untuk dilakukan. Selain layanan orkestrasi, \textit{deployment plan} model pun perlu memiliki standar agar dapat digunakan oleh \textit{client} atau \textit{developer}.

Sistem dibuat dengan menyediakan infrastruktur yang berorientasi layanan untuk pengerahan komponen aplikasi yang elastis pada perangkat yang memiliki sumber daya yang terbatas pada IoT dengan skala besar. Sistem harus memiliki dukungan untuk melakukan \textit{deployment} berbasis push, serta mengetahui kondisi dari setiap perangkat. Sistem yang dibuat perlu memiliki cara untuk membuat sebuah resep deployment yang terdiri dari beberapa komponen iot serta
pemrosesan paket aplikasi yang disesuaikan dengan platform perangkat, dan mekanisme skalabilitas untuk mengelola penyebaran ke jumlah perangkat yang besar dengan efisien.

Untuk membuat implementasi dari sistem \textit{remote deployment}, terdapat beberapa rintangan yang perlu diatasi. Rintangan tersebut, dapat dirumuskan ke dalam poin-poin sebagai berikut.

\begin{enumerate}
  \item Kondisi saat ini belum banyak sistem \textit{IoT} yang dibuat dan diorkestrasi menggunakan \textit{kubernetes}
  \item Belum ada \textit{deployment plan} yang dapat mendefinisikan proses \textit{deployment} serta agar \textit{remote deployment} dapat diorkestrasi  dengan menggunakan \textit{kubernetes}
  \item Sistem dapat berjalan meskipun dalam kondisi buruk.
  \item Sistem dapat mengetahui kondisi dari masing masing perangkat serta proses \textit{deployment} yang sedang berjalan.
  \item Sistem dapat melakukan deployment dengan metode \textit{push}, sehingga setiap perangkat yang terhubung akan melakukan proses pembaruan secara otomatis.
\end{enumerate}
