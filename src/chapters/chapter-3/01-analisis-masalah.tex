\section{Analisis Persoalan}
\label{sec:analisis-persoalan}

Untuk meningkatkan kualitas layanan \textit{IoT} khususnya pada \textit{Smart Home System} perlu dibuat suatu cara agar layanan yang dibuat memiliki latensi yang rendah serta memiliki kemudahan untuk melakukan \textit{scaling} ataupun melakukan \textit{deployment} dalam skala besar. Terdapat beberapa \textit{requirements} yang dibutuhkan agar peningkatan layanan kualitas \textit{IoT} pada \textit{Smart Home System} dapat dilakukan

\begin{enumerate}
  \item Kondisi saat ini belum banyak sistem \textit{IoT} yang dibuat dan di\textit{deploy} menggunakan KubeEdge, sehingga perlu dibuat suatu cara standard untuk melakukan proses \textit{deployment} pada \textit{IoT}
  \item Proses \textit{service discovery} pada \textit{IoT} harus dapat dengan mudah dilakukan
  \item Sistem harus memiliki \textit{low latency} untuk mempercepat komunikasi antar layanan
  \item Sistem harus memiliki skalabilitas yang baik agar proses \textit{scaling} dapat dilakukan dengan mudah
  \item Walaupun sistem bersifat \textit{distributed}, sistem yang dibuat haruslah memiliki sifat \textit{fault tolerant} serta \textit{high availability} untuk mencegah adanya \textit{downtime}
\end{enumerate}


menjawab beberapa permasalahan mulai dari tinggi nya \textit{latency} dan sulitnya melakukan \textit{service discovery} ataupun \textit{device discovery}
Untuk membuat sebuah \textit{Smart Home System} yang terintegrasi dengan \textit{Service Mesh} ada beberapa hal yang perlu di konsiderasi
