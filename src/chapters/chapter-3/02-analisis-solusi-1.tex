\subsection{Alternatif Solusi}
\label{sec:analisis-solusi}

Berdasarkan analisis permasalahan pada bagian \ref{sec:analisis-permasalahan} serta studi literatur, Terdapat tiga alternatif solusi untuk mengatasi permasalahan tesrebut. Sistem \textit{remote deployment} dapat dibuat dengan 3 cara yaitu berbasis Kubernetes, berbasis \textit{apache zookeper}, serta membuat sendiri dari \textit{scratch}.

\subsubsection{Membuat sistem \textit{remote deployment} berbasis Kubernetes}
Kubernetes, sebagai layanan orkestrasi kontainer yang kuat, menawarkan mekanisme otomatisasi \textit{deployment}, skalabilitas, dan manajemen aplikasi yang sangat cocok untuk lingkungan \textit{cloud} dan terdistribusi. Dalam konteks IoT, Kubernetes bisa dimanfaatkan untuk mengelola dan mengorkestrasi aplikasi berbasis kontainer pada lingkungan IoT ataupun pada \textit{cloud}. Kubernetes mendukung model \textit{microservices} yang fleksibel dan resilien serta memiliki banyak kelebihan yaitu  meliputi kemampuan autoscaling, pemulihan otomatis dari kegagalan, dan manajemen beban kerja yang efisien. Namun, Kubernetes juga memiliki kekurangan yaitu membutuhkan sumber daya yang relatif tinggi, yang mungkin tidak ideal untuk perangkat IoT dengan keterbatasan sumber daya. Namun, hal ini dapat diatasi dengan distribusi Kubernetes yang khusus dibuat untuk lingkungan IoT yaitu microk8s, K3s, serta KubeEdge. Dari ketiga solusi ini, dipilih K3s sebagai distribusi yang digunakan karena kemudahan dalam penggunaan serta \textit{resource friendly} karena dapat berjalan pada \textit{device} yang hanya memiliki RAM 900mb tanpa ada masalah sama sekali. Selain itu kemudahan dalam penggunaan dan supportnya untuk seluruh arsitektur membuat K3s menjadi pilihan yang digunakan untuk tugas akhir ini. Perbandingan ketiga distribusi dapat dilihat pada tabel \ref{tab:perbandingan-distribusi-kubernetes}.

\subsubsection{Membuat sistem \textit{remote deployment} berbasis \textit{Apache ZooKeper}}
Apache ZooKeeper menyediakan layanan koordinasi terdistribusi yang sangat diperlukan pada lingkungan aplikasi terdistribusi. Dalam konteks sistem \textit{remote deployment} pada lingkungan IoT, ZooKeeper dapat dimanfaatkan untuk mengelola konfigurasi dan melakukan sinkronisasi diantara komponen sistem terutama dalam lingkungan terdistribusi. Kelebihan menggunakan ZooKeeper antara lain konsistensi data yang tinggi dan model pemrograman yang relatif sederhana. Kekurangannya, ZooKeeper tidak secara langsung mendukung kasus penggunaan IoT yang spesifik tanpa integrasi tambahan, serta membutuhkan \textit{resource} yang sangat besar sehingga bisa menjadi \textit{bottleneck} jika tidak dirancang dengan hati-hati, terutama dalam lingkungan dengan jumlah perangkat yang sangat besar. Oleh karena itu, diperlukan adanya penyesuaian dan adopsi agar menjadi kompatibel dan tidak menjadi \textit{bottleneck}.

\subsubsection{Membuat sistem \textit{remote deployment} dari \textit{scratch}}
Mengembangkan sistem provisioning skala besar dari \textit{scratch} memberikan kebebasan maksimal dalam desain arsitektur yang spesifik sesuai dengan kebutuhan sistem. Hal ini memungkinkan untuk mengoptimasi kinerja, efisiensi sumber daya, dan integrasi dengan teknologi yang cocok. Kelebihannya termasuk fleksibilitas desain dan potensi untuk inovasi. Namun, pendekatan ini juga memiliki kekurangan signifikan seperti waktu pengembangan yang lebih lama, resiko keamanan yang lebih tinggi karena kurangnya \textit{testing}, dan tantangan dalam pemeliharaan serta skalabilitas sistem di masa depan.