\section{Analisis Solusi}

Untuk melakukan penelitian pengembangan arsitektur integrasi \textit{Servce Mesh} dengan \textit{IoT} khususnya pada \textit{smart home system}, perlu dibuat sebuah solusi yang dapat memenuhi \textit{requirements} yang sebelumnya telah dijelaskan pada \ref{sec:analisis-persoalan}. Berikut merupakan analisis solusi yang akan diimplementasikan untuk setiap persoalan

\subsection{Arsitektur standar untuk \textit{deployment} pada KubeEdge}
Untuk membuat arsitektur pada KubeEdge perlu memerlukan beberapa konsiderasi terutama pada bagian \textit{EdgeCore}. Secara konsep arsitektur pada KubeEdge tidak jauh berbeda dari kubernetes pada umumnya, namun terdapat perbedaan cara berkomunkasi dari bagian \textit{cloud} dengan \textit{edge} serta komunikasi antar perangkat \textit{IoT} pada \textit{Edge} yang menggunakan MQTT. Dengan melakukan percobaan \textit{deploy} dengan KubeEdge dan melihat referensi pada \ref{sec:riset-terkait} proses standardiasi arsitektur akan mudah untuk dilakukan

\subsection{Meningkatkan \textit{Service discovery} pada layanan}
Dengan melakukan integrasi dengan \textit{Service Mesh} proses \textit{service discovery} menjadi lebih mudah untuk dilakukan karena untuk setiap aplikasi yang di\textit{deploy} menggunakan KubeEdge, akan ditambah sebuah \textit{sidecar proxy} untuk proses penerimaan maupun pengiriman \textit{request}. Dengan adanya \textit{sidecar proxy} proses \textit{service discovery} menjadi lebih mudah untuk dilakukan.

\subsection{Menurunkan \textit{low latency} pada layanan}
Tantangan utama pada aplikasi yang terintegrasi dengan \textit{service mesh} adalah \textit{latency} akan dijamin bertambah karena adanya \textit{extra hop} yang digunakan saat pengiriman maupun penerimaan request. Untuk mengatasi masalah ini dapat digunakan metode lain untuk implementasi \textit{service mesh} seperti \textit{zero-trust tunnel} ataupun menggunakan \textit{eBPF}
\subsection{Membuat sistem dengan skalabilitas yang baik}
Dengan menggunakan kubernetes, khususnya \textit{KubeEdge} dapat dibuat sebuah konfigurasi file \textit{deployment} yang telah memenuhi semua \textit{requirements}. File ini akan menjadi acuan untuk proses \textit{deployment} kedepannya.

\subsection{Membuat sistem dengan \textit{high availability} dan \textit{fault tolerant}}
Dengan memanfaatkan keunggulan \textit{rollout} dan \textit{rollback} serta \textit{scaling} untuk \textit{deployment} pada kubernetes. Masalah ini menjadi mudah untuk diatasi karena secara umum kubernetes akan memastikan proses \textit{upgrade} ataupun \textit{downgrade} berjalan terlebih dahulu sebelum mematikan layanan yang lama. Karena pada \textit{IoT} memiliki \textit{spec} perangkat yang rendah, proses \textit{scaling} yang digunakan harus menjadi bagian yang di konsiderasi