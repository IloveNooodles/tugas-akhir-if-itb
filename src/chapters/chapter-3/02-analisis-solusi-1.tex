\subsection{Alternatif Solusi}
\label{sec:analisis-solusi}

Berdasarkan analisis permasalahan pada bagian \ref{sec:analisis-permasalahan} serta studi literatur, Terdapat tiga alternatif solusi dalam pembuatan PERISAI. yaitu membuat arsitektur berbasis Kubernetes, arsitektur berbasis \textit{apache zookeper}, serta membuat arsitektur dengan mengikuti referensi yang sudah ada.

\subsubsection{Membuat sistem berbasis Kubernetes}
Kubernetes, sebagai layanan orkestrasi kontainer yang kuat, menawarkan mekanisme otomatisasi \textit{deployment}, skalabilitas, dan manajemen aplikasi yang sangat cocok untuk lingkungan \textit{cloud} dan terdistribusi. Dalam konteks IoT, Kubernetes bisa dimanfaatkan untuk mengelola dan mengorkestrasi aplikasi berbasis kontainer pada lingkungan IoT ataupun pada \textit{cloud}. Kubernetes mendukung model \textit{microservices} yang fleksibel dan resilien serta memiliki banyak kelebihan yaitu meliputi kemampuan autoscaling, pemulihan otomatis dari kegagalan, dan manajemen beban kerja yang efisien. Kubernetes dapat memanfaatkan affinity selector untuk melakukan \textit{targeted deployment}, sehingga proses \textit{deployment} dapat diberikan kepada perangkat yang memerlukan \textit{deployment} saja. Namun, Kubernetes juga memiliki kekurangan yaitu membutuhkan sumber daya yang relatif tinggi, yang mungkin tidak ideal untuk perangkat IoT dengan keterbatasan sumber daya. Namun, hal ini dapat diatasi dengan distribusi Kubernetes yang khusus dibuat untuk lingkungan IoT yaitu microk8s, K3s, serta KubeEdge. Dari ketiga solusi ini, distribusi K3s dapat dijalankan pada perangkat yang hanya memiliki RAM 900mb tanpa ada masalah. Selain itu kemudahan dalam penggunaan dan supportnya untuk seluruh arsitektur membuat K3s menjadi pilihan yang digunakan untuk tugas akhir ini. Perbandingan ketiga distribusi dapat dilihat pada lampiran \ref{tab:perbandingan-distribusi-kubernetes}.

\subsubsection{Membuat sistem berbasis \textit{Apache ZooKeper}}
Apache ZooKeeper menyediakan layanan koordinasi terdistribusi yang sangat diperlukan pada lingkungan aplikasi terdistribusi. Dalam konteks sistem \textit{remote deployment} pada lingkungan IoT, ZooKeeper dapat dimanfaatkan untuk mengelola konfigurasi dan melakukan sinkronisasi diantara komponen sistem terutama pada lingkungan terdistribusi. Kelebihan menggunakan ZooKeeper antara lain konsistensi data yang tinggi dan model pemrograman yang relatif sederhana. Kekurangannya, ZooKeeper tidak secara langsung mendukung kasus \textit{deployment} aplikasi karena hanya berfokus pada masalah konfigurasi, Selain itu zookeper juga membutuhkan \textit{resource} yang sangat besar sehingga bisa menjadi \textit{bottleneck} jika tidak dirancang dengan hati-hati, terutama dalam lingkungan dengan jumlah perangkat yang sangat besar. Zookeper juga tidak Oleh karena itu, diperlukan adanya penyesuaian dan adopsi agar menjadi kompatibel dan tidak menjadi \textit{bottleneck}.

\subsubsection{Membuat sistem berbasis LEONORE dan DIANE}
Mengembangkan sistem provisioning skala besar berbasis LEONORE dan DIANE sudah menyelesaikan beberapa masalah yang telah disebutkan. Namun kekurangannya ialah, desain menjadi tidak fleksibel, sulit untuk dikustomisasi, serta terdapat resiko keamanan jika tidak mengimplementasikan dengan benar. Selain itu proses implementasi membutuhkan waktu yang lama. Proses pengujian yang sulit pun menjadi salah satu tantangan dalam mengimplementasikan metode ini.

\pagebreak

Berdasarkan ketiga alternatif solusi yang telah dijelaskan, Dipilihlah kubernetes sebagai dasar dari PERISAI. Alasan pemilihan kubernetes karena menjawab permasalahan yang telah dijelaskan pada bagian \ref{sec:analisis-permasalahan} khususnya \textit{targeted deployment}.Perbandingan ketiga alternatif solusi dapat dilihat pada tabel \ref{tab:perbandingan-analisis-solusi}

\bgroup
\begin{table}[ht]
  \def\arraystretch{1.3}
  \caption{Perbandingan Ketiga Alternatif Solusi}
  \label{tab:perbandingan-analisis-solusi}
  \centering
  \begin{tabular}{|p{2cm}|p{2cm}|p{2cm}|p{1.8cm}|p{1.7cm}|p{1.7cm}|}
    \hline
    Solusi           & Berjalan di berbagai perangkat                            & Melakukan \textit{targeted deployment} & Berjalan pada perangkat dengan sumber daya terbatas & Mengatur banyak perangkat & Waktu pembuatan sistem \\
    \hline
    Kubernetes       & Ya, seluruh perangkat yang dapat melakukan kontainerisasi & Ya                                     & Ya dengan K3s                                       & Ya                        & Cepat                  \\
    \hline
    Zookeper         & Ya, seluruh perangkat yang memiliki java                  & Tidak                                  & Tidak                                               & Ya                        & Cepat                  \\
    \hline
    LEONORE \& DIANE & Tidak                                                     & Tidak                                  & Mungkin                                             & Ya                        & Lama                   \\
    \hline
  \end{tabular}
\end{table}
\egroup