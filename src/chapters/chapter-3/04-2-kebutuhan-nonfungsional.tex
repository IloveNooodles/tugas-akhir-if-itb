\subsubsection{Kebutuhan Non-Fungsional}
Berikut merupakan kebutuhan non fungsional dari sistem yang akan dibuat, agar lebih jelas kebutuhan non fungsional dibuat ke dalam bentuk tabel. Semua kebutuhan non-fungsional memiliki awalan ID NF lalu diikuti oleh dua angka. Terdapat lima parameter non fungsional yang dapat dilihat pada tabel \ref{tab:kebutuhan-non-fungsional}

\bgroup
\begin{table}[ht]
  \def\arraystretch{1.7}
  \caption{Kebutuhan Non-Fungsional}
  \label{tab:kebutuhan-non-fungsional}
  \centering
  \begin{tabular}{|c|p{3cm}|p{8cm}|}
    \hline
    ID   & Parameter      & Kebutuhan                                                                                                                                                                                                                               \\
    \hline
    NF01 & Maintanability & Sistem dapat dimaintain dari jauh dengan mudah dengan melakukan aksi melalui dashboard                                                                                                                                                  \\
    \hline
    NF02 & Security       & Sistem akan menjamin keamanan dari \textit{service} dengan cara menyediakan validasi pada \textit{middleware} serta setiap \textit{request} dapat terhindar dari serangan. Sehingga hanya user yang terautentikasi yang bisa mengakses. \\
    \hline
    NF03 & Portability    & Sistem dapat diakses dimana saja melalui perangkat laptop                                                                                                                                                                               \\
    \hline
  \end{tabular}
\end{table}
\egroup