\subsubsection{Kebutuhan Non-Fungsional}
Sistem memiliki 5 parameter kebutuhan non-fungsional yang dapat dilihat pada tabel \ref{tab:kebutuhan-non-fungsional}. Semua kebutuhan non-fungsional memiliki awalan ID NF lalu diikuti oleh dua angka.

\bgroup
\begin{table}[ht]
  \def\arraystretch{1.7}
  \caption{Kebutuhan Non-Fungsional}
  \label{tab:kebutuhan-non-fungsional}
  \centering
  \begin{tabular}{|c|p{3cm}|p{8cm}|}
    \hline
    ID   & Parameter               & Kebutuhan                                                                                                                                                                                                                          \\
    \hline
    NF01 & \textit{Maintanability} & Sistem dapat dimaintain dari jauh dengan mudah dengan melakukan aksi melalui \textit{dashboard}.                                                                                                                                   \\
    \hline
    NF02 & \textit{Security}       & Sistem menjamin keamanan dari \textit{service} dengan cara menyediakan validasi pada \textit{middleware} serta setiap \textit{request} dapat terhindar dari serangan. Hanya \textit{user} yang terautentikasi yang bisa mengakses. \\
    \hline
    NF03 & \textit{Portability}    & Sistem dapat diakses dimana saja melalui perangkat laptop.                                                                                                                                                                         \\
    \hline
  \end{tabular}
\end{table}
\egroup