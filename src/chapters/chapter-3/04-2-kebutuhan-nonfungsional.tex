\subsubsection{Kebutuhan Non-Fungsional}
Berikut merupakan kebutuhan non fungsional dari sistem yang akan dibuat, agar lebih jelas kebutuhan non fungsional dibuat ke dalam bentuk tabel di bawah ini. Semua kebutuhan non-fungsional memiliki ID NF lalu diikuti oleh dua angka
\begin{table}[h]
  \caption{Kebutuhan Non-Fungsional}
  \label{tab:kebutuhan-non-fungsional}
  \centering
  \begin{tabular}{|c|p{3cm}|p{8cm}|}
    \hline
    ID   & Parameter      & Kebutuhan                                                  \\
    \hline
    NF01 & Availability   & Tingkat ketersediaan sistem minimal 98\%                   \\
    \hline
    NF02 & Maintanability & Sistem dapat dimaintain dari jauh dengan mudah             \\
    \hline
    NF03 & Security       & Sistem akan menjamin keamanan dari masing masing perangkat \\
    \hline
    NF04 & Safety         & Sistem akan menjamin keamanan dari masing masing perangkat \\
    \hline
    NF05 & Portability    & Sistem dapat diakses dimana saja menggunakan komputer      \\
    \hline
  \end{tabular}
\end{table}