\subsection{Rancangan Detail Komponen Dashboard}
\label{sec:rancangan-dashboard}

Seperti yang telah disebutkan pada bagian \ref{subsec:model-usecase} mengenai \textit{use case} diagram, bagian \ref{subsec:arsitektur-behavioural}, dan bagian \ref{subsec:arsitektur-struktural} dapat dilakukan pemetaan halaman yang dibuat untuk sistem dashboard. Terdapat dua modul yaitu modul \textit{API Connector} serta \textit{display}.

Modul \textit{API Connector} dibuat dengan cara melakukan HTTP Request ke sistem \textit{service}. Modul display yang bertanggung jawab untuk menampilkan halaman yang digunakan oleh user. Pada modul ini terdapat 10 halaman yang dapat dilihat pada daftar di bawah ini.

\begin{enumerate}
  \item Halaman Login

        Halaman ini digunakan sebagai entrypoint dari sistem \textit{dashboard}. Pada halaman ini terdapat beberapa input yang dapat \textit{user} masukan untuk mengirimkan kredensial ke server. Setelah melalui halaman ini, barulah semua fitur dapat diakses.

  \item Halaman utama

        Halaman ini adalah halaman yang dituju oleh \textit{user} ketika telah menyelesaikan proses login. Halaman ini berisi \textit{summary} dari seluruh objek yang terdapat pada perusahaan ini.

  \item Halaman \textit{account}

        Halaman ini adalah halaman yang dapat diakses oleh \textit{user} setelah login. Halaman ini berisi informasi perusahaan \textit{user} serta daftar \textit{user} lain yang terdaftar.

  \item Halaman \textit{devices}

        Halaman ini adalah halaman yang dapat diakses oleh \textit{user} setelah login. Halaman ini berisi informasi mengenai perangkat apa saja yang ada pada perusahaan. \textit{user} dapat menghapus serta menambahkan perangkat pada halaman ini. Selain itu \textit{user} juga dapat mengunjungi laman detail dari masing masing perangkat.

  \item Halaman \textit{devices detail}

        Halaman ini adalah halaman yang dapat diakses oleh \textit{user} setelah login. Halaman ini dapat dituju dengan cara pergi ke halaman \textit{devices} dan memilih \textit{devices} mana yang ingin dilihat informasi lebih lanjut.


  \item Halaman \textit{groups}

        Halaman ini adalah halaman yang dapat diakses oleh \textit{user} setelah login. Halaman ini berisi informasi mengenai \textit{groups} apa saja yang ada pada perusahaan. \textit{user} dapat menghapus serta menambahkan \textit{groups} pada halaman ini. Selain itu \textit{user} juga dapat mengunjungi laman detail dari setiap \textit{groups}.

  \item Halaman \textit{groups detail}

        Halaman ini adalah halaman yang dapat diakses oleh \textit{user} setelah login. Halaman ini dapat dituju dengan cara pergi ke halaman \textit{groups} dan memilih \textit{groups} mana yang ingin dilihat informasi lebih lanjut


  \item Halaman \textit{deployments}

        Halaman ini adalah halaman yang dapat diakses oleh \textit{user} setelah login. Halaman ini berisi informasi mengenai \textit{deployment images} serta \textit{deployment plan} yang ada pada sistem. Selain itu pada halaman ini juga dapat melakukan manajemen serperti menambahkan atau menghapus baik \textit{deployment images} atauapun \textit{deployment plan}. Dari halaman ini pun, dapat diakses detail \textit{deployment images} maupun \textit{deployment plan} serta melakukan \textit{remote deployment}

  \item Halaman \textit{deployments detail}

        Halaman ini menunjukan riwayat \textit{deployment} apa saja yang telah dilakukan, statusnya serta target dari \textit{deployment}.

  \item Halaman \textit{faq}

        Halaman ini bertujuan untuk memberikan informasi mengenai tata cara hal yang perlu dilakukan sebelum mendaftarkan \textit{device} ke sistem


\end{enumerate}