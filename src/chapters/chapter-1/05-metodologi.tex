\section{Metodologi}

Terdapat metodologi yang digunakan untuk melaksanakan tugas akhir ini, berikut adalah tahapan pelaksanaan.
\begin{enumerate}
  \item \textbf{Identifikasi Permasalahan}

        Tahapan ini adalah tahapan untuk melakukan identifikasi permasalahan. Hasil dari tahapan ini dijadikan gagasan utama dan arah kerja dalam tugas akhir ini.

  \item \textbf{Analisis dan Perancangan Solusi}

        Setelah mengidentifikasi permasalahan, dilakukan analisis perancangan solusi yang bertujuan untuk mencari metode dan pendekatan yang dapat dikembangkan untuk menyelesaikan permasalahan yang ada. Analisis ini dimulai dari eksplorasi metode melalui studi literatur lalu dilanjutkan dengan penelitian yang pernah dilakukan.

  \item \textbf{Implementasi}

        Setelah merancang solusi, gagasan tersebut akan dikembangkan dan diimplementasikan. Hasil dari tahap ini berupa desain arsitektur yang optimal untuk digunakan pada lingkungan \textit{IoT} yang terintegrasi dengan \textit{service mesh}

  \item \textbf{Pengujian}

        Setelah implementasi berhasil dilakukan, maka hasil implementasi akan menuju rangkaian pengujian untuk memastikan kebenaran implementasi

  \item \textbf{Analisis}

        Setelah hasil implementasi diuji dengan baik, hasil tersebut akan menuju tahap analisis.

  \item \textbf{Evaluasi}

        Setelah berhasil dianalisis, akan dilakukan evalusi untuk memberikan suatu kesimpulan ataupun saran mengenai penelitian yang dilakukan
\end{enumerate}