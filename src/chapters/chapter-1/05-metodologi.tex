\section{Metodologi}

Terdapat beberapa metodologi yang digunakan untuk melaksanakan tugas akhir ini, berikut adalah tahapan pelaksanaannya:
\begin{enumerate}
  \item \textbf{Identifikasi Permasalahan}

        Pada tahap ini, dilakukan eksplorasi untuk mencari desain ataupun arsitektur secara kasar agar sistem \textit{remote deployment} dapat diimplementasi. Setelah itu dilakukan identifikasi permasalahan pada sistem tersebut agar sistem yang dibuat dapat memenuhi semua kebutuhan.

  \item \textbf{Analisis dan Perancangan Solusi}

        Setelah mengidentifikasi permasalahan, dilakukan analisis perancangan solusi yang bertujuan untuk mencari metode dan pendekatan yang dapat dikembangkan untuk menyelesaikan permasalahan \textit{remote deployment}. Analisis ini dimulai dari eksplorasi metode melalui studi literatur lalu dilanjutkan dengan penelitian yang pernah dilakukan.

  \item \textbf{Implementasi}

        Setelah merancang solusi, gagasan tersebut akan dikembangkan dan diimplementasikan. Hasil dari tahap ini berupa desain arsitektur yang memenuhi seluruh kebutuhan untuk sistem \textit{remote deployment} yang berjalan pada lingkungan IoT serta telah terintegrasi dengan Kubernetes

  \item \textbf{Pengujian dan Evaluasi}

        Setelah implementasi berhasil dilakukan, dilakukan serangkaian pengujian untuk memastikan kebenaran implementasi. Setelah pengujian dilakukan, hasil implementasi akan dievaluasi agar mendapatkan \textit{feedback} terkait hal-hal yang dapat ditingkatkan kedepannya.

\end{enumerate}