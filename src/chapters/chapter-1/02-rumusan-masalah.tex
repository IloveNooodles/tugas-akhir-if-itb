\section{Rumusan Masalah}

Berdasarkan latar belakang yang ada, rumusan tugas akhir ini adalah sebagai berikut.
\begin{enumerate}
  \item Bagimana cara meningkatkan performa layanan IoT khususnya pada \textit{Smart Home System} dengan menggunakan \textit{service mesh}?
  \item Bagaimana cara melakukan \textit{device discovery} yang optimal khususnya pada \textit{Smart Home System}?
  \item Bagaimana cara melakukan \textit{remote deployment} untuk setiap perangkat yang ada pada \textit{Smart Home System}?
\end{enumerate}


tujuan
1. optimizing edge gateway biar lebih cepet (raspi)
2. edge gateway gimana caranya biar bisa bikin software (orchestrating segala software disana, mainly instalasi remote sm healtcheck (agora))

- gateway ini buat nerusin data sensor (udah ada port masing2) pengen bisa di kelola pake machine learning pipeline dan ini tuh bakal di treat sebagai service berlayer2, outputnya di piepline ke service lain dan ini remote deployment semuanya.

- cannonical (edge gateway)

1. apakah ini novel (penting ga, studi literatur)
2. mahasiswa capable apa engga (progress)
3. mahasiswa udah progress apa blm

- edge computing platform

- Gaada, atau udah ada tapi sistemnya gabisa (benchmarking)
- implementasi. innetwrok computing, bedain OS dan arsitektur komputerntya. Bandingin reliabilitynya (cari metrics buat compare)