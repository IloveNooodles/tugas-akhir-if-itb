\section{Jadwal Pelaksanaan Tugas Akhir}

Konten dari Tugas Akhir ini akan dibagi menjadi lima bab sebagai berikut.
\begin{enumerate}
    \item \textbf{Pendahuluan}
    
    Pada Bab I akan dijelaskan gagasan utama dari tugas akhir ini yang berisi dari latar belakang, rumusan masalah, tujuan, batasan, metodologi hingga sistematika pembahasan.

    \item \textbf{Studi Literatur}
    
    Selanjutnya, Bab II akan menjelaskan hasil studi literatur yang berkaitan dengan pengerjaan tugas akhir ini. Bab II ini berisi tentang pemahaman dasar seputar topik yang akan dibahas pada tugas akhir ini.

    \item \textbf{Analisis Persoalan dan Rancangan Solusi}
    
    Pada Bab III akan dijelaskan analisis persoalan untuk menyusun rancangan solusi. Rancangan tersebut akan dijelaskan pada bab ini sebelum diimplementasikan. Rancangan solusi akan dipaparkan dalam bentuk diagram dan kajian dalam bab ini.

    \item \textbf{Implementasi dan Pengujian}
    
    Bab IV ini akan berisikan kajian terhadap implementasi yang telah dibuat. Bab ini juga akan membahas tahap-tahap pengujian dan hasilnya. Perbandingan beberapa model prediksi akan dibahas pada bab ini.

    \item \textbf{Kesimpulan dan Saran}
    
    Bab V akan menutup tugas akhir ini. Konten pada bab ini akan menjawab rumusan masalah. Bab ini juga akan menyebutkan saran-saran perbaikan yang bisa dipakai untuk penelitian berikutnya. Bab ini akan menyimpulkan hasil implementasi dan rancangan solusi terhadap masalah yang sudah diidentifikasi.
\end{enumerate}