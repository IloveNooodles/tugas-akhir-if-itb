\section{Latar Belakang}

Di era digital saat ini, konsep \textit{smart home} telah berkembang pesat, memberikan kemudahan dan efisiensi dalam kehidupan sehari-hari. Dengan pertumbuhan pesat \textit{Internet of Things} (\textit{IoT}), jumlah perangkat yang terhubung dalam sebuah \textit{smart home} meningkat secara signifikan, menghasilkan kompleksitas baru dalam manajemen dan operasionalnya. Perangkat-perangkat ini, yang berkisar dari sensor suhu hingga sistem keamanan cerdas, memerlukan komunikasi yang andal dan efisien untuk berfungsi secara optimal.

Kendala utama yang dihadapi saat ini adalah bagaimana mengelola komunikasi antar perangkat ini dengan cara yang efisien, aman, dan skalabel. Di sinilah peran dari \textit{service mesh} menjadi penting. \textit{service mesh} menawarkan solusi untuk mengelola komunikasi antar layanan dalam arsitektur mikroservis, yang semakin banyak diterapkan dalam sistem \textit{IoT}. Implementasi \textit{service mesh} dalam sistem \textit{IoT} \textit{smart home} memungkinkan komunikasi yang lebih efisien dan terkontrol antar perangkat, sekaligus menyediakan fitur seperti keamanan, pemantauan, dan penanganan kesalahan yang lebih baik.

Selain itu, \textit{service mesh} juga membantu dalam mengatasi masalah skalabilitas. Seiring bertambahnya jumlah perangkat \textit{IoT}, sistem harus mampu menyesuaikan diri tanpa mengurangi performa atau keamanan. Dengan \textit{service mesh}, dapat diciptakan sebuah sistem yang dinamis dan adaptif, mampu menangani penambahan atau pengurangan perangkat dengan mudah.

Pengintegrasian \textit{service mesh} dalam sistem \textit{IoT} \textit{smart home} tidak hanya meningkatkan efisiensi operasional, tetapi juga membuka peluang untuk inovasi lebih lanjut dalam bidang automasi rumah dan pengalaman pengguna yang lebih kaya.

Penelitian ini bertujuan untuk mengeksplorasi potensi integrasi \textit{service mesh} dalam sistem \textit{IoT} \textit{smart home}, mencari solusi untuk tantangan yang ada, dan mengusulkan arsitektur yang dapat diimplementasikan dalam skenario kehidupan nyata untuk mencapai rumah yang lebih cerdas, aman, dan efisien.