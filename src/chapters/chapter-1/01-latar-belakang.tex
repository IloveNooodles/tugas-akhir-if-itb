\section{Latar Belakang}

Di era digital saat ini, konsep \textit{smart home} telah berkembang pesat, memberikan kemudahan dan efisiensi dalam kehidupan sehari-hari. Dengan pertumbuhan pesat \textit{Internet of Things} \textit{(IoT)}, jumlah perangkat yang terhubung dalam sebuah \textit{smart home} meningkat secara signifikan, menghasilkan kompleksitas baru dalam manajemen dan operasionalnya. Perangkat-perangkat ini, yang berkisar dari sensor suhu, aktuator, tampilan, kendaraan, dan sebagainya akan semakin maju dengan adanya perkembangan pada sejumlah aplikasi yang menggunakan jumlah dan variasi data yang mungkin sangat besar untuk memberikan layanan baru kepada masyarakat, perusahaan, maupun individu \parencite{IOTSmartCity}.

Tiga Kendala utama yang dihadapi saat ini \textit{Perbedaan protokol komunikasi}, \textit{device discovery}, dan \textit{remote deployment}. Dengan munculnya berbagai macam perangkat \textit{IoT} memungkinkan kita untuk menggunakan metode yang berbeda beda juga dalam hal komunikasi antar layanan. Oleh karena itu perlu digunakan sebuah protokol standar dalam penggunaan IoT dalam \textit{smart home system}.

\textit{Device discovery} juga merupakan salah satu masalah yang ditemukan ketika melakukan pembuatan \textit{smart home system} dengan \textit{IoT}. Sulitnya untuk menemukan perangkat \textit{IoT} yang dapat diintegrasikan dengan sistem \textit{smart home} membuat perangkat harus di konfigurasi satu persatu, hal ini membuat proses pembuatan sistem menjadi terhambat. Masalah yang berkaitan dengan permasalahan ini juga yaitu \textit{Remote deployment}. Kedua masalah ini merupakan masalah yang berkaitan dengan skalabilitas ketika pembuatan \textit{smart home system} dengan \textit{IoT}.

Seiring bertambahnya jumlah perangkat \textit{IoT}, sistem harus mampu menyesuaikan diri tanpa mengurangi performa atau keamanan. Dengan kemudahan kedua proses tersebut, penamabahan perangkat \textit{IoT} baru dapat dilakukan dengan cepat dan mudah tanpa harus melakukan konfigurasi secara satu perasatu sehingga membuat proses pembuatan sistem menjadi lebih cepat, efektif, dan efisien.

Di sinilah peranan dari \textit{service mesh} menjadi sangat berpengaruh. \textit{service mesh} menawarkan solusi untuk mengelola komunikasi antar layanan serta mendukung proses standardisasi komunikasi yang lebih efisien dan terkontrol antar perangkat, sekaligus menyediakan fitur seperti keamanan, pemantauan, dan penanganan kesalahan yang lebih baik. Selain itu, \textit{service mesh} juga membantu dalam mengatasi masalah \textit{remote deployment} serta \textit{device discovery} yang sekaligus menyelesaikan permasalahan skalabilitas.

Penelitian ini bertujuan untuk membuat sebuah sistem \textit{Smart home system} berbasis \textit{service mesh} dengan \textit{Kubernetes} agar dapat membuat suatu solusi untuk tantangan yang telah disebutkan sebelumnya. Penelitian ini juga akan mengusulkan arsitektur yang dapat diimplementasikan dalam skenario kehidupan nyata untuk mencapai \textit{smart home system} yang lebih cerdas, aman, dan efisien serta memiliki performa yang tinggi.