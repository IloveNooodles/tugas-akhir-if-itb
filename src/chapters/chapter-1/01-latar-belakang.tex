\section{Latar Belakang}
\label{sec:latar-belakang}

Di era digital saat ini, \textit{Internet of Things} \textit{(IoT)} sudah menjadi bagian tak terpisahkan dari kehidupan manusia. Berbagai sistem dan aplikasi, mulai dari rumah cerdas hingga sistem parkir, telah mengintegrasikan IoT untuk memberikan kemudahan. Seiring dengan berkembangnya pengguna sistem, jumlah perangkat IoT yang dibutuhkan pun akan semakin banyak. Hal ini dapat menimbulkan masalah kompleksitas baru dalam manajemen dan operasionalnya \parencite{IOTSmartCity}.

Seiring bertambahnya jumlah pengguna perangkat \textit{Internet of Things (IoT)}, proses manajemen perangkat menjadi semakin lama dan kompleks. Salah satu proses yang penting dalam pengelolaan perangkat IoT adalah \textit{deployment} \parencite{OTAKeyPrinciples}. Menurut Cisco survey pada tahun 2017 kepada lebih dari 1500 perusahaan IT, 74\% memiliki masalah dalam hal ini \parencite{RemoteDeployment}.

Awalnya proses \textit{deployment} dilakukan secara \textit{wired} melalui kabel USB. Namun, seiring berkembangnya jaman, muncul teknologi baru untuk menyelesaikan masalah \textit{deployment} yaitu dilakukan secara \textit{wireless} melalui mekanisme \textit{Over the Air (OTA)}. Mekanisme ini sudah menyelesaikan beberapa masalah waktu ketika mengelola IoT terutama dalam melakukan proses \textit{deployment}. Meskipun metode OTA telah membawa banyak perubahan dan perbaikan, tantangan dalam manajemen perangkat yang skalabel tetap menjadi masalah \parencite{ElJaouhari2022}.

Menurut \parencite{RemoteDeployment}, terdapat 4 masalah utama pada saat proses \textit{remote deployment}. Yaitu perbedaan Jenis IoT, Sulitnya mendiagnosa \textit{device IoT} secara \textit{remote}, \textit{resource constraint} pada \textit{hardware}, serta masalah \textit{security}. Selain itu, menurut penelitian lain yang dilakukan oleh \parencite{studyovertheair1}, permasalahan utama dalam hal \textit{deployment} yaitu masih minimnya dokumentasi dari penyedia \textit{hardware} serta belum ada standardisasi proses untuk melakukan OTA. Hal ini membuat pengembang aplikasi menjadi kesulitan karena harus menyesuaikan proses \textit{deployment} untuk berbagai OEM (\textit{Original Equipment Manufacturer}). Selain itu, Ketika melakukan proses \textit{deployment} untuk banyak perangkat, \textit{deployment} yang dibuata tidak bisa diseragamkan. Hal ini diakibatkan oleh perbedaan kemampuan perangkat mulai dari daya komputasi, sensor ataupun aktuator yang dimiliki, serta jenis perangkat yang akan digunakan. Sehingga perlu adanya sistem yang dapat melakukan \textit{targeted deployment} untuk perangkat tertentu saja.

Penelitian ini berfokus untuk menyelesaikan masalah \textit{heterogen IoT} serta \textit{resource constraint} dalam proses \textit{remote deployment} pada skala besar. Dalam penelitian ini, akan dibangun PERISAI, yaitu sebuah sistem \textit{remote deployment} yang bersifat \textit{platform agnostic} serta \textit{low resource freindly}. PERISAI diharapkan dapat meningkatkan efektivitas dan skalabilitas pengelolaan perangkat IoT dalam jumlah besar. Dengan implementasi PERISAI, proses \textit{deployment} dapat dilakukan untuk berbagai jenis tipe serta dapat dijalankan meskipun memiliki batasan kemampuan komputasi. Selain itu dengan adanya PERISAI, diharapkan proses \textit{deployment} skala besar lebih mudah untuk dilakukan.