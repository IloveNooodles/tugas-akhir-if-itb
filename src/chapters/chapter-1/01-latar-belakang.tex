\section{Latar Belakang}
\label{sec:latar-belakang}

Di era digital saat ini, \textit{Internet of Things} \textit{(IoT)} sudah menjadi bagian tak terpisahkan dari kehidupan manusia. Berbagai sistem dan aplikasi, mulai dari rumah cerdas hingga sistem parkir, telah mengintegrasikan IoT untuk memberikan kemudahan dan efisiensi. Seiring dengan berkembangnya pengguna sistem, jumlah perangkat IoT yang dibutuhkan pun akan semakin banyak. Hal ini dapat menimbulkan masalah kompleksitas baru dalam manajemen dan operasionalnya \parencite{IOTSmartCity}.

Seiring bertambahnya jumlah perangkat \textit{IoT}, sistem harus mampu beradaptasi tanpa mengurangi kualitas dan keamanan. Kemudahan dalam proses adaptasi ini memungkinkan penambahan perangkat \textit{IoT} baru dengan cepat dan mudah, sehingga sistem menjadi lebih cepat, efektif, dan efisien. Namun, menurut \parencite{RemoteDeployment}, masih banyak sistem dan perangkat IoT yang belum mampu melakukan \textit{update} secara \textit{remote}. Hal ini mengakibatkan peningkatan proses operasional pada setiap perangkat \textit{IoT}

Di sinilah peran \textit{remote deployment} menjadi sangat penting. \textit{Remote deployment} menawarkan solusi untuk mengelola aplikasi secara terpusat melalui internet. Hal ini mendukung proses standardisasi komunikasi yang lebih efisien dan terkontrol antar perangkat, sehingga mengurangi proses miskonfigurasi serta waktu operasional untuk melakukan konfigurasi secara manual. Dengan ini, \textit{remote deployment} dapat membantu dalam mengatasi masalah skalabilitas.

Penelitian ini bertujuan untuk mengembangkan sistem \textit{remote deployment} dengan memanfaatkan \textit{Kubernetes}. \textit{Kubernetes} akan berperan dalam mengorkestrasi proses deployment perangkat \textit{IoT} secara terpusat dan efisien. Sistem ini diharapkan dapat meningkatkan efektivitas dan skalabilitas pengelolaan perangkat \textit{IoT} dalam jumlah besar. Dengan implementasi \textit{remote deployment}, kompleksitas dan biaya operasional dapat diminimumkan sehingga proses \textit{maintenance} perangkat \textit{IoT} untuk skala besar menjadi mudah untuk dilakukan.