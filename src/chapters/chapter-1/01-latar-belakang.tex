\section{Latar Belakang}

\textit{Information Retrieval} (IR) adalah metode penemuan bahan seperti dokumen yang bersifat terstruktur yang memenuhi kebutuhan informasi dari dalam koleksi besar yang tersimpan di dalam komputer \parencite{inforetrieval}. Metode IR sangat sering dijumpai contohnya dalam pencarian informasi berkaitan dengan representasi, penyimpanan, pengaturan, dokumen, halaman web, katalog \textit{online}, catatan, dan objek multimedia. Salah satu aplikasi yang membantu dalam melakukan hal tersebut adalah \textit{Elastic Search}.

\textit{Elastic Search} merupakan sistem mesin pencarian dan analitik terdistribusi yang dibangun di Apache Lucene. \textit{Elastic Search} dengan cepat menjadi sistem mesin pencari paling populer dan biasa digunakan untuk analisis log, pencarian teks lengkap, inteligensi keamanan, analisis bisnis, dan kasus penggunaan inteligensi operasional \parencite{elasticsearch}.

Proses \textit{Elastic Search} menggunakan JVM. Umumnya pada \textit{Elastic Search}, hampir 50 persen memori yang tersedia akan dialokasikan ke JVM. Pemrosesan Elasticsearch umumnya dilakukan pada memori. Sehingga, berdasarkan riset \parencite{jvm}, Pemrosesan data berukuran besar dalam in-memory computing pada JVM memakan sangat banyak memory. Dalam konteks \textit{Elastic Search}, memori sangat banyak dipakai karena Apache Lucene membutuhkan melakukan \textit{indexing} dan \textit{caching}. Pemberian memori yang tepat diperlukan untuk memberikan kinerja maksimal terhadap \textit{information retrieval system} agar \textit{indexing} dan \textit{caching} dapat berjalan secara maksimal. Namun, pada kondisi nyatanya, pengunaan memori untuk \textit{caching} dan \textit{indexing} tidak efisien untuk semua jenis data dan kondisi. Sebagai contoh, data yang jarang dicari sebenarnya opsional untuk dilakukan pengindeksan dan disimpan pada\textit{cache}. Harapannya, penggunaan memori yang tidak terlalu diperlukan dapat dibebaskan sehingga dapat dipakai oleh aplikasi atau keperluan lain yang lebih membutuhkan. Tidak hanya itu, proses \textit{indexing} sendiri sangat memakan sumber daya prosesor sehingga apabila ukuran \textit{cache} terlalu kecil hal ini akan berdampak menyibukkan prosesor untuk memproses permintaan \textit{query} yang masuk. Pengalokasian sumber daya terhadap \textit{Elastic Search} secara efisien harus menyesuaikan dengan kebutuhan dan kondisi yang ada.

Dalam mengelola infrastruktur aplikasi terdapat tiga buah hal yang saling berkorelasi. Tiga hal tersebut adalah biaya, kinerja, dan efisiensi. Biaya operasional dapat menjadi faktor penentu atau batasan yang signifikan saat mengelola infrastruktur. Sementara, kinerja aplikasi merupakan \textit{output} dari melakukan pengelolaan infrastruktur. Dan terakhir, efisiensi, adalah cara untuk mengoptimalkan dengan semua batasan yang ada. Dalam ketiga hal ini, tidak ada satu patokan yang bisa dipakai dalam mengatur biaya, kinerja, dan efisiensi. Setiap pengelola harus menemukan titik keseimbangan yang sesuai dengan kebutuhan dan prioritas masing-masing. Penentuan titik keseimbangan tersebut menjadi faktor untuk melakukan efisiensi karena tentunya kebutuhan dan prioritas setiap orang berbeda. Sulitnya menentukan titik keseimbangan tersebut ditambah permintaan untuk memaksimalkan kinerja berdasarkan variabel seperti \textit{throughput} pada operasi-operasi tertentu akan menambah kompleksitas pengelola dalam mengalokasikan sumber daya \textit{Elastic Search}. Oleh karena itu, diperlukannya sebuah sistem yang membantu pengelola untuk mengalokasikan sumber daya sekaligus menimalisir \textit{cost} infrastruktur yang dapat melihat dari pola penggunaan dan kebutuhan yang ada.