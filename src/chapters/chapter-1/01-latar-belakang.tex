\section{Latar Belakang}
\label{sec:latar-belakang}

Di era digital saat ini, \textit{Internet of Things} \textit{(IoT)} sudah menjadi bagian tak terpisahkan dari kehidupan manusia. Berbagai sistem dan aplikasi, mulai dari rumah cerdas hingga sistem parkir, telah mengintegrasikan IoT untuk memberikan kemudahan. Seiring dengan berkembangnya pengguna sistem, jumlah perangkat IoT yang dibutuhkan pun akan semakin banyak. Hal ini dapat menimbulkan masalah kompleksitas baru dalam manajemen dan operasionalnya \parencite{IOTSmartCity}.

Seiring bertambahnya jumlah pengguna perangkat \textit{Internet of Things (IoT)}, proses manajemen perangkat menjadi semakin lama dan kompleks. Salah satu proses yang penting dalam pengelolaan perangkat IoT adalah proses pembaruan perangkat \textit{(software updates)}. Menurut Cisco survey pada tahun 2017 kepada lebih dari 1500 perusahaan IT, 74\% memiliki masalah dalam hal ini \parencite{RemoteDeployment}. Awalnya proses \textit{software updates} dilakukan secara \textit{wired} melalui kabel USB. Namun, seiring berkembangnya jaman, muncul teknologi baru untuk menyelesaikan masalah \textit{software updates} yaitu dilakukan secara \textit{wireless} melalui mekanisme \textit{Over the Air (OTA)}. Mekanisme ini sudah menyelesaikan beberapa masalah waktu ketika mengelola \textit{IoT} terutama dalam melakukan proses \textit{software updates}. Meskipun metode OTA telah membawa banyak perubahan dan perbaikan, tantangan dalam manajemen perangkat yang skalabel tetap menjadi masalah. \parencite{ElJaouhari2022}

Menurut \parencite{RemoteDeployment}, terdapat 4 masalah utama dalam melakukan \textit{deployment} dan \textit{updates}. Perbedaan Jenis \textit{IoT}, Sulitnya mendiagnosa \textit{device IoT} secara \textit{remote}, \textit{resource constraint} pada \textit{hardware}, serta masalah \textit{security}. Selain itu, menurut \parencite{studyovertheair1} permasalahan utama dalam hal \textit{software updates} yaitu masih minimnya dokumentasi dari penyedia \textit{hardware} serta belum ada standardisasi proses untuk melakukan OTA. Oleh karena itu, perlu dibuat suatu standardisasi yang bersifat \textit{platform agnostic} untuk menyelesaikan masalah ini. 

Penelitian ini berfokus untuk menyelesaikan masalah \textit{heterogen IoT} serta \textit{resource constraint} dalam proses \textit{remote deployment}. Dalam penelitian ini, akan dibangun sebuah sistem \textit{remote deployment} yang bersifat \textit{platform agnostic} serta \textit{low resource freindly} dengan memanfaatkan \textit{Kubernetes} sebagai alat orkestrasi. \textit{Kubernetes} akan berperan dalam mengorkestrasi proses deployment perangkat \textit{IoT} secara terpusat dan efisien. Sistem ini diharapkan dapat meningkatkan efektivitas dan skalabilitas pengelolaan perangkat \textit{IoT} dalam jumlah besar. Dengan implementasi \textit{remote deployment}, proses \textit{software updates} dapat dilakukan untuk berbagai jenis tipe serta dapat dijalankan meskipun memiliki \textit{resource constraint}. Selain itu dengan adanya sistem ini, diharapkan proses \textit{software updates} skala besar lebih mudah untuk dilakukan.

% These run custom software that is usually platform-dependent and has to answer to numerous constraints due to limited hardware capability and low power-consumption requirements \parencite{RemoteDeployment}.

% Paganini, P. Faulty firmware OTA Update Bricked Hundreds of LockState Smart Locks. Available online: https://securityaffairs.co/wordpress/62043/hacking/smart-locks-faulty-firmware.html (accessed on 6 June 2020).

% One of the latest trends over recent years in the technological world, industry, and daily life has been the usage of Cyber-Physical Systems (CPS) and the Internet of Things (IoT) devices, which are present in fields such as industry, medicine, military, home automation, and many more [29]. The use of long-lived IoT and CPS has made them susceptible to obsolescence and change, just like “conventional” software, demanding systematic support for periodic updates of their embedded software. Some of these IoT and CPS systems require updates to be handled wirelessly and is in this context that Over-the-Air updates are used to address periodic updates of the embedded software. However, there is little documentation on how Original Equipment Manufacturers (OEMs) should design OTA firmware updates systems in a structured or standardized way. \parencite{studyovertheair1}

% A situation that forces OEMs to develop custom and not standardized OTA update systems solutions for their IoT and CPS developments \parencite{studyovertheair1}

% Seiring bertambahnya jumlah perangkat \textit{IoT}, sistem harus mampu beradaptasi tanpa mengurangi kualitas sistem. Kemudahan dalam proses adaptasi ini memungkinkan penambahan perangkat \textit{IoT} baru dengan cepat dan mudah, sehingga sistem menjadi lebih cepat, efektif, dan efisien. Namun, menurut \parencite{RemoteDeployment}, masih banyak sistem dan perangkat IoT yang belum mampu melakukan \textit{update} secara \textit{remote}. Hal ini mengakibatkan peningkatan proses operasional pada setiap perangkat \textit{IoT}