\section{Latar Belakang}

Dengan berkembangnya paradigma \textit{cloud computing}, arsitektur mikroservis telah menjadi pilihan populer bagi perusahaan yang ingin meningkatkan skalabilitas dan kecepatan pengembangan aplikasi mereka. Arsitektur mikroservis memungkinkan pengembang untuk mengembangkan, men-deploy, dan mengelola layanan secara independen, sehingga mempercepat proses development. Namun, komunikasi antara layanan yang independen ini menimbulkan tantangan serta kompleksitas baru. Setiap layanan perlu mengetahui tentang layanan lain serta harus dapat berkomunikasi secara aman, dan perlu di-monitor untuk memastikan kinerja yang optimal. Masalah ini semakin rumit dengan meningkatnya jumlah layanan yang harus diatur.

Sebagai lapisan infrastruktur yang terpisah, \textit{service mesh} memfasilitasi komunikasi interservis yang cepat, andal, dan aman. Dengan menambahkan abstraksi jaringan di tingkat infrastruktur, \textit{service mesh} membebaskan pengembang dari beban mengelola aspek jaringan, memungkinkan mereka untuk fokus pada logika bisnis aplikasi. Biasanya service mesh hadir dengan menggunakan \textit{sidecar proxies} yang ditempatkan bersama dengan setiap layanan. \textit{Service mesh} memberdayakan fitur-fitur seperti \textit{service discovery}, \textit{load balancing}, \textit{failure recovery}, \textit{metrics analysis}, dan \textit{monitoring} tanpa perubahan kode yang signifikan pada layanan itu sendiri.

Namun, penggunaan \textit{service mesh} memiliki tantangan untuk diimplementasi. Implementasi dan pengelolaan infrastruktur ini memerlukan pemahaman yang mendalam tentang topologi jaringan dan konfigurasi yang tepat untuk mendapatkan manfaat maksimal. Solusi \textit{service mesh} populer seperti Istio, Linkerd, dan Consul memiliki terbilang cukup sulit untuk dipelajari dan biasanya akan memberikan latensi tambahan dalam komunikasi interservis. Oleh karena itu, penting untuk mengevaluasi \textit{trade-off} antara kompleksitas yang dikelola dan keuntungan yang diperoleh dari penerapan \textit{service mesh} dalam lingkungan produksi.

Melihat pentingnya \textit{service mesh} dalam menangani kompleksitas yang dihasilkan oleh arsitektur mikroservis, penelitian ini bertujuan untuk memberikan wawasan komprehensif tentang bagaimana teknologi ini beroperasi dan nilai yang ditambahkannya ke dalam sistem distribusi saat ini. Melalui penelitian ini, akan dijelaskan mekanisme kerja \textit{service mesh}, seberapa besar \textit{service mesh} berpengaruh terhadap arsitektur sistem yang ada, dan apa saja pertimbangan penting ketika memilih dan mengimplementasikan solusi \textit{service mesh} yang tepat untuk kebutuhan organisasi.