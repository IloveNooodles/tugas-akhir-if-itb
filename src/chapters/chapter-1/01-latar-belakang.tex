\section{Latar Belakang}
\label{sec:latar-belakang}

Di era digital saat ini, \textit{Internet of Things} \textit{(IoT)} sudah menjadi bagian tak terpisahkan dari kehidupan manusia. Berbagai sistem dan aplikasi, mulai dari rumah cerdas hingga sistem parkir, telah mengintegrasikan IoT untuk memberikan kemudahan. Seiring dengan berkembangnya pengguna sistem, jumlah perangkat IoT yang dibutuhkan pun akan semakin banyak. Hal ini dapat menimbulkan masalah kompleksitas baru dalam manajemen dan operasionalnya \parencite{IOTSmartCity}.

Seiring bertambahnya jumlah pengguna perangkat \textit{Internet of Things (IoT)}, proses manajemen perangkat menjadi semakin lama dan kompleks. Salah satu proses yang penting dalam pengelolaan perangkat IoT adalah proses pembaruan perangkat \textit{(software updates)} \parencite{OTAKeyPrinciples}. Menurut Cisco survey pada tahun 2017 kepada lebih dari 1500 perusahaan IT, 74\% memiliki masalah dalam hal ini \parencite{RemoteDeployment}. Awalnya proses \textit{software updates} dilakukan secara \textit{wired} melalui kabel USB. Namun, seiring berkembangnya jaman, muncul teknologi baru untuk menyelesaikan masalah \textit{software updates} yaitu dilakukan secara \textit{wireless} melalui mekanisme \textit{Over the Air (OTA)}. Mekanisme ini sudah menyelesaikan beberapa masalah waktu ketika mengelola IoT terutama dalam melakukan proses \textit{software updates}. Meskipun metode OTA telah membawa banyak perubahan dan perbaikan, tantangan dalam manajemen perangkat yang skalabel tetap menjadi masalah. \parencite{ElJaouhari2022}

Menurut \parencite{RemoteDeployment}, terdapat 4 masaxlah utama dalam melakukan \textit{deployment} dan \textit{updates}. Perbedaan Jenis IoT, Sulitnya mendiagnosa \textit{device IoT} secara \textit{remote}, \textit{resource constraint} pada \textit{hardware}, serta masalah \textit{security}. Selain itu, menurut \parencite{studyovertheair1} permasalahan utama dalam hal \textit{software updates} yaitu masih minimnya dokumentasi dari penyedia \textit{hardware} serta belum ada standardisasi proses untuk melakukan OTA. Oleh karena itu, perlu dibuat suatu standardisasi yang bersifat \textit{platform agnostic} untuk menyelesaikan masalah ini.

Penelitian ini berfokus untuk menyelesaikan masalah \textit{heterogen IoT} serta \textit{resource constraint} dalam proses \textit{remote deployment}. Dalam penelitian ini, akan dibangun sebuah sistem \textit{remote deployment} yang bersifat \textit{platform agnostic} serta \textit{low resource freindly} dengan memanfaatkan Kubernetes sebagai alat orkestrasi. Kubernetes akan berperan dalam mengorkestrasi proses deployment perangkat IoT secara terpusat dan efisien. Sistem ini diharapkan dapat meningkatkan efektivitas dan skalabilitas pengelolaan perangkat IoT dalam jumlah besar. Dengan implementasi \textit{remote deployment}, proses \textit{software updates} dapat dilakukan untuk berbagai jenis tipe serta dapat dijalankan meskipun memiliki \textit{resource constraint}. Selain itu dengan adanya sistem ini, diharapkan proses \textit{software updates} skala besar lebih mudah untuk dilakukan.