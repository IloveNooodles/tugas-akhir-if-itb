\section{Batasan Masalah}
\label{sec:batasan-masalah}

Terdapat batasan yang diambil dalam pelaksanaan tugas akhir ini, yaitu sebagai berikut.

\begin{enumerate}
    \item Implementasi \textit{Autoscale} dengan kontrol fleksibel akan fokus pada level \textit{pods} Kubernetes atau \textit{node} pada \textit{Elastic Search}.
    \item Pengguna dapat menetapkan sejumlah \textit{rule} yang akan digunakan untuk mengontrol alokasi sumber daya \textit{pods} \textit{Elastic Search}.
    \item Tidak melakukan riset perbandingan model prediksi satu dengan yang lainnya.
    \item Penentuan \textit{rule} untuk mengubah alokasi sumber daya tidak akan ditentukan secara otomatis melainkan ditentukan oleh pengguna. Hal ini disebabkan oleh kebutuhan serta karakteristik penggunaan \textit{Elastic Search} yang dapat berbeda antara pengguna.
    \item Tidak melakukan optimasi model prediksi terutama saat data sudah sangat besar.
    \item \textit{ElasticSearch} yang dipakai dalam pengembangan dan pengujian dibuat didalam \textit{cluster} Kubernetes lokal dengan batasan konfigurasi:
        \begin{enumerate}
            \item \textit{Node} Kubernetes hanya terdiri dari satu \textit{pod Elastic Search}.
            \item \textit{Elastic Search} dibuat dari kosong menggunakan \textit{image} yang disediakan oleh docker.
            \item Versi \textit{Elastic Search} yang digunakan adalah versi 8.8.1.
            \item \textit{Elastic Search} hanya memiliki satu \textit{shard} dan satu \textit{replica}.
            \item \textit{Elastic Search Cluster} berada di pod yang sama.
            \item Data \textit{Elastic Search} diletakkan dalam sebuah \textit{Persistent Volume Claim}.
        \end{enumerate}
 \end{enumerate}

