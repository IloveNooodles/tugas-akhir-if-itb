\chapter{Tabel Pengujian}

\bgroup
\begin{table}[ht]
  \def\arraystretch{1.5}
  \caption{Skenario dan Hasil Pengujian Domain \textit{Company}}
  \label{tab:pengujian-domain-company}
  \centering
  \begin{tabular}{|p{2cm}|p{2cm}|p{3cm}|p{3cm}|p{1.5cm}|}
    \hline
    \centering{ID Fungsional} & \centering{ID Pengujian} & \centering{Skenario}                                                                 & \centering{Ekspektasi}                                                                             & Realita \\
    \hline
    F01                       & P01                      & Admin ingin mendaftarkan \textit{company} dengan cluster name yang tersedia          & Admin berhasil mendaftarkan \textit{company}                                                       & Sesuai  \\
    \hline
    F01                       & P02                      & Admin ingin mendaftarkan \textit{company} dengan cluster name yang tidak valid       & Admin tidak berhasil mendaftarkan karena cluster name tidak valid                                  & Sesuai  \\
    \hline
    F02                       & P03                      & Admin membuat \textit{request} untuk melihat seluruh \textit{company} pada sistem    & Admin dapat melihat seluruh \textit{company} pada sistem                                           & Sesuai  \\
    \hline
    F03                       & P04                      & Admin melakukan DELETE \textit{request} untuk menghapus \textit{company} pada sistem & Admin dapat menghapus \textit{company} pada sistem dengan memberikan id company yang ingin dihapus & Sesuai  \\
    \hline
  \end{tabular}
\end{table}
\egroup


\bgroup
\begin{table}[ht]
  \def\arraystretch{1.3}
  \caption{Skenario dan Hasil Pengujian Domain \textit{User}}
  \label{tab:pengujian-domain-user}
  \centering
  \begin{tabular}{|p{2cm}|p{2cm}|p{3cm}|p{3cm}|p{1.5cm}|}
    \hline
    \centering{ID Fungsional} & \centering{ID Pengujian} & \centering{Skenario}                                                                           & \centering{Ekspektasi}                                                                                    & Realita \\
    \hline
    F04                       & P05                      & Admin mendaftarkan \textit{user} dengan kredensial yang valid                                  & Admin berhasil mendaftarkan \textit{user}                                                                 & Sesuai  \\
    \hline
    F05                       & P06                      & Admin menghapus \textit{user} dengan id yang valid                                             & Admin berhasil menghapus \textit{user} dari \textit{database}                                             & Sesuai  \\
    \hline
    F06                       & P07                      & \textit{User} ingin login ke dalam sistem dengan memasukan kredensial yang valid               & User berhasil login ke dalam sistem                                                                       & Sesuai  \\
    \hline
    F06                       & P08                      & \textit{User} ingin login ke dalam sistem dengan memasukan kredensial yang tidak valid         & User tidak berhasil login ke dalam sistem                                                                 & Sesuai  \\
    \hline
    F07                       & P09                      & \textit{User} ingin melakukan logout dari sistem                                               & \textit{User} berhasil melakukan logout dengan menekan tombol \textit{logout} pada \textit{sidebar}       & Sesuai  \\
    \hline
    F08                       & P10                      & \textit{User} mendapatkan informasi mengenai detail \textit{company} perusahaan yang ia masuki & {User} berhasil mendapatkan informasi detail \textit{company} dengan cara mengunjungi halaman /account    & Sesuai  \\
    \hline
    F09                       & P11                      & \textit{User} mendaptkan daftar \textit{user} lain pada \textit{company} yang sama.            & \textit{User} berhasil mendapatkan informasi datar \textit{user} dengan cara mengunjungi halaman /account & Sesuai  \\
    \hline
  \end{tabular}
\end{table}
\egroup

\bgroup
\begin{table}[ht]
  \def\arraystretch{1.3}
  \caption{Skenario dan Hasil Pengujian Domain \textit{Device}}
  \label{tab:pengujian-domain-device}
  \centering
  \begin{tabular}{|p{2cm}|p{2cm}|p{3cm}|p{3cm}|p{1.5cm}|}
    \hline
    \centering{ID Fungsional} & \centering{ID Pengujian} & \centering{Skenario}                                                                                 & \centering{Ekspektasi}                                                    & Realita \\
    \hline
    F10                       & P12                      & \textit{User} membuat \textit{device} yang valid                                                     & \textit{User} berhasil mendaftarkan \textit{device} kepada sistem         & Sesuai  \\
    \hline
    F10                       & P13                      & {User} membuat \textit{device} yang tidak valid                                                      & \textit{User} gagal mendaftarkan \textit{device} kepada sistem            & Sesuai  \\
    \hline
    F11                       & P14                      & \textit{User} ingin melihat daftar \textit{device} yang tersedia dengan mengunjungi halaman /devices & User berhasil melihat daftar \textit{device} yang tersedia                & Sesuai  \\
    \hline
    F12                       & P15                      & \textit{User} ingin menghapus salah satu \textit{device} yang dimiliki                               & \textit{User} berhasil untuk menghapus \textit{device} yang ingin dihapus & Sesuai  \\
    \hline
  \end{tabular}
\end{table}
\egroup


\bgroup
\begin{table}[ht]
  \def\arraystretch{1.3}
  \caption{Skenario dan Hasil Pengujian Domain \textit{Groups}}
  \label{tab:pengujian-domain-groups}
  \centering
  \begin{tabular}{|p{2cm}|p{2cm}|p{3cm}|p{3cm}|p{1.5cm}|}
    \hline
    \centering{ID Fungsional} & \centering{ID Pengujian} & \centering{Skenario}                                                                               & \centering{Ekspektasi}                                                   & Realita \\
    \hline
    F13                       & P16                      & \textit{User} membuat \textit{group} yang dengan nama "group-baru"                                 & \textit{User} berhasil mendaftarkan \textit{group} kepada sistem         & Sesuai  \\
    \hline
    F13                       & P17                      & {User} membuat \textit{group} yang memiliki nama duplikat yaitu "group-baru"                       & \textit{User} gagal mendaftarkan \textit{group} kepada sistem            & Sesuai  \\
    \hline
    F14                       & P18                      & \textit{User} ingin melihat daftar \textit{group} yang tersedia dengan mengunjungi halaman /groups & User berhasil melihat daftar \textit{group} yang tersedia                & Sesuai  \\
    \hline
    F15                       & P19                      & \textit{User} ingin menghapus salah satu \textit{group} yang dimiliki                              & \textit{User} berhasil untuk menghapus \textit{group} yang ingin dihapus & Sesuai  \\
    \hline
  \end{tabular}
\end{table}
\egroup

\bgroup
\begin{table}[ht]
  \def\arraystretch{1.3}
  \caption{Skenario dan Hasil Pengujian Sistem \textit{RaspberryPi}}
  \label{tab:pengujian-sistem-raspi}
  \centering
  \begin{tabular}{|p{2cm}|p{2cm}|p{3cm}|p{3cm}|p{1.5cm}|}
    \hline
    \centering{ID Fungsional} & \centering{ID Pengujian}                            & \centering{Skenario}                                                                                 & \centering{Ekspektasi}                                                                    & Realita \\
    \hline
    F01                       & P20                                                 & \textit{Admin} membuat \textit{company} yang dengan nama "raspi-company"                             & \textit{Admin} berhasil mendaftarkan \textit{company}                                     & Sesuai  \\
    \hline
    F04                       & P21                                                 & \textit{Admin} membuat \textit{user} yang dengan email "raspi@gmail.com"                             & \textit{Admin} berhasil mendaftarkan \textit{user}                                        & Sesuai  \\
    \hline
    F06                       & P22                                                 & \textit{User} ingin login dengan kredensial "raspi@gmail.com"                                        & \textit{User} berhasil login ke dalam sistem                                              & Sesuai  \\
    \hline
    F10                       & P23                                                 & \textit{User} ingin mendaftarkan dua \textit{devices} dengan nama "raspi-master" dan
    raspi-worker"             & \textit{User} berhasil mendaftarkan \textit{device} & Sesuai                                                                                                                                                                                                     \\
    \hline
    F13                       & P24                                                 & \textit{User} ingin mendaftarkan \textit{group} dengan nama "raspi-group-blink"                      & \textit{User} berhasil mendaftarkan \textit{group}                                        & Sesuai  \\
    \hline
    F16                       & P25                                                 & \textit{User} ingin mendaftarkan \textit{device} pada \textit{group} dengan nama "raspi-group-blink" & \textit{User} berhasil mendaftarkan \textit{device} ke \textit{group} "raspi-group-blink" & Sesuai  \\
    \hline
    F17                       & P26                                                 & \textit{User} ingin mendaftarkan \textit{deployment images} dengan nama "raspi-image-blink"          & \textit{User} berhasil mendaftarkan \textit{deployment images}                            & Sesuai  \\
    \hline
  \end{tabular}
\end{table}
\egroup

\bgroup
\begin{table}[ht]
  \def\arraystretch{1.3}
  \centering
  \begin{tabular}{|p{2cm}|p{2cm}|p{3cm}|p{3cm}|p{1.5cm}|}
    \hline
    \centering{ID Fungsional} & \centering{ID Pengujian} & \centering{Skenario}                                                                                                                                                                                                                  & \centering{Ekspektasi}                                                 & Realita \\
    \hline

    F18                       & P27                      & \textit{User} melihat daftar \textit{deployment images} yang tersedia dengan mengunjungi halaman /deployments                                                                                                                         & \textit{User} berhasil melihat daftar \textit{deployment images}       & Sesuai  \\
    \hline
    F20                       & P28                      & \textit{User} membuat \textit{deployment plan} dengan nama "raspi-deployment-blink"                                                                                                                                                   & \textit{User} berhasil untuk membat \textit{deployment plan}           & Sesuai  \\
    \hline
    F21                       & P29                      & \textit{User} melihat daftar \textit{deployment plan} yang tersedia dengan mengunjungi halaman /deployments                                                                                                                           & \textit{User} berhasil melihat daftar \textit{deployment plan}         & Sesuai  \\
    \hline
    F23                       & P30                      & \textit{User} melakukan \textit{remote deployment} dengan menggunakan \textit{deployment plan} "raspi-deployment-blink" dengan dua tipe \textit{deployment} yaitu \textit{targeted} dan \textit{custom} dengan tujuan \textit{groups} & \textit{User} berhasil melakukan \textit{remote deployment}            & Sesuai  \\
    \hline
    F24                       & P31                      & \textit{User} ingin melihat riwayat deployment dari "raspi-blink-deployment"                                                               dengan menekan tombol elipsis dan tombol detail pada tabel                                 & \textit{User} berhasil melihat riwayat \textit{deployment} yang sesuai & Sesuai  \\
    \hline
  \end{tabular}
\end{table}
\egroup

\bgroup
\begin{table}[ht]
  \def\arraystretch{1.3}
  \caption{Skenario dan Hasil Pengujian Sistem pada \textit{Cluster GCP}}
  \label{tab:pengujian-sistem-gcp}
  \centering
  \begin{tabular}{|p{2cm}|p{2cm}|p{3cm}|p{3cm}|p{1.5cm}|}
    \hline
    \centering{ID Fungsional} & \centering{ID Pengujian} & \centering{Skenario}                                                                                              & \centering{Ekspektasi}                                                   & Realita \\
    \hline
    F01                       & P32                      & \textit{Admin} membuat \textit{company} yang dengan nama "test-semhas"                                            & \textit{Admin} berhasil mendaftarkan \textit{company}                    & Sesuai  \\
    \hline
    F04                       & P33                      & \textit{Admin} membuat \textit{user} yang dengan email  "test@gmail.com"                                          & \textit{Admin} berhasil membuat \textit{user} dengan email yang sesuai   & Sesuai  \\
    \hline
    F06                       & P34                      & \textit{User} ingin melakukan login dengan email  "test@gmail.com"                                                & \textit{User} berhasil melakukan login                                   & Sesuai  \\
    \hline
    F10                       & P35                      & \textit{User} ingin mendaftarkan \textit{device} dengan nama "raspberrypi-pi-1" dengan nama node "master-cluster" & \textit{User} berhasil mendaftarkan \textit{device}                      & Sesuai  \\
    \hline
    F18                       & P36                      & \textit{User} melihat daftar  \textit{deployment images} yang tersedia dengan mengunjungi halaman /deployments    & \textit{User} berhasil melihat daftar seluruh \textit{deployment images} & Sesuai  \\
    \hline
    F20                       & P37                      & \textit{User} membuat \textit{deployment plan} dengan nama "deploy-mqtt-client"                                   & \textit{User} berhasil untuk membuat \textit{deployment} \textit{plan}   & Sesuai  \\
    \hline
    F22                       & P38                      & \textit{User} menghapus \textit{deployment plan} dengan nama "deploy-mqtt-client"                                 & \textit{User} berhasil melihat menghapus \textit{deployment plan}        & Sesuai  \\
    \hline
    F19                       & P39                      & \textit{User} menghapus satu \textit{deployment images} dengan nama mqtt-client                                   & \textit{User} berhasil menghapus \textit{deployment images}              & Sesuai  \\
    \hline
  \end{tabular}
\end{table}
\egroup

\bgroup
\begin{table}[ht]
  \def\arraystretch{1.3}
  \caption{Skenario dan Hasil Pengujian Sistem degan Kasus Gagal}
  \label{tab:pengujian-sistem-gagal}
  \centering
  \begin{tabular}{|p{2cm}|p{2cm}|p{3cm}|p{3cm}|p{1.5cm}|}
    \hline
    \centering{ID Fungsional} & \centering{ID Pengujian} & \centering{Skenario}                                                                                                                & \centering{Ekspektasi}                                                                                                            & Realita \\
    \hline
    F06                       & P40                      & \textit{User} ingin melakukan login dengan email  "test@gmail.com"                                                                  & \textit{User} berhasil melakukan login                                                                                            & Sesuai  \\
    \hline
    F17                       & P41                      & \textit{User} ingin mendaftarkan \textit{deployment images} dengan nama "nonexist-image"                                            & \textit{User} berhasil melakukan mendaftarkan \textit{deployment images}                                                          & Sesuai  \\
    \hline
    F18                       & P42                      & \textit{User} melihat daftar  \textit{deployment images} yang tersedia dengan mengunjungi halaman /deployments                      & \textit{User} berhasil melihat daftar seluruh \textit{deployment images}                                                          & Sesuai  \\
    \hline
    F20                       & P43                      & \textit{User} membuat \textit{deployment plan} dengan nama "deployment-plan-mqtt-client-failed"                                     & \textit{User} berhasil untuk membuat \textit{deployment} \textit{plan}                                                            & Sesuai  \\
    \hline
    F20                       & P44                      & \textit{User} membuat \textit{deployment plan} dengan nama "deployment-image"                                                       & \textit{User} berhasil untuk membuat \textit{deployment} \textit{plan}                                                            & Sesuai  \\
    \hline
    F23                       & P45                      & \textit{User} melakukan targeted \textit{remote deployment} dengan menggunakan deployment plan "deployment-plan-mqtt-client-failed" & \textit{User} gagal dalam proses \textit{remote deployment} karena tidak ditemukan label "temperatur=false" pada \textit{devices} & Sesuai  \\
    \hline
    F23                       & P46                      & \textit{User} \textit{User} melakukan targeted \textit{remote deployment} dengan menggunakan deployment plan "deployment-plan"      & \textit{User} gagal dalam proses \textit{remote deployment} karena tidak ditemukan image "nonexist-image" pada docker hub         & Sesuai  \\
    \hline
  \end{tabular}
\end{table}
\egroup