\chapter{Studi Literatur}

Bab Studi Literatur digunakan untuk mendeskripsikan kajian literatur yang terkait dengan persoalan tugas akhir. Tujuan studi literatur adalah:

\begin{enumerate}
    \item menunjukkan kepada pembaca adanya gap seperti pada rumusan masalah yang memang belum terselesaikan,
    \item memberikan pemahaman yang secukupnya kepada pembaca tentang teori atau pekerjaan terkait yang terkait langsung dengan penyelesaian persoalan, serta
    \item menyampaikan informasi apa saja yang sudah ditulis/dilaporkan oleh pihak lain (peneliti/Tugas Akhir/Tesis) tentang hasil penelitian/pekerjaan mereka yang sama atau mirip kaitannya dengan persoalan tugas akhir.
\end{enumerate}


\section{Contoh Subbab}
Perujukan literatur dapat dilakukan dengan menambahkan entri baru di berkas. Tulisan ini merujuk pada \parencite{knuth2001art,vasp1} atau \parencite{4026885} dan \parencite{Kim2006}

Sekarang mau ke bab berapa yaaaa.... hmm... ke bab \ref{sec:latarbelakang} ahhhhh. 

\blindtext

\subsection{Contoh Subsubbab}

\blindtext

\begin{figure}[h]
    \centering
    \includegraphics[width=0.8\textwidth]{chapter-2-infrastructure-diagram.png}
    \caption{Contoh gambar}
\end{figure}

\subsubsection{Subsubsubbab}

\blindtext

\begin{table}[h]
    \caption{Tabel random}
    \vspace{0.25cm}
    \begin{center}
        \begin{tabular}{|c|c|c|c|}
            \hline
            Title1 & Title2 & Title3 & Title4  \tabularnewline
            \hline
            1647   & 1.97   & 0.68   & 1.90 \tabularnewline
            2301   & 2.92   & 1.06   & 2.75 \tabularnewline
            2969   & 3.23   & 1.16   & 3.78 \tabularnewline
            3791   & 4.39   & 1.40   & 4.14 \tabularnewline
            4625   & 6.72   & 1.87   & 5.59 \tabularnewline
            \hline
        \end{tabular}
    \end{center}
\end{table}

\section{Menyisipkan Persamaan}

Beberapa contoh menyisipkan persamaan.


\subsection{Contoh Bikin Equation}
\textbf{text tebal} dan ini \emph{miring}, bikin persamaan di baris yang sama, tinggal pake dolar2 $\Psi(\vec{r}_1,...,\vec{r}_N)$, sehingga persamaan Schr\"{o}dinger, terus, persamaan yang dinomeri kayak gini 
%ini contoh bikin persamaan, ..... :D
\begin{equation}
    \left[ \sum_{i}^{N}-\frac{\hbar^2}{2m}\nabla_i^2 + \sum_{i}^{N}V(\vec{r}_i)+ \sum_{i<j}^{N}(\vec{r}_i,\vec{r}_j)\right]\Psi = E\Psi 
\end{equation}

untuk $N$-elektron, dengan $\hat{H}$=Hamiltonian, $E$=Energi total, $\hat{T}$=Energi kinetik, $\hat{V}$=Energi potensial, dan $\hat{U}$=Interaksi ektron-elektron.

\subsection{Bikin Matrix}
Lalalallala.... bikin matrix sekarang, yang ini dikecilin, pake smaller
    {\smaller
        \begin{equation}
            \Psi({\bf r}_1, {\bf r}_2, \cdots {\bf r}_N) = \frac{1}{\sqrt{N!}}\left| \begin{array}{llcl}
                \phi_1({\bf r}_1)     & \phi_2({\bf r}_1)     & \cdots                & \phi_N({\bf r}_1)     \\
                \phi_1({\bf r}_2)     & \phi_2({\bf r}_2)     & \cdots                & \phi_N({\bf r}_2)     \\
                \phi_1({\bf r}_3)     & \phi_2({\bf r}_3)     & \cdots                & \phi_N({\bf r}_3)     \\
                \multicolumn{1}{c}{.} & \multicolumn{1}{c}{.} & \multicolumn{1}{c}{.} & \multicolumn{1}{c}{.} \\
                \multicolumn{1}{c}{.} & \multicolumn{1}{c}{.} & \multicolumn{1}{c}{.} & \multicolumn{1}{c}{.} \\
                \multicolumn{1}{c}{.} & \multicolumn{1}{c}{.} & \multicolumn{1}{c}{.} & \multicolumn{1}{c}{.} \\
                \phi_1({\bf r}_N)     & \phi_2({\bf r}_N)     & \cdots                & \phi_N({\bf r}_N)     \\
            \end{array} \right|
        \end{equation}
    }

