\clearpage
\chapter*{ABSTRACT}
\addcontentsline{toc}{chapter}{Abstract}

\begin{center}
  \center
  \begin{singlespace}
    \large\bfseries\MakeUppercase{Design and Implementation of Remote Deployment System for IoT Devices}
    
    \normalfont\normalsize
    By:
    
    \bfseries \theauthor
  \end{singlespace}
\end{center}


\begin{singlespace}
  \small
  As the number of Internet of Things (IoT) device users increases, the device management process becomes more time-consuming and complex. One critical process in IoT device management is the deployment process, which is typically carried out serially from one device to another. This method is highly time-consuming and requires resources that could be better utilized for other management processes. Effective management of the deployment process is essential to ensure the reliability and optimal performance of the growing IoT network. With the advancement of IoT technology, the Over The Air (OTA) method has been introduced, allowing deployments to be done remotely. Despite the many improvements brought by OTA, scalable device management remains a challenge.
  
  This study aims to design a remote deployment system to address the issues found in the OTA method. The designed system must be platform agnostic, meaning it can run on any platform, and it must be capable of operating on resource-constrained devices to manage a large number of IoT applications. By addressing these requirements, the system seeks to provide an efficient and scalable solution for IoT device management.
  
  The implementation of the remote deployment system was successfully achieved using Kubernetes as the orchestration tool. Kubernetes, which has proven effective in managing containerized applications in cloud environments, can be implemented in IoT environments using the K3s distribution to address resource constraints. With the help of Kubernetes, the deployment process can be executed on any device with containerization features. The implemented system allows remote deployment across various platforms, from Raspberry Pi and virtual machines to local clusters created with containers. Deployment was carried out on two Raspberry Pi devices running as a Kubernetes cluster to control two lights connected to each Raspberry Pi. As a result, the remote deployment process was successfully conducted even on resource-constrained Raspberry Pi devices.
  
  \textbf{\textit{Keywords: IoT, Raspberry Pi, Remote Deployment, Kubernetes }}
\end{singlespace}
\clearpage

\clearpage