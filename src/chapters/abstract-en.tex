\clearpage
\chapter*{ABSTRACT}
\addcontentsline{toc}{chapter}{Abstract}

\begin{center}
  \center
  \begin{singlespace}
    \large\bfseries\MakeUppercase{PERISAI: DESIGN AND IMPLEMENTATION OF REMOTE DEPLOYMENT SYSTEM FOR IOT APPLICATIONS}
    
    \normalfont\normalsize
    By:
    
    \bfseries \theauthor
  \end{singlespace}
\end{center}


\begin{singlespace}
  \small
  As the number of Internet of Things (IoT) device users increases, the device management process becomes longer and more complex. One of the crucial processes in IoT device management is deployment. This process is usually carried out serially from one device to another, which is very time-consuming and labor-intensive, resources that could otherwise be utilized for other management tasks. Managing the deployment process in IoT is key to ensuring the reliability and optimal performance of the growing network. With the development of IoT technology, the Over The Air (OTA) method has been introduced, allowing deployments to be carried out remotely. Although the OTA method has brought many improvements, scalable device management remains a challenge.
  
  This research aims to design a remote deployment system to address the issues found in the OTA method. The system must be platform-agnostic, able to run on any platform, capable of operating on devices with limited resources to manage a large number of IoT applications, and able to deploy to specific devices. By meeting these requirements, the system can provide an efficient solution and be used for large-scale IoT device management.
  
  This research present PERISAI, an implementation of a remote deployment system using Kubernetes as the orchestration tool. Kubernetes, which has proven effective in managing containerized applications in the cloud environment, can be implemented in IoT environments using the K3s distribution to address resource-constrained issues. With the help of Kubernetes, the deployment process can be run on all devices with containerization features. With PERISAI, the remote deployment process can be carried out on all platforms, from Raspberry Pi, virtual machines, to local clusters created with containers. Deployment was carried out on two Raspberry Pis running as a Kubernetes cluster to turn on two lamps connected to each Raspberry Pi. As a result, the remote deployment process can be carried out successfully even on Raspberry Pis with limited resources.
  
  \textbf{\textit{Keywords: IoT, Raspberry Pi, Remote Deployment, Kubernetes }}
\end{singlespace}
\clearpage

\clearpage