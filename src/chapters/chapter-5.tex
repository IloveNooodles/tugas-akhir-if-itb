\chapter{Penutup}

Bab Kesimpulan dan Saran akan menjadi bagian akhir dan penutup dari penelitian tugas akhir ini. Bab ini akan membahas kesimpulan yang berisi ketercapaian tujuan penelitian tugas akhir dengan permasalahan yang diselesaikan dalam penelitian tugas akhir. Selain itu, bab ini akan membahas saran yang dapat dilakukan untuk pengembangan atau penelitian selanjutnya.

\section{Kesimpulan}
Penelitian tugas akhir ini mengimplementasikan cara untuk melakukan \textit{remote deployment} pada lingkungan \textit{IoT} dengan menggunakan kubernetes. Setelah dilakukan analisis, implementasi, dan pengujian, dapat diambil kesimpulan sebagai berikut.
\begin{enumerate}
  \item Sistem \textit{remote deployment} pada lingkungan \textit{IoT} dengan \textit{Kubernetes} berhasil dibuat dengan arsitektur yang tersentralisasi untuk mengatur proses \textit{remote deployment} pada seluruh \textit{company}. Sistem sudah dibuat dengan baik dan berhasil untuk memenuhi kebutuhan fungsional yang telah didefinisikan sebelumnya, namun memiliki \textit{single point of failure} karena arsitektur yang terpusat.
  \item Dengan sistem \textit{remote deployment}, perusahaan dapat dengan mudah untuk melakaukan manajemen perangkat \textit{IoT} dan melakukan \textit{deployment} dengan mudah.
  \item Kubernetes sudah cocok sebagai \textit{resource manager} yang dipakai untuk proses \textit{remote deployment} pada sistem ini. Dengan seluruh fungsionalitasnya, kubernetes dapat memudahkan dalam proses manajemen perangkat, group, serta \textit{deployment plan}.
\end{enumerate}

\section{Saran}
Adapun banyak kekurangan dan kelemahan yang ditemukan dalam penelitian tugas akhir ini. Berikut adalah beberapa saran yang dapat dilakukan untuk pengembangan atau penelitian selanjutnya.
\begin{enumerate}
  \item Dapat diimplementasikan versioning pada deployment ketika melakukan \textit{rollback}. Ketika versi yang dicapai sudah tidak lagi ditemukan barulah \textit{deployment} di hapus dari \textit{cluster}.
  \item Dapat diimplementasikan dashboard untuk \textit{admin} agar memudahkan proses manajemen \textit{company} dan \textit{user}.
  \item Dapat membuat proses \textit{remote deployment} lebih terkustomisasi dengan cara membuat \textit{deployment plan} lebih \textit{flexible} lagi.
  \item Menambahkan fitur seperti \textit{remote command} untuk melakukan eksekusi \textit{command} di setiap perangkat tanpa harus melakukannya satu persatu
  \item Proses registrasi \textit{node} pada \textit{cluster} masih dilakukan secara manual, dapat digunakan sistem \textit{device discovery} untuk mengeleminasi proses yang redundan.
  \item Sistem dapat dibuat sebagai \textit{microservice} untuk menghindari \textit{single point of failure}
  \item Pembuatan \textit{swagger} dalam membuat \textit{service} akan membantu dokumentasi serta memudahkan proses \textit{testing}.
\end{enumerate}