\chapter{Penutup}

Bab Kesimpulan dan Saran akan menjadi bagian akhir dan penutup dari penelitian tugas akhir ini. Bab ini akan membahas kesimpulan yang berisi ketercapaian tujuan penelitian tugas akhir dengan permasalahan yang diselesaikan dalam penelitian tugas akhir. Selain itu, bab ini akan membahas saran yang dapat dilakukan untuk pengembangan atau penelitian selanjutnya.

\section{Kesimpulan}
Penelitian tugas akhir ini mengimplementasikan cara untuk melakukan \textit{remote deployment} pada lingkungan IoT dengan menggunakan kubernetes. Setelah dilakukan analisis, implementasi, dan pengujian, dapat diambil kesimpulan sebagai berikut.
\begin{enumerate}
  \item  Penelitian ini berhasil merancang dan mengimplementasikan arsitektur sistem \textit{remote deployment} yang bersifat \textit{platform agnostic} menggunakan Kubernetes. Sistem ini memungkinkan untuk melakukan proses \textit{deployment} dengan metode tanpa kabel (non-serial) terbukti dari implementasi pada Raspberry Pi dengan dua node serta pada lingkungan \textit{virtual machine} dengan konfigurasi node yang sama. Pengujian fungsional dan \textit{end-to-end} mengkonfirmasi bahwa sistem berjalan dengan baik dan memenuhi semua kebutuhan fungsional yang telah didefinisikan.
  \item Sistem \textit{remote deployment} berbasis Kubernetes dapat beroperasi dengan baik pada perangkat dengan keterbatasan sumber daya hardware, seperti Raspberry Pi. Implementasi dengan menggunakan K3s, distribusi Kubernetes yang ringan, memungkinkan deployment pada perangkat low-resource, membuktikan bahwa solusi ini dapat diadaptasi untuk lingkungan IoT dengan sumber daya terbatas.
\end{enumerate}

\section{Saran}
Adapun banyak kekurangan dan kelemahan yang ditemukan dalam penelitian tugas akhir ini. Berikut adalah beberapa saran yang dapat dilakukan untuk pengembangan atau penelitian selanjutnya.
\begin{enumerate}
  \item Dapat diimplementasikan versioning pada deployment ketika melakukan \textit{rollback}. Ketika versi yang dicapai sudah tidak lagi ditemukan barulah \textit{deployment} di hapus dari \textit{cluster}.
  \item Dapat diimplementasikan	extit{dashboard}untuk \textit{admin} agar memudahkan proses manajemen \textit{company} dan \textit{user}.
  \item Dapat membuat proses \textit{remote deployment} lebih terkustomisasi dengan cara membuat \textit{deployment plan} lebih \textit{flexible} lagi.
  \item Menambahkan fitur seperti \textit{remote command} untuk melakukan eksekusi \textit{command} di setiap perangkat tanpa harus melakukannya satu persatu
  \item Proses registrasi \textit{node} pada \textit{cluster} masih dilakukan secara manual, dapat digunakan sistem \textit{device discovery} untuk mengeleminasi proses yang redundan.
  \item Sistem dapat dibuat sebagai \textit{microservice} untuk menghindari \textit{single point of failure}
  \item Pembuatan \textit{swagger} dalam membuat \textit{service} akan membantu dokumentasi serta memudahkan proses \textit{testing}.
\end{enumerate}