\chapter{Penutup}

Bab Kesimpulan dan Saran akan menjadi bagian akhir dan penutup dari penelitian tugas akhir ini. Bab ini akan membahas kesimpulan yang berisi ketercapaian tujuan penelitian tugas akhir dengan permasalahan yang diselesaikan dalam penelitian tugas akhir. Selain itu, bab ini akan membahas saran yang dapat dilakukan untuk pengembangan atau penelitian selanjutnya.

\section{Kesimpulan}
Penelitian tugas akhir ini mengimplementasikan cara untuk melakukan \textit{remote deployment} pada lingkungan IoT dengan menggunakan kubernetes. Setelah dilakukan analisis, implementasi, dan pengujian, dapat diambil kesimpulan bahwa penelitian ini berhasil membuat PERISAI yang bersifat \textit{platform agnostic} serta dapat berjalan pada perangkat yang memiliki daya komputasi yang terbatas dengan menggunakan Kubernetes. PERISAI dapat dijalankan pada Raspberry Pi 2 Model B v1.1 yang memiliki RAM sebesar 900Mb dengan menggunakan K3s sebagai distribusi kubernetes. PERISAI memungkinkan untuk melakukan proses \textit{targeted deployment} dengan menggunakan label yang terdapat pada kubernetes. Hal ini dapat dibuktikan dari pengujian pada Raspberry Pi dengan dua node serta pada lingkungan \textit{virtual machine} dengan konfigurasi node yang sama. Pengujian fungsional dan \textit{end-to-end} mengkonfirmasi bahwa sistem berjalan dengan baik dan memenuhi semua kebutuhan fungsional yang telah didefinisikan.

\section{Saran}
Adapun banyak kekurangan dan kelemahan yang ditemukan dalam penelitian tugas akhir ini. Berikut adalah beberapa saran yang dapat dilakukan untuk pengembangan atau penelitian selanjutnya.
\begin{enumerate}
  \item Dapat diimplementasikan versioning ketika melakukan \textit{deployment}. Ketika versi yang dicapai sudah tidak lagi ditemukan, \textit{deployment} akan dihapus dari \textit{cluster}.
  \item Dapat diimplementasikan	\textit{dashboard} untuk \textit{admin} agar memudahkan proses manajemen \textit{company} dan \textit{user}.
  \item Dapat membuat proses \textit{remote deployment} lebih terkustomisasi dengan cara membuat \textit{deployment plan} lebih \textit{flexible} lagi.
  \item Menambahkan fitur seperti \textit{remote command} untuk melakukan eksekusi \textit{command} di setiap perangkat tanpa harus melakukannya satu persatu
  \item Proses registrasi \textit{node} pada \textit{cluster} masih dilakukan secara manual, dapat digunakan sistem \textit{device discovery} untuk mengeleminasi proses yang redundan.
  \item Pembuatan \textit{swagger} dalam membuat \textit{service} akan membantu dokumentasi serta memudahkan proses \textit{testing}.
\end{enumerate}