%--------------------------------------------------------------------%
%
% Berkas utama templat LaTeX.
%
% author Petra Barus, Peb Ruswono Aryan, Faris Rizki Ekananda
%
%--------------------------------------------------------------------%
%
% Berkas ini berisi struktur utama dokumen LaTeX yang akan dibuat.
%
%--------------------------------------------------------------------%

\documentclass[bahasa, 12pt, a4paper, onecolumn, oneside, final]{report}

\input{config/if-itb-thesis.sty}

\makeatletter

\makeatother

\addbibresource{references.bib}

\begin{document}

%Basic configuration
\title{\textit{Automatic Scaling} dengan Kontrol Fleksibel Berdasarkan Model Prediktif Berbasis \textit{Time Series} untuk Efisiensi Sumber Daya Kubernetes pada Pods \textit{Elastic Search}}
\date{}
\author{
    Steven Nataniel Kodyat \\
    NIM: 13519002
}
\newcommand\tanggalpengesahan{31 Juli 2023}

\pagenumbering{roman}
\setcounter{page}{1}

\input{chapters/cover}
\clearpage
\pagestyle{empty}

\begin{center}
    \smallskip
    
    \Large \bfseries \MakeUppercase{\thetitle}
    \vfill
    
    \Large Laporan Tugas Akhir
    \vfill
    
    \large Oleh
    
    \Large \theauthor
    
    \large Program Studi Teknik Informatika \\
    
    \normalsize \normalfont
    Sekolah Teknik Elektro dan Informatika \\
    Institut Teknologi Bandung \\
    
    \vfill
    \normalsize \normalfont
    Telah disetujui dan disahkan sebagai Laporan Tugas Akhir \\
    di Bandung, pada tanggal \tanggalpengesahan
    
    \vspace{0.5cm}
    Pembimbing,
    
    \vfill
    \underline{Achmad Imam Kistijantoro, S.T., M.Sc., Ph.D.} \\
    NIP. 19730809 200604 1 001
    
\end{center}
\clearpage

\input{chapters/statement}

\pagestyle{plain}

\clearpage
\chapter*{ABSTRAK}
\addcontentsline{toc}{chapter}{Abstrak}
\begin{center}
    \center
    \begin{singlespace}
      \large\bfseries\MakeUppercase{\thetitle}
  
      \normalfont\normalsize
      Oleh:
  
      \bfseries \theauthor
    \end{singlespace}
\end{center} 

\begin{singlespace}
    \small
    Contoh teks abstrak
    % Dengan berkembangnya dunia digital, aplikasi sehari-hari sangat mempengaruhi kehidupan manusia, dari \textit{search engine}, media sosial hingga \textit{e-commerce}. Aplikasi-aplikasi ini telah mengubah cara kita berinteraksi dan mengakses informasi. Tentu dalam pembuatan aplikasi-aplikasi tersebut akan sangat dependen ke sistem temu balik informasi. Salah satu sistem temu balik informasi yang paling populer saat ini adalah \textit{Elastic Search}. Dalam aplikasinya, \textit{Elastic Search} digunakan secara luas untuk menyediakan kemampuan pencarian yang canggih seperti menemukan produk dengan cepat, hingga platform media sosial yang memberikan hasil pencarian yang relevan dan personalisasi. Namun, kekurangan \textit{Elastic Search} adalah, secara bawaan, sistem ini akan mengambil memori yang tersedia untuk melakukan pemrosesan, apabila dialokasi terlalu sedikit, prosesor akan menjadi kewalahan karena harus melakukan operasi pencarian tanpa bantuan memori. Di sisi lain, perampatan sumber daya tidak selalu menimbulkan kinerja yang buruk untuk \textit{Elastic Search} karena ketergantungannya terhadap konteks data yang disimpan dan kebutuhan pengguna.
    
    % Oleh karena itu, diperlukan teknik \textit{autoscaling} yang fleksibel agar \textit{Elastic Search} dapat berjalan secara optimal dan sesuai dengan toleransi \textit{tradeoff} antara biaya dan kinerja. \textit{Autoscaler} ini akan dibangun diatas \textit{Kubernetes} sebagai \textit{container orchestration} dan \textit{ARIMA} sebagai model prediksi. Sistem ini akan mengambil \textit{metrics} dari \textit{Elastic Search} dan melakukan prediksi \textit{throughput} dan utilisasi prosesor serta memori. Hasil prediksi tersebut akan digunakan untuk memenuhi persyaratan yang ditentukan oleh pengguna untuk melakukan \textit{scaling}. Persyaratan dapat disusun oleh pengguna sebagai kumpulan kondisi yang akan dipakai oleh sistem sebagai acuan untuk melakukan keputusan \textit{scaling}. Pengujian sudah dilakukan untuk setiap komponen dan satu sistem penuh untuk memastikan spesifikasi dan fungsional berjalan sesuai kebutuhan. Perbandingan dengan \textit{Vertical} dan \textit{Horizontal Autoscaler}-pun sudah dilakukan, secara garis besar, metode ini dapat menggantikan opsi \textit{Vertical} dan \textit{Horizontal Autoscaler} pada konteks \textit{pods Elastic Search}.

    % Pada akhirnya, sistem \textit{autoscaler} dengan model prediksi dapat lebih baik dalam melakukan \textit{scaling} dibandingkan dengan \textit{autoscaler} sederhana yang memakai \textit{threshold}. Dan model prediksi yang bisa digunakan adalah model prediksi berbasis \textit{time series} seperti ARIMA.
    % \textbf{\textit{Kata kunci: Autoscaler, Kubernetes, Flexible Control, ARIMA, Elastic Search, Predictive Autoscaler}}
\end{singlespace}
\clearpage
% \clearpage
\chapter*{ABSTRACT}
\addcontentsline{toc}{chapter}{Abstract}

\begin{center}
    \center
    \begin{singlespace}
      \large\bfseries\MakeUppercase{Automatic Scaling with Flexible Control Based on Time Series Predictive Model for Kubernetes Resource Efficiency on Elastic Search Pods}
  
      \normalfont\normalsize
      By:
  
      \bfseries \theauthor
    \end{singlespace}
\end{center} 


\begin{singlespace}
    \small
    With the advancement of the digital world, applications greatly influence human life, from search engines and social media to e-commerce. These applications have transformed the way we interact and access information. Naturally, the development of such applications heavily relies on information retrieval systems. One of the most popular information retrieval systems nowadays is Elastic Search. In its application, Elastic Search is widely used to provide sophisticated search capabilities, enabling quick product discovery in e-commerce and relevant, personalized search results in social media platforms. However, a drawback of Elastic Search is that, by default, it will consume all available memory for processing. If allocated too little memory, the processor will struggle to perform search operations without memory assistance. On the other hand, resource overprovisioning doesn't always lead to poor performance for Elastic Search, as it depends on the context of stored data and user needs.

    Therefore, a flexible autoscaling technique is needed to ensure Elastic Search runs optimally and aligns with the tradeoff tolerance between cost and performance. This autoscaler will be built on Kubernetes for container orchestration and use ARIMA as the prediction model. The system will retrieve metrics from Elastic Search and predict throughput and processor and memory utilization. These predictions will be used to meet user-defined requirements for scaling. Users can define conditions that the system will use as references to make scaling decisions. Testing has been conducted for each component and a full system to ensure that specifications and functionality meet the requirements. Comparisons with Vertical and Horizontal Autoscalers have been made, and in essence, this method can replace the options of Vertical and Horizontal Autoscalers in the context of Elastic Search pods.

    Ultimately, the autoscaler with a prediction model performs better in scaling compared to a simple autoscaler that uses thresholds. A suitable prediction model is a time series-based prediction model like ARIMA.
    \textbf{\textit{Keywords: Autoscaler, Kubernetes, Flexible Control, ARIMA, Elastic Search, Predictive Autoscaler}}
\end{singlespace}
\clearpage

\clearpage
\chapter*{Kata Pengantar}
\addcontentsline{toc}{chapter}{Kata Pengantar}

Puji dan syukur penulis panjatkan kepada Tuhan Yang Maha Esa atas berkat dan rahmatnya, laporan tugas akhir yang berjudul \thetitle{} dapat diselesaikan dalam rangka memenuhi syarat kelulusan tingkat sarjana. Perlu diakui pengerjaan tugas akhir ini didukung oleh banyak pihak. Khususnya, penulis ingin mengucapkan terima kasih kepada:

\begin{enumerate}
  \item Bapak Dr.techn. Muhammad Zuhri Catur Candra, S.T., M.T., selaku dosen pembimbing atas segala bentuk dukungan yang telah diberikan dan kesabarannya dalam membimbing penulis serta memberikan saran dalam pengerjaan tugas akhir.
  \item Dicky Prima Satya, S.T, M.T., Bapak Adi Mulyanto, S.T, M.T., Robithoh Annur, S.T., M.Eng., Ph.D., dan Tricya Esterina Widagdo, ST., M.Sc. selaku dosen koordinator tim tugas akhir atas usahanya mengingatkan mahasiswa program studi Teknik Informatika untuk mengerjakan tugas akhirnya.
  \item Kedua orangtua penulis atas dukungan yang diberikan sehingga penulis dapat menempuh pendidikan di Institut Teknologi Bandung.
  \item Seluruh dosen program studi Teknik Informatika ITB yang telah memberikan ilmu pengetahuan yang sangat berharga bagi penulis.
  \item Teman-teman di program studi Teknik Informatika atas kontribusi berupa ide, waktu, dan masukan yang diberikan selama penulis mengerjakan tugas akhir.
  \item Staf tata usaha program studi Teknik Informatika yang membantu penulis dalam menyelesaikan urusan administrasi tugas akhir.
  \item Salman ITB serta tim Beasiswa Perintis yang telah membantu penulis sehingga penulis dapat menempuh pendidikan di Institut Teknologi Bandung dengan mudah.
\end{enumerate}

Akhir kata, penulis mengucapkan terima kasih kepada semua pihak yang telah terlibat dalam pengerjaan tugas akhir ini. Penulis juga ingin menyampaikan mohon maaf apabila terdapat kesalahan maupun kekurangan dalam laporan tugas akhir ini. Penulis berharap semoga tugas akhir ini dapat bermanfaat bagi pembaca dan riset-riset kedepannya.

\begin{flushright}
  \vspace{0.5cm}
  Bandung, \tanggalpengesahan
  
  
  \vspace{1.5cm}
  
  Muhammad Garebaldhie ER Rahman
\end{flushright}

\titleformat*{\section}{\centering\bfseries\Large\MakeUpperCase}
\titlespacing*{\chapter}{0pt}{0pt}{4pt}

% Setting judul toc, lot, lof, bib
\renewcommand{\contentsname}{DAFTAR ISI}
\renewcommand{\listfigurename}{DAFTAR GAMBAR}
\renewcommand{\listtablename}{DAFTAR TABEL}
\renewcommand{\bibname}{DAFTAR PUSTAKA}

\tableofcontents
\listofappendices
\listoffigures
\listoftables

\newpage

\titleformat*{\section}{\bfseries\large}
\pagenumbering{arabic}

%----------------------------------------------------------------%
% Konfigurasi Bab
%----------------------------------------------------------------%
\setcounter{page}{1}
\renewcommand{\chaptername}{BAB}
\renewcommand{\thechapter}{\Roman{chapter}}
%----------------------------------------------------------------%

%----------------------------------------------------------------%
% Dafter Bab
% Untuk menambahkan daftar bab, buat berkas bab misalnya `chapter-6` di direktori `chapters`, dan masukkan ke sini.
%----------------------------------------------------------------%
\chapter{Pendahuluan}

Konten pada bab ini berisi terkait gambaran umum dan permasalahan yang akan diselesaikan dalam tugas akhir ini. Bab ini akan dimulai dari penjelasan latar belakang dari masalah yang diselesaikan, rumusan masalah, tujuan, batasan masalah, metodologi yang digunakan, dan berakhir pada jadwal pelaksaaan tugas akhir ini.

\section{Latar Belakang}
\label{sec:latar-belakang}

Di era digital saat ini, \textit{Internet of Things} \textit{(IoT)} sudah menjadi bagian tak terpisahkan dari kehidupan manusia. Berbagai sistem dan aplikasi, mulai dari rumah cerdas hingga sistem parkir, telah mengintegrasikan IoT untuk memberikan kemudahan dan efisiensi. Seiring dengan berkembangnya pengguna sistem, jumlah perangkat IoT yang dibutuhkan pun akan semakin banyak. Hal ini dapat menimbulkan masalah kompleksitas baru dalam manajemen dan operasionalnya \parencite{IOTSmartCity}.

Seiring bertambahnya jumlah perangkat \textit{IoT}, sistem harus mampu beradaptasi tanpa mengurangi kualitas sistem. Kemudahan dalam proses adaptasi ini memungkinkan penambahan perangkat \textit{IoT} baru dengan cepat dan mudah, sehingga sistem menjadi lebih cepat, efektif, dan efisien. Namun, menurut \parencite{RemoteDeployment}, masih banyak sistem dan perangkat IoT yang belum mampu melakukan \textit{update} secara \textit{remote}. Hal ini mengakibatkan peningkatan proses operasional pada setiap perangkat \textit{IoT}

Di sinilah peran \textit{remote deployment} menjadi sangat penting. \textit{Remote deployment} menawarkan solusi untuk mengelola aplikasi secara terpusat melalui internet. Hal ini mendukung proses standardisasi komunikasi yang lebih efisien dan terkontrol antar perangkat, sehingga mengurangi proses miskonfigurasi serta waktu operasional untuk melakukan konfigurasi secara manual. Dengan ini, \textit{remote deployment} dapat membantu dalam mengatasi masalah skalabilitas.

Penelitian ini bertujuan untuk mengembangkan sistem \textit{remote deployment} dengan memanfaatkan \textit{Kubernetes}. \textit{Kubernetes} akan berperan dalam mengorkestrasi proses deployment perangkat \textit{IoT} secara terpusat dan efisien. Sistem ini diharapkan dapat meningkatkan efektivitas dan skalabilitas pengelolaan perangkat \textit{IoT} dalam jumlah besar. Dengan implementasi \textit{remote deployment}, kompleksitas dan biaya operasional dapat diminimumkan sehingga proses \textit{maintenance} perangkat \textit{IoT} untuk skala besar menjadi mudah untuk dilakukan.

\section{Rumusan Masalah}

Berdasarkan latar belakang yang ada, rumusan tugas akhir ini adalah sebagai berikut.
\begin{enumerate}
  \item Bagimana cara meningkatkan performa layanan IoT khususnya pada \textit{Smart Home System} dengan menggunakan \textit{service mesh}?
  \item Bagaimana cara melakukan \textit{device discovery} yang optimal khususnya pada \textit{Smart Home System}?
  \item Bagaimana cara melakukan \textit{remote deployment} untuk setiap perangkat yang ada pada \textit{Smart Home System}?
\end{enumerate}


tujuan
1. optimizing edge gateway biar lebih cepet (raspi)
2. edge gateway gimana caranya biar bisa bikin software (orchestrating segala software disana, mainly instalasi remote sm healtcheck (agora))

- gateway ini buat nerusin data sensor (udah ada port masing2) pengen bisa di kelola pake machine learning pipeline dan ini tuh bakal di treat sebagai service berlayer2, outputnya di piepline ke service lain dan ini remote deployment semuanya.

- cannonical (edge gateway)

1. apakah ini novel (penting ga, studi literatur)
2. mahasiswa capable apa engga (progress)
3. mahasiswa udah progress apa blm

- edge computing platform

- Gaada, atau udah ada tapi sistemnya gabisa (benchmarking)
- implementasi. innetwrok computing, bedain OS dan arsitektur komputerntya. Bandingin reliabilitynya (cari metrics buat compare)

\section{Tujuan}

Tujuan yang akan dicapai untuk tugas akhir ini adalah sebagai berikut.

\begin{enumerate}
  \item Membuat sistem yang digunakan untuk melakukan \textit{remote deployment} pada \textit{IoT} dengan memanfaatkan \textit{Kubernetes}.
\end{enumerate}

\section{Batasan Masalah}
\label{sec:batasan-masalah}

Terdapat batasan yang diambil dalam pelaksanaan tugas akhir ini, yaitu sebagai berikut.

\begin{enumerate}
  \item Penelitian ini hanya berfokus untuk menyelesaikan masalah pada bagian \textit{resource constraint} serta \textit{platform agnostic}, sehingga permasalahan \textit{security} tidak dibahas secara mendalam.
  \item Analisis akan terbatas pada layanan dan lingkungan IoT khususnya perangkat \textit{RaspberryPi}.
  \item Perangkat IoT memiliki akses internet dan \textit{terminal}.
  \item Diasumsikan setiap perangkat IoT sudah terhubung kedalam \textit{cluster}, proses registrasi \textit{cluster} secara manual dapat diabaikan.
\end{enumerate}



\section{Metodologi}

Terdapat beberapa metodologi yang digunakan untuk melaksanakan tugas akhir ini, berikut adalah tahapan pelaksanaannya:
\begin{enumerate}
  \item \textbf{Identifikasi Permasalahan}

        Pada tahap ini, dilakukan eksplorasi untuk mencari arsitektur PERISAI agar sistem \textit{remote deployment} dapat diimplementasi. Setelah itu dilakukan identifikasi permasalahan pada sistem tersebut agar sistem yang dibuat dapat memenuhi semua kebutuhan.

  \item \textbf{Analisis dan Perancangan Solusi}

        Setelah mengidentifikasi permasalahan, dilakukan analisis perancangan solusi yang bertujuan untuk mencari metode dan pendekatan yang dapat dikembangkan untuk membangun PERISAI. Analisis ini dimulai dari eksplorasi metode melalui studi literatur lalu dilanjutkan dengan penelitian yang pernah dilakukan.

  \item \textbf{Implementasi}

        Setelah merancang solusi, gagasan tersebut akan dikembangkan dan diimplementasikan. Hasil dari tahap ini berupa desain dan implementasi arsitektur yang memenuhi seluruh kebutuhan untuk sistem \textit{remote deployment} yang berjalan pada lingkungan IoT.

  \item \textbf{Pengujian dan Evaluasi}

        Setelah implementasi berhasil dilakukan, dilakukan serangkaian pengujian untuk memastikan kebenaran implementasi. Setelah pengujian dilakukan, hasil implementasi akan dievaluasi agar mendapatkan \textit{feedback} terkait hal-hal yang dapat ditingkatkan kedepannya.

\end{enumerate}

\section{Jadwal Pelaksanaan Tugas Akhir}

Konten dari Tugas Akhir ini akan dibagi menjadi lima bab sebagai berikut.
\begin{enumerate}
    \item \textbf{Pendahuluan}
    
    Pada Bab I akan dijelaskan gagasan utama dari tugas akhir ini yang berisi dari latar belakang, rumusan masalah, tujuan, batasan, metodologi hingga sistematika pembahasan.

    \item \textbf{Studi Literatur}
    
    Selanjutnya, Bab II akan menjelaskan hasil studi literatur yang berkaitan dengan pengerjaan tugas akhir ini. Bab II ini berisi tentang pemahaman dasar seputar topik yang akan dibahas pada tugas akhir ini.

    \item \textbf{Analisis Persoalan dan Rancangan Solusi}
    
    Pada Bab III akan dijelaskan analisis persoalan untuk menyusun rancangan solusi. Rancangan tersebut akan dijelaskan pada bab ini sebelum diimplementasikan. Rancangan solusi akan dipaparkan dalam bentuk diagram dan kajian dalam bab ini.

    \item \textbf{Implementasi dan Pengujian}
    
    Bab IV ini akan berisikan kajian terhadap implementasi yang telah dibuat. Bab ini juga akan membahas tahap-tahap pengujian dan hasilnya. Perbandingan beberapa model prediksi akan dibahas pada bab ini.

    \item \textbf{Kesimpulan dan Saran}
    
    Bab V akan menutup tugas akhir ini. Konten pada bab ini akan menjawab rumusan masalah. Bab ini juga akan menyebutkan saran-saran perbaikan yang bisa dipakai untuk penelitian berikutnya. Bab ini akan menyimpulkan hasil implementasi dan rancangan solusi terhadap masalah yang sudah diidentifikasi.
\end{enumerate}
\chapter{Studi Literatur}
\label{chapter:studi-literatur}

Bab ini akan diisi oleh studi literatur yang berkaitan dengan topik persoalan tugas akhir untuk memberikan informasi mengenai dasar teori dan studi yang dipakai. Bab ini diharapkan dapat membantu pembaca untuk lebih mengerti tentang penelitian tugas akhir ini.

\section{\textit{Software Deployment}}
Secara informal, \textit{software deployment} mengacu pada semua aktivitas yang membuat \textit{software} tersedia untuk digunakan \parencite{softwareDeploymentCarzaniga1998characterization}. Secara formal, \textit{Software Deployment} dapat didefinisikan sebagai sebuah proses dan jadwal dari suatu set aktivitas pasca produksi yang dilakukan oleh ataupun untuk pengguna \textit{software} tersebut untuk membuat \textit{software} dapat digunakan serta \textit{up to date} \parencite{ARCANGELI2015198}. Selain itu, \textit{software deployment} juga dapat dianggap sebagai proses yang terdiri dari sejumlah aktivitas yang saling terkait, termasuk rilis \textit{software} pada akhir siklus pengembangan, konfigurasi, instalasi ke dalam lingkungan eksekusi, dan pengeksekusian \textit{software} \parencite{softwareDeploymentFuturePast}.

\subsection{Aktivitas}

Aktivitas pada \textit{software deployment} terdiri dari
\textit{release}, \textit{installation}, \textit{activation}, \textit{deactivation}, \textit{retire}, \textit{update}, \textit{reorganization}, serta \textit{redistribution}. Berikut merupakan penjelasan tentang tahapan tersebut serta visualisasinya yang dapat dilihat pada Gambar II.1.

\begin{enumerate}
  \item \textit{Release} adalah tahapan untuk mempersiapkan \textit{software} untuk didistribusikan
  \item \textit{Installation} adalah tahapan awal untuk melakukan \textit{integrasi} \textit{software} pada perangkat pengguna.
  \item \textit{Activation} adalah proses yang dilakukan ketika menyalakan \textit{software}.
  \item \textit{Deactivation} adalah proses untuk menghentikan \textit{software}.
  \item \textit{Deinstallation} adalah proses untuk menghilangkan seluruh bagian dari \textit{software} pada perangkat pengguna.
  \item \textit{Retire} adalah proses untuk menandakan bahwa \textit{software} sudah bersifat \textit{obsolete} dan tidak akan dilakukan \textit{maintanance} lagi.
  \item \textit{Update} adalah proses membuat versi terbaru dari \textit{software}.
\end{enumerate}

\begin{figure}[ht]
  \centering
  \includegraphics[width=0.8\textwidth]{resources/chapter-2/deployment-phase.jpg}
  \caption{\textit{Deployment Phase \parencite{ARCANGELI2015198}}}
  \label{fig:deployment-phase}
\end{figure}

\subsection{Kategori}

Awal mulanya \textit{deployment} dilakukan dengan membungkus \textit{software} menjadi sebuah \textit{installer}. \textit{Installer} dapat diunduh dan dijalankan oleh \textit{pengguna} untuk membuat \textit{software} tersedia pada perangkat yang digunakan. Jika terdapat \textit{update software}, pengguna harus mengunduh \textit{installer} dengan versi terbaru untuk digunakan \parencite{softwareDeploymentCarzaniga1998characterization}. 

Tentunya hal ini kurang efektif karena pengguna harus melakukan pengecekan secara berkala untuk mengetahui apakah terdapat versi terbaru atau tidak. Oleh karena itu, muncul teknologi \textit{deployment} yaitu \textit{package manager}. \textit{Package Manager} dapat digunakan untuk proses \textit{installing, updating, and generally managing software} \parencite{softwareDeploymentCarzaniga1998characterization}. Namun, muncul beberapa masalah seperti \textit{dependency hell}, \textit{dependency conflict} serta \textit{platfrom-constrained} karena \textit{package manager} tidak tersedia di semua \textit{operating system}. 

Pada tahun 2014, muncul teknologi baru yaitu \textit{docker, a cloud-centric platform-as-a-service} yang bersifat \textit{}, bertujuan untuk menyelesaikan masalah \textit{dependency conflict dan dependency hell} \parencite{merkel2014docker}. Docker menerapkan \textit{deployment} berbasis \textit{container} yang menggunakan namespace pada linux kernel dan \textit{cgroups} dalam melakukan manajemen \textit{resource} nya sehingga dapat dibuat sebuah sistem yang berjalan khusus untuk sebuah \textit{software} dengan minimal tanpa perlu adanya \textit{conflict} pada \textit{dependency}. 

Dengan adanya docker, metode \textit{deployment} mulai beralih ke arah \textit{cloud deployment}. Pada penelitian yang dilakukan oleh \parencite{wurster2020essential}, \textit{cloud deployment} dapat dikategorikan menjadi tiga bagian yaitu \textit{General-Purpose (GP)}, \textit{Provider-Specific (ProvS)}, serta \textit{Platform-Specific (PlatS)}.

\begin{enumerate}
  \item \textit{General-Purpose (GP)}

        Teknologi ini mendukung semua fitur dan mekanisme \textit{deployment} mulai dari \textit{single-cloud}, \textit{hybrid}, dan \textit{multi-cloud} serta berbagai jenis \textit{layanan cloud (XaaS)}. Beberapa teknologi yang mencakup kategori ini: Puppet, Chef, Ansible, OpenStack Heat, Terraform, SaltStack, Juju, dan Cloudify.

  \item \textit{Provider-Specific (ProvS)}

        Kategori ini menyediakan fitur untuk membuat \textit{reusable entity}. ProvS hanya mendukung \textit{deployment single-cloud} karena ditawarkan oleh penyedia \textit{cloud} tertentu, sehingga hanya mendukung layanan cloud yang ditawarkan oleh penyedia tersebut. Beberapa teknologi yang mencakup kategori ini: AWS CloudFormation dan Azure Resource Manager.

  \item \textit{Platform-Specific (PlatS)}

        Kategori ini mendukung \textit{multi-cloud} dan \textit{resuable deployment}. Kategori ini dibatasi dalam hal model pengiriman dan penggunaan bundel pada platform tertentu untuk membuat \textit{deployment}. <isalnya, Kubernetes hanya \textit{deployment} dengan \textit{container}. Beberapa teknologi yang mencakup kategori ini: Kubernetes, CFEngine, dan Docker Compose.

\end{enumerate}





% \section{\textit{Deployment Plan}}

\textit{Deployment plan} adalah pemetaan antara sistem, \textit{software}, dan deployment, yang dapat dikaitkan dengan data untuk konfigurasi serta \textit{dependency} dari masing masing bagian dan alasan untuk melakukan \textit{deployment} \parencite{ARCANGELI2015198}.

\section{Kubernetes}

Kubernetes adalah sebuah solusi \textit{open source} yang berguna untuk melakukan orkestrasi berbagai aplikasi yang telah di bungkus dalam suatu lingkungan yang disebut sebagai \textit{container}. Kubernetes berfungsi untuk melakukan \textit{deployment} otomatis, \textit{auto scaling} secara otomatis, serta membuat \textit{network} untuk menghubungkan \textit{container} dengan \textit{container} lainnya. Kubernetes membantu mengelola dan mempercepat proses pengembangan layanan yang rumit dengan skala yang besar \parencite{helmkubernetes}.

Kubernetes memiliki beberapa komponen yang digunakan ketika mengelola layanan. \textit{Node, pod, service} dan \textit{deployment} merupakan empat komponen utama yang sering kali menjadi komponen utama ketika membuat suatu layanan dengan kubernetes. \textit{Node} dapat dianalogikan dengan lingkungan \textit{virutual machine} yang memiliki kemampuan terbatas. \textit{Pod} merupakan suatu tempat untuk menjalankan berbagai macam \textit{container} didalamnya. Satu pod dapat memiliki lebih dari satu \textit{container} untuk dioperasikan. \textit{Service} berfungsi untuk membuka akses eksternal ke dalam \textit{pod} yang secara umum bersifat internal dan tidak dapat diakses dari luar. Terakhir yaitu \textit{deployment} adalah suatu konfigurasi untuk menjalankan layanan yang akan dibuat, konfigurasi deployment juga melingkupi komponen \textit{pod} dan \textit{service}.

Kubernetes memiliki fitur seperti \textit{auto scaling, self-healing, device discovery, load balancing} serta \textit{rollout} dan \textit{rollback}. \textit{Auto scaling} sering digunakan untuk menjaga sistem untuk terus beroperasi dengan cara melakukan replikasi layanan dengan jumlah yang ditentukan. \textit{Self healing} memastikan bahwa layanan yang sedang mengalami kegagalan dapat diperbaiki secara otomatis. \textit{Rollout} dan \textit{Rollback} sering digunakan ketika melakukan proses \textit{deployment} untuk mencegah adanya \textit{downtime} ketika meluncurkan layanan versi baru. Seluruh fitur yang disebutkan bekerja sama untuk membuat kubernetes dapat mengelola aplikasi dalam skala yang masif dan mempertahankan kualitas layanan yang dibuat.

\subsection{\textit{Pod component}}

\textit{Pod} merupakan abstraksi unit atau komponen terkecil pada Kubernetes untuk memudahkan proses pengembangan. \textit{Pod} merupakan sebuah lingkunan linux yang digunakan secara bersama namun memiliki sumber daya yang terpisah dan terbatas melalui teknologi linux cgroups dan namespace untuk menjalankan satu atau lebih \textit{container}. \textit{Pod} bersifat \textit{ephemeral} sehingga seluruh sumber daya akan hilang apabila \textit{pod} mengalami kegagalan \parencite{pod}.

Untuk memastikan bahwa layanan dapat selalu berjalan dengan baik, semua kegiatan yang berkaitan dengan \textit{pod} mulai dari \textit{scaling} hingga \textit{health check} akan dilakukan oleh Kubernetes. Kubernetes akan bertanggung jawab untuk melakukan \textit{penjadwalan} serta pencocokan konfigurasi maupun siklus hidup dari \textit{pod} tersebut. Dengan abstraksi \textit{pod}, proses pengembangan layanan menggunakan Kubernetes semakin mudah untuk dipahami dan dilakukan.

\subsection{\textit{Service}}

Service merupakan suatu abstraksi untuk membuat pod dapat diakses secara external. \textit{Pod} bersifat dan ephemeral dan tidak dapat diakses dari luar pod. Dengan abstraksi service kubernetes dapat membuat \textit{pod} tersebut diakses secara eksternal dengan cara membuat suatu layanan intermediet untuk meneruskan \textit{request} dari eksternal.
Layanan intermediet yang disediakan oleh \textit{service} diantaranya \textit{ClusterIp, NodePort, LoadBalancer} dan  \textit{ExternalName}.

ClusterIP bersifat internal dan sangat berkaitan dengan erat dengan \textit{pod}. Apabila \textit{service} tidak di buat konfigurasinya \textit{pod} akan selalu memiliki \textit{service} dengan tipe \textit{ClusterIP}. \textit{NodePort} dan \textit{LoadBalancer} akan membuat \textit{pod} dapat ditemukan oleh layanan eksternal dan diakses melalui perangkat lain. Kedua tipe tersebut akan membuat sebuah endpoint untuk meneruskana \textit{request} yang masuk berdasarkan label yang diletakan ketika membuat \textit{pod} \parencite{service}
\subsection{\textit{Deployment}}

\textit{Deployment} merupakan abstraksi yang menggabungkan kedua abstraksi \textit{pod} dan \textit{service}. \textit{Deployment} dapat melihat kondisi seluruh sistem pada saat keadaan awal maupun keadaan berubah. \textit{Deployment} akan menyimpan keadaan sistem secara berkala pengecekan secara deklaratif untuk mendeteksi perubahan keadaan sistem saat ini dengan keadaan sistem yang diinginkan \parencite{deployment}.

\textit{Deployment} memiliki fitur \textit{rollout} dan \textit{rollback} untuk meningkatkan \textit{availability} layanan. \textit{rollback} berarti melakukan penurunan versi dari layanan yang saat ini sedang berjalan sedangkan \textit{rollout} untuk melakukan \textit{upgrade} layanan. Kedua fitur ini berjalan dengan cara membuat \textit{replica} dari layanan yang sedang berjalan untuk mencegah penurunan \textit{availability}. Ketika \textit{deployment} ingin menaikkan versi layanan yang digunakan, \textit{deployment} akan membuat \textit{pods} dengan versi terbaru. Ketika \textit{pods} ini sudah aktif beroprasi dan memiliki status \textit{healthy}, layanan dengan versi sebelumnya baru bisa dihapus \parencite{deployment}.

% Node merupakan suatu lingkungan terisolasi yang menjadi komponen dasar dalam pengelolaan aplikasi dengan kubernetes. 

\section{Kubernetes distribution}
Dalam penelitian ini akan dibahas tiga distribusi Kubernetes, yaitu KubeEdge, MicroK8s, serta K3s. Masing masing dari distribusi ini memiliki kegunaan dan fungsi nya masing masing. Walaupun semuanya memiliki \textit{support} untuk IoT namun setiap distribusi memiliki cara unik tersendiri untuk menyelesaikan masalahnya.

\subsection{KubeEdge}
KubeEdge merupakan solusi \textit{edge architecture} \textit{open source} yang mengembangkan Kubernetes secara lebih jauh untuk domain spesifik yaitu IoT \parencite{kubeedge}. Arsitektur KubeEdge memungkinkan untuk melakukan konfigurasi perangkat IoT yang berada di \textit{edge} secara terpusat melalui komponen \textit{cloud}. Dengan adanya dua komponen \textit{edge core} dan \textit{cloud core}, komunikasi antara platform aplikasi menjadi \textit{seamless}.

Untuk menghubungkan \textit{cloud core} dengan \textit{edge core}, KubeEdge menggunakan sebuah \textit{controller} yang dapat melakukan \textit{device discovery} terhadap \textit{edge core}. Setelah \textit{edge core} ditemukan, \textit{controller} akan meneruskan \textit{request} ke komponen berikutnya yaitu \textit{Sync Service}. KubeEdge menggunakan KubeBus untuk melakukan komunikasi antara \textit{cloud} dan \textit{edge}, KubeBus menggunakan protokol HTTP untuk meneruskan \textit{request} dari \textit{cloud core} menuju \textit{edge core}. Setelah \textit{request} diterima oleh \textit{edge core}, akan digunakan protokol MQTT untuk menerima ataupun mengirim data dari perangkat IoT ke \textit{edge core} dan juga sebaliknya. Ilustrasi arsitektur dapat dilihat pada gambar \ref*{fig:arsitektur-kube-edge}.

\begin{figure}[ht]
  \centering
  \includegraphics[width=0.5\textwidth]{resources/chapter-2/arsitektur-kube-edge.jpg}
  \caption{Arsitektur KubeEdge \parencite{kubeedge}}
  \label{fig:arsitektur-kube-edge}
\end{figure}

\subsection{Microk8s}
MicroK8s adalah sebuah platform \textit{open-source} yang digunakan untuk mengotomatisasi distribusi, \textit{scaling}, dan manajemen aplikasi yang berbasis kontainer. MicroK8s menyediakan fungsionalitas inti pada Kubernetes, dengan ukuran yang kecil, dan dapat di-\textit{scale} dari satu node hingga menjadi kluster produksi yang besar \parencite{microk8s}.

Dengan mengurangi penggunaan sumber daya yang dibutuhkan untuk menjalankan Kubernetes, MicroK8s memungkinkan penggunaan Kubernetes dalam berbagai lingkungan, seperti:

\begin{enumerate}
  \item Mengubah Kubernetes menjadi alat pengembangan yang ringan.
  \item Menjadikan Kubernetes tersedia untuk digunakan dalam lingkungan minimal seperti GitHub CI (Continuous Integration).
  \item Menyesuaikan Kubernetes untuk aplikasi IoT pada perangkat dengan \textit{resource} yang terbatas.
\end{enumerate}


\subsection{K3s}
K3s adalah sebuah platform \textit{open-source} yang memfasilitasi penggunaan Kubernetes dengan ukuran yang lebih ringan dan mudah diimplementasikan. K3s dikembangkan untuk memudahkan distribusi dan manajemen Kubernetes dalam lingkungan yang lebih sederhana dan bersifat minimalis. K3s dibuat untuk mendukung pengembangan pada ranah IoT karena K3s memiliki dukungan sepenuhnya untuk arsitektur ARM serta cocok untuk digunakan pada lingkungan \textit{edge} dan IoT \parencite{k3s}.

K3s memiliki dua komponen utama, yaitu \textit{server} dan \textit{agent}. \textit{Server} dapat dikatakan sebagai sebuah \textit{control plane} atau \textit{master node} yang digunakan pada K3s dan berfungsi untuk mengatur seluruh permintaan ataupun \textit{request} dari \textit{agent}. \textit{Server} memiliki tanggung jawab penuh terhadap masing masing \textit{agent} yang terhubung mulai dari menyimpan data dari masing masing \textit{agent}, \textit{controller}, serta \textit{scheduler} \parencite{k3s}. Di sisi lain, \textit{agent} berfungsi sebagai \textit{slave node} yang akan mengeksekusi semua perintah dari \textit{server} atau \textit{master node}. Ilustrasi arsitektur dapat dilihat pada gambar \ref{fig:arsitektur-k3s}.

\begin{figure}[ht]
  \centering
  \includegraphics[width=1\textwidth]{resources/chapter-2/arsitektur-k3s.jpg}
  \caption{Arsitektur K3s \parencite{k3s}}
  \label{fig:arsitektur-k3s}
\end{figure}

\pagebreak

\section{\textit{IoT}}



\textit{Internet of Things (IoT)} adalah sebuah infrastruktur dari objek, orang, sistem, serta sumber daya informasi yang saling terhubung dengan layanan cerdas untuk memproses informasi dari dunia fisik dan virtual \parencite{Dias2019}. Konsep dan paradigma \textit{Internet of Things} ini merupakan perwujudan dari evolusi teknologi informasi karena dapat membuat kehidupan menjadi lebih baik dalam berbagai sektor. Setiap objek yang ditemukan pada kegiatan sehari hari, bertransformasi menjadi entitas cerdas yang mampu berinteraksi dengan lingkungan sekitarnya dan jaringan digital secara lebih luas. \textit{IoT} memperkenalkan kemungkinan baru dalam otomatisasi dan pengambilan keputusan yang berbasis data, membuka jalan bagi inovasi lintas sektor \parencite{madakam2015internet}.

\begin{figure}[ht]
  \centering
  \includegraphics[width=0.8\textwidth]{resources/chapter-2/gambar-iot.jpg}
  \caption{Perangkat IoT yang Terdapat pada Lingkungan Sekitar \parencite{sotres2017practical}}
  \label{fig:iot-kehidupan-sehari-hari}
\end{figure}

\textit{IoT} memiliki aplikasi yang luas di berbagai sektor, termasuk industri, kesehatan, transportasi, dan pertanian. Dalam sektor industri, \textit{IoT} memungkinkan otomatisasi proses dan pemantauan efisiensi mesin secara real-time. Di bidang kesehatan, \textit{IoT} berkontribusi pada pengembangan perangkat medis yang terhubung untuk pemantauan kesehatan pasien. Dalam transportasi, \textit{IoT} mendukung pengembangan kendaraan otonom dan sistem manajemen lalu lintas cerdas. Di sektor pertanian, \textit{IoT} digunakan untuk memantau kondisi tanah dan iklim, membantu petani dalam pengambilan keputusan seperti pada gambar \ref{fig:iot-kehidupan-sehari-hari}

Dalam penerapannya \textit{IoT} dapat dibagi menjadi tiga lapisan, yaitu lapisan persepsi, lapisan jaringan, dan lapisan aplikasi secara berurutan. Lapisan persepsi bertanggung jawab atas pengumpulan data dalam \textit{IoT}. Lapisan ini terdiri dari berbagai jenis sensor, seperti sensor suhu, sensor kelembaban, RFID, kamera, GPS, dan sebagainya. Lapisan jaringan terdiri dari berbagai jenis jaringan, seperti internet, jaringan komunikasi 2G dan 3G, serta jaringan siaran. Lapisan jaringan terutama digunakan untuk mengumpulkan data dari lapisan persepsi dan memproses data tersebut untuk lapisan aplikasi. Terakhir yaitu Lapisan aplikasi, Lapisan ini adalah antarmuka antara pengguna dan \textit{IoT}. Banyak aplikasi, termasuk logistik, rantai pasokan, pertanian, industri, keamanan publik, pengelolaan perkotaan, telemedis, rumah pintar, transportasi pintar, dan pemantauan lingkungan, diaktifkan melalui \textit{IoT} \parencite{SmartHomeSystemBasedOnIoTTechnologies}.

Seiring bertambahnya jumlah perangkat \textit{IoT} perlu dibuat suatu cara agar sistem \textit{scalable}. \textit{Device discovery} merupakan salah satu masalah yang perlu diatasi untuk membuat sistem \textit{IoT} yang \textit{scalable} karena dapat meningkatkan \textit{quality of service} sehingga meningkatkan availabilty \parencite{DeviceDiscovery}. Tidak hanya itu, banyak munculnya perangkat \textit{IoT} baru yang memerlukan \textit{update} secara berkala menimbulkan masalah baru yaitu sulitnya untuk melakukan \textit{update} untuk setiap perangkat yang ada apabila jumlahnya semakin meningkat sehingga peran \textit{remote deployment} menjadi sangat penting dalam menyelesaikan masalah ini \parencite{RemoteDeployment}.

% \section{\textit{Service}}

\textit{Service} dapat diartikan sebagai sebuah teknologi atau cara kerja yang dapat digunakan untuk memberikan keuntungan ataupun menyelesaikan suatu pekerjaan. Selain itu \textit{service} juga dapat diartikan sebagai abstraksi dari sebuah proses bisnis. \parencite{osullivan2002}

Secara umum \textit{Service} Memiliki tiga fitur utama yaitu.
\begin{enumerate}
  \item \textit{Service} dapat melakukan sebuah aksi atau pekerjaan untuk orang lain,
  \item \textit{Service} adalah sebuah aset yang memiliki sebuah nilai yang dapat diturunkan dari penyedia ke pengguna,
  \item \textit{Service} dapat di bungkus pada service lainnya (sub-services),
\end{enumerate}

Untuk menggabungkan \textit{service}, terdapat dua cara yaitu agregasi dan komposisi. Agregasi memiliki pendekatan untuk menggabungkan dua atau lebih \textit{service} dan membuat satu \textit{entrypoint} untuk seluruh \textit{service} tersebut. Berbeda dengan komposisi, komposisi menggunakan pendekatan untuk mengintegrasikan seluruh sub-\textit{services} yang ada dengan cara membuat jalur komunikasi antar \textit{service} dan setiap \textit{service} memiliki hubungan tertentu dengan \textit{service} lainnya.

Interaksi pada sebuah \textit{service} haruslah minimal memiliki tiga elemen, \textit{service provider}, \textit{service requestor/client}, serta \textit{\textit{service registry}}, namun terdapat juga elemen ke empat pada beberapa kasus yaitu \textit{service broker}. Illustrasi dapat dilhat pada gambar \ref{fig:tiga-elemen-service}

\begin{figure}[ht]
  \centering
  \includegraphics[width=0.8\textwidth]{chapter-2/tiga-elemen-service.jpg}
  \caption{Tiga Elemen pada Interaksi Service, \parencite{abugessaisa2023}}
  \label{fig:tiga-elemen-service}
\end{figure}

\textit{Service requestor} mengirimkan permintaannya serta kebutuhannya kepada \textit{service registry} untuk dicarikan service yang sesuai dengan kebutuhannya. Namun pada beberapa kasus khusus, \textit{service requestor}
sudah memiliki \textquotedblleft contract \textquotedblright dengan \textit{provider} tujuannya sehingga bisa langsung melakukan \textit{request} ke \textit{provider} tanpa bantuan dari \textit{service registry}.

Service \textit{provider} menyediakan \textit{service} yang dapat dikonsumsi oleh publik, agar \textit{service} nya dapat digunakan oleh requestor, \textit{service} \textit{provider} akan memberikan list \textit{services} yang dimiliki kepada registry untuk disimpan pada sebuah \textquotedblleft catalogue \textquotedblright yang dimiliki oleh registry.\textit{Service registry} berperan sebagai penengah dalam komunikasi \textit{requestor} dan \textit{provider}. \textit{Service registry} memiliki \textquotedblleft catalogue \textquotedblright yang menyimpan list berbagai macam \textit{service provider} yang dapat digunakan oleh \textit{service requestor}.

Dengan adanya ketiga elemen tersebut, interaksi pada \textit{services} dapat berjalan sehingga menciptakan suatu fungsionalitas seperti proses bisnis ataupun pemenuhan kebutuhan lainnya.


% \section{\textit{Service discovery}}

\textit{Service discovery} adalah proses mencari layanan yang tersedia dan relevan untuk permintaan tertentu berdasarkan deskripsi semantik fungsional dan non-fungsional. \parencite{klusch2014servicediscovery}. \textit{Servie discovery} merupakan masalah yang sangat penting untuk menyelesaikan masalah konfigurasi. \textit{Service discovery} memungkinkan perangkat dan layanan untuk saling menemukan satu sama lain, mengatur diri, dan berkomunikasi dengan lancar. Namun, seringkali \textit{service discovery} tidak dimanfaatkan secara baik sehingga menghabiskan waktu untuk mencari layanan secara aktif dan mengatur perangkat dan program secara manual. Terkadang, konfigurasi layanan memerlukan keahlian khusus lainnya yang tidak berhubungan dengan apa yang ingin dicapai, sehingga hal ini membuat proses menjadi terhambat. Service discovery membantu untuk mengatasi masalah ini dengan membuat proses ini lebih otomatis dan mudah dilakukan \parencite{ServiceDiscovery}.

Dalam pengaplikasiannya terdapat beberapa \textit{pattern} yang dapat diaplikasikan dalam implementasi \textit{service discovery}. Menurut \parencite{micoservicearchitecture}, terdapat lima \textit{pattern} yang dapat digunakan ketika mengimplementasikan \textit{service discovery} yaitu \textit{3rd Party Registration, Client Side Serivce, Self Registration, serta Server-side service discovery}. Masing masing \textit{pattern} memiliki keunggulan dan kegunaan khusus sehingga perlu dipahami fungsi dari setiap \textit{pattern}.

\subsection{\textit{3rd Party Registration}}
\textit{3rd Party Registration pattern} adalah solusi yang digunakan pada kasus ketika service dapat di-\textit{register} dan \textit{unregister} pada \textit{registar} atau \textit{provider} nya. Solusi ini mewajibkan untuk melakukan registrasi kepada \textit{registar} ketika layanan baru dinyalakan dan melakukan \textit{unregister} ketika layanan dimatikan. layanan \textit{3rd party} yang bertanggung jawab sebagai \textit{registar} untuk mengelola hal ini \parencite{3rdpartyintegration}. Beberapa \textit{tools} yang telah menyediakan sistem layanan seperti ini yaitu \textit{Netflix Prana} yang akan bertindak sebagai \textit{side car} untuk aplikasi non-\textit{JVM} seperti Eureka ataupun AWS \textit{autoscaling groups} yang akan secara otomatis untuk meregister dan menghilangkan EC2 instance pada \textit{load balancer} nya.

Keuntungan untuk menggunakan \textit{pattern} ini ialah proses pembuatan kode yang mudah karena bergantung kepada \textit{3rd party} untuk proses registrasinya. Serta layanan \textit{3rd party} pun akan bertanggung jawab untuk melakukan pengecekan secara berkala agar sistem terus aktif sehingga meningkatkan \textit{availabilty} sistem. Namun, terdapat beberapa kekurangan diantaranya layanan \textit{3rd party} ini hanya bisa untuk melakukan \textit{discovery} secara umum. Seringkali dibutuhkan suatu solusi spesifik yang menjawab permasalahan khusus yang tidak bisa hanya diselesaikan dengan solusi seperti ini, perlu dibuat solusi \textit{custom} yang dapat menambah komponen baru ataupun arsitektur lain yang dapat menyelesaikan masalah tersebut \parencite{3rdpartyintegration}.

\subsection{\textit{Client Side}}
\textit{Client Side pattern} adalah suatu solusi \textit{service discovery} dengan proses \textit{client} yang mencari lokasi \textit{service} yang ingin dituju. \textit{Client} akan mencari lokasi dari tujuan dengan cara melakukan \textit{query} ke \textit{service registry} sehingga lokasi dari \textit{service} yang dituju dapat ditemukan secara dinamis dan \textit{runtime}. Beberapa keunggulan dari \textit{pattern} ini adalah mengurangi kompleksitas dan latensi karena mengurangi jumlah \textit{node} yang perlu di kunjungi. Namun salah satu kekurangannya yaitu meningkatnya kompleksitas kode yang dibuat, karena proses pencarian \textit{service} diletakkan di \textit{client} maka proses pencarian dan perubahan sepenuhnya dibebankan pada \textit{client}. Hal ini membuat proses pencarian tidak \textit{scalable} dan sulit ketika akan melakukan perubahan \parencite{clientsidediscovery}. Visualisasi dari \textit{client side service discovery} dapat dilihat pada \textbf{Gambar \ref{fig:client-side-discovery}}

\begin{figure}[ht]
  \centering
  \includegraphics[width=1\textwidth]{resources/chapter-2/client-side-discovery.jpg}
  \caption{Visualisasi \textit{client side discovery} \parencite{clientsidediscovery} }
  \label{fig:client-side-discovery}
\end{figure}

\subsection{\textit{Self Registration}}
\textit{Pattern self registration service discovery} merupakan \textit{pattern} yang mudah untuk dilakukan. Pada dasarnya \textit{pattern} ini mirip dengan \textit{3rd party registration} yaitu sistem akan meregister \textit{service} ketika service dinyalakan dan \textit{service} akan di-\textit{unregister} ketika dimatikan. Meskipun solusi ini sederhana namun terdapat dua halangan diantaranya

\begin{enumerate}
  \item \textit{service} yang \textit{crash} harus dapat di-\textit{unregister} dari \textit{registry}
  \item \textit{service} yang sedang tidak bisa melayani permintaan harus di-\textit{unregister} dari \textit{registry} untuk menghindari terjadinya \textit{crash}
\end{enumerate}

Selain itu \textit{client} juga harus melakukan poling untuk mendapatkan data terbaru mengenai \textit{service} yang sedang aktif dan dapat menerima permintaan \parencite{selfregistration}.

\subsection{\textit{Server-side}}
Solusi ini memiliki visualisasi yang sama dengan \textit{client side discovery} pada \textbf{Gambar \ref{fig:client-side-discovery}}. Perbedaan yang mencolok yaitu alih alih \textit{client} yang melakukan request ke \textit{service registry}, \textit{client} akan melakukan \textit{request} kepada \textit{load balancer} dan \textit{load balancer} lah yang akan melakukan pencarian layanan mana yang sedang aktif dan dapat menerima permintaan. Cara ini merupakan cara yang paling umum digunakan untuk melakukan \textit{scaling} pada \textit{service} \parencite{servicesidediscovery}.


\section{Penelitian dan Riset Terkait}
\label{sec:riset-terkait}
Berikut adalah beberapa penelitian dan riset yang pernah dilakukan sebelumnya dan berhubungan dengan tugas akhir ini.

\subsection{\textit{Large-Scale Provisioning of
    Resource Constrained
    IoT Deployments, LEONORE}}
\label{subsec:leonore}

Riset dilakukan oleh Michael Vogler, Johannes M. Schleicher, Christian Inzinger, Stefan Nastic, Sanjin Sehic and Schahram Dustdar dari Vienna University of Technology. Riset ini menjelaskan mengenai cara pembuatan sebuah infrastruktur untuk melakukan \textit{provisioning} perangkat \textit{IoT} dalam skala besar. Penelitian ini berfokus untuk membuat sebuah pendekatan terstruktur dalam menyediakan layanan deployment lingkungan IoT dengan dua metode yaitu Push dan Pull.

LEONORE dibuat untuk menyelesaikan tantangan yaitu mengelola jutaan perangkat \textit{IoT} yang heterogen pada sistem skala besar yang memiliki pada domain \textit{smart city}. Solusi yang sudah ada sering kali, bersifat \textit{partial} ataupun manual dalam menangani sebagian infrastruktur. Tentunya, Hal ini tidak efisien dan mahal karena membutuhkan banyak tenaga kerja. Oleh karena itu, LEONORE menghadirkan solusi provisioning yang skalabel dan elastis untuk mengelola perubahan dan kebutuhan baru.

Arsitektur LEONORE dibuat dengan membuat 4 API yang dapat diakses oleh mulai dari  \textit{User API, Repository API, Device API, serta Provisioning}. Arsitektur ini memilki cara kerja yaitu menyimpan seluruh image atau dapat disebut sebagai deployment plan pada \textit{repository}, apabila repository membutuhkan dependency lain maka akan diletakan pada bagian \textit{artifact}. Artifact dibuat dan di proses oleh \textit{package builder} yang seluruh \textit{resources} nya diatur oleh komponen manajemen yaitu \textit{package, dependency management serta gateway management}. Setelah siap untuk di\textit{deploy}, bagian \textit{iot gateway handler} melakukan provisioning kepada target \textit{device}. Secara umum arsitektur leonore dapat dilihat pada gambar \ref{fig:arsitektur-leonore}

\begin{figure}[ht]
  \centering
  \includegraphics[width=0.8\textwidth]{resources/chapter-2/arsitektur-leonore.jpg}
  \caption{Arsitektur Leonore \parencite{vogler2015leonore}}
  \label{fig:arsitektur-leonore}
\end{figure}

Illustrasi cara kerja LEONORE dapat dilihat pada gambar \ref{fig:sequence-leonore}. Berikut adalah cara kerja dari sistem LEONORE:
\begin{enumerate}
  \item Mengecek Ketersediaan \textit{Artifact}.
  \item Mengambil \textit{Iot Gateway} sebagai group yang akan di \textit{deploy}.
  \item \textit{Delegate deployment} task ke setiap \textit{IoT Gateway}.
  \item Analisa kompatibilitas untuk setiap nodes.
  \item Resolve \textit{dependency} dan membuat \textit{application package} sesuai dengan group \textit{IoT Gateway}
  \item Jalankan \textit{provisioning}
  \item Tunggu hingga setiap nodes selesai lalu lakukan cek untuk setiap nodes hingga proses selesai
\end{enumerate}

\begin{figure}[ht]
  \centering
  \includegraphics[width=0.8\textwidth]{resources/chapter-2/leonore-sequence.jpg}
  \caption{Sequence Diagram Proses \textit{Deployment} LEONORE \parencite{vogler2015leonore}}
  \label{fig:sequence-leonore}
\end{figure}

Untuk mengevaluasi kinerja LEONORE, dilakukan pengujian pada \textit{cloud}, berisi 1000 perangkat yang divirtualisasi menggunakan \textit{docker}. Hasilnya LEONORE dapat melakukan \textit{provisioning} seluruh perangkat dengan waktu yang \textit{reasonable}. Metode \textit{pull provisioning} menghasilkan latensi yang tinggi dibandingkan metode \textit{push} yang memberikan hasil yang cukup baik.

\subsection{China Electronic Toll Colleciton}
China memiliki masyarakat yang sangat banyak dan setiap masyarakat memiliki sekurang kurangnya satu kendaraan. Seiring bertambah nya masyarakat di China, maka jalanan yang ada di China akan semakin penuh dengan kendaraan yang akan menghasilkan kemacetan terutama pada tol bagian pembayaran. Untuk mengatasi masalah ini China menggunakan \textit{Electronic Toll Colleciton (ETC)} yang di integrasikan dengan setiap kendaraan untuk mempercepat proses ini \parencite{penelitianterkait1}.

China menggunakan KubeEdge untuk melakukan proses \textit{deployment} \textit{ETC} untuk 100,000 \textit{nodes} dengan total 500,000 aplikasi yang diluncurkan menggunakan KubeEdge tersebar untuk 29 dari 34 provinsi. Proses \textit{deployment} dilakukan secara otomatis dengan membuat sistem \textit{workflow engine} pada kubernetes sehingga proses \textit{deployment} dapat dilakukan dengan cepat dan mudah. Dengan menggunakan metode ini sistem pembayaran tol di China menjadi 10x lebih cepat dari sebelumnya \parencite{penelitianterkait1}.

\begin{figure}[h]
  \centering
  \includegraphics[width=0.8\textwidth]{resources/chapter-2/china-highways.jpg}
  \caption{Implementasi sistem \textit{ETC} di China \parencite{penelitianterkait1}}
  \label{fig:china-highways}
\end{figure}

\begin{figure}[h]
  \includegraphics[width=0.8\textwidth]{resources/chapter-2/arsitektur-china-highways.jpg}
  \caption{Arsitektur sistem \textit{ETC} di China \parencite{penelitianterkait1}}
  \label{fig:architecture-china-highways}
\end{figure}
\subsection{A Model for the Remote Deployment, Update, and Safe Recovery for Commercial Sensor-Based IoT Systems}
Penelitian ini menggali tantangan-tantangan khusus terkait infrastruktur yang didedikasikan untuk penyebaran dan manajemen aplikasi secara jarak jauh. Penelitian ini membahas tantangan-tantangan manajemen terkait sistem sensor IoT dan mengusulkan sebuah cara serta metodologi untuk mengatasi hal tersebut.

Penelitian ini mengimplementasikan solusi sebagai sistem infrastruktur perangkat lunak untuk produk IoT bisnis yang lengkap. Penelitian ini melakukan \textit{deployment} pada 100 perangkat penjual minuman yang tersebar di tiga lokasi. Setiap perangkat bergantung pada sensor yang memantau statusnya dan pada \textit{gateway} yang mengendalikan perilakunya. Arsitektur sistem dapat dilihat pada gambar \ref{fig:architecture-remote-deployments}.

\begin{figure}[ht]
  \centering
  \includegraphics[width=0.8\textwidth]{resources/chapter-2/arsitektur-remote-deployment.jpg}
  \caption{Arsitektur Remote \textit{Deployment} \parencite{RemoteDeployment}}
  \label{fig:architecture-remote-deployments}
\end{figure}

Selama penelitian ini berlangsung, penelitian ini berhasil menerima 133 \textit{update} pada perangkat IoT. 80\% perangkat beroperasi tanpa gangguan selama 250 hari. Sedangkan, 20\% mengalami kegagalan akibat faktor eksternal. Dari 80\% tersebut, 30\% mengalami kegagalan pembaruan sementara akibat kapabilitas perangkat yang berkurang \parencite{RemoteDeployment}.

Solusi yang dibuat penelitian ini mengandalkan keamanan serta \textit{failsafe} yang dapat melakukan \textit{remote deployment} dengan baik serta aman sehingga dapat mendeteksi kegagalan yang terjadi pada perangkat dan melakukan \textit{recovery} dengan cepat. Berikut merupakan beberapa cara untuk melakukan \textit{remote deployment} atau seringkali disebut sebagai OTA \textit{(Over the air)}.

\begin{figure}[ht]
  \centering
  \includegraphics[width=0.8\textwidth]{resources/chapter-2/perbandingan-remote-deployment.jpg}
  \caption{Perbandingan Tata Cara \textit{Remote Deployment} \parencite{RemoteDeployment}}
  \label{fig:comparison-remote-deployments}
\end{figure}

Dapat dilihat dari gambar \ref{fig:comparison-remote-deployments} bahwa terdapat berbagai solusi untuk berbagai tipe \textit{remote deployment}. Pada kasus ini, dapat dibuat suatu cara yang mengadopsi \textit{update type} serta koneksi dari keempat tipe tersebut. Perangkat melakukan \textit{polling} kepada \textit{server} untuk mengecek apakah terdapat versi terbaru atau tidak. Selain itu, dari sisi \textit{Server} juga dapat membuat suatu notifikasi yang dapat diterima oleh perangkat jika terdapat \textit{update} baru yang siap digunakan.
% \blankpage
\chapter{Analisis Persoalan dan Rancangan Solusi}

Tujuan utama penulisan bab ini adalah untuk menguraikan rencana penyelesaian masalah implementasi sistem \textit{remote deployment} pada lingkungan IoT. Bagian ini memaparkan proses analisis masalah hingga menjadi solusi.


\section{Analisis}

\section{Analisis Persoalan}

\textit{Autoscaler} merupakan salah satu teknologi yang bermanfaat untuk membantu pengelola infrastruktur dalam melakukan \textit{scaling} pada teknologi \textit{container orchestration} seperti Kubernetes. Dengan menggunakan \textit{autoscaler}, pengguna dapat mengatur dan mengontrol (\textit{scaling}) sumber daya dari sekumpulan \textit{pods} dengan otomatis. Pada teknologi \textit{autoscaler} yang paling sederhana, sistem akan secara otomatis melakukan \textit{scaling} apabila suatu metrik melewati ambang batas tertentu. Beberapa \textit{autoscaler} yang paling populer saat ini adalah \textit{horizontal autoscaling} dan \textit{vertical autoscaling} yang melakukan \textit{scaling} berdasarkan metrik sistem seperti utilisasi prosesor, memori serta beban permintaan dalam satuan waktu. Perkembangan dari metode \textit{autoscaling} telah menyediakan pengguna berbagai opsi untuk mengotomasi alokasi sumber daya. Namun, dari semua metode yang sekarang sudah ada, terdapat beberapa kekurangan yang dapat diperbaiki, diantaranya sebagai berikut.

\begin{enumerate}
    \item Terdapat keperluan pengelola untuk menambah dan mengurangi secara manual terhadap konfigurasi \textit{deployment autoscaler} yang beracuan pada \textit{throughput} yang sebenarnya bisa diotomasi dengan sistem melalui aturan yang dibuat oleh pengelola.
    \item \textit{Autoscaler} sederhana umumnya dibuat untuk sesuatu yang sangat umum seperti pemrosesan seperti \textit{web service}, \textit{CRON Job}, dan sebagainya. Hal ini sedikit berbeda jika berkaitan dengan sistem \textit{information retrieval}, karena sistem \textit{information retrieval} memiliki data yang berbeda sehingga kegunaan dan keperluannya menjadi bervariatif.
    \item \textit{Metrics} yang digunakan tidak melihat data waktu sebelumnya yang bisa menjadi acuan untuk melihat pola penggunaan. Ditambah dengan fakta bahwa \textit{autoscaler} sederhana berpatokan dengan sebuah ambang batas sehingga tidak efisien terhadap kinerja sistem yang fluktuatif. Apabila hanya mengandalkan angka ambang batas, ketika sistem melewati angka ambang batas untuk waktu yang sangat singkat, maka sistem akan mencoba untuk melakukan \textit{scaling} yang sebenarnya bisa dianggap tidak perlu jika hanya \textit{scaling} untuk waktu yang singkat, \parencite{riset1}. 
    \item Setiap pengelola infrastruktur memiliki variabel \textit{cost} dan target kinerja sistem yang berbeda-beda. Toleransi terhadap target kinerja sistem dan \textit{cost} yang berbeda memerlukan \textit{autoscaler} yang dapat dikonfigurasi dengan fleksibel.
\end{enumerate}

Pada tugas akhir ini, akan dilakukan penelitian untuk melakukan pengembangan metode \textit{autoscale} yang berjalan diatas Kubernetes yang spesifik untuk mengontrol alokasi sumber daya \textit{Elastic Search}. Dengan melakukan pengembangan tersebut, diharapkan penelitian ini dapat meningkatkan efisiensi \textit{autoscale} pada \textit{Elastic Search} dengan kontrol fleksibel berdasarkan model prediktif berbasis \textit{time series}. Pendekatan prediksi berbasis \textit{time series} dilakukan karena metrik sistem sangat dekat dengan data historis dan korelasi data dengan waktu. Melalui referensi studi literatur yang sudah dilakukan, banyak penelitian yang menggunakan model prediksi dengan \textit{time series} seperti ARIMA, LSTM, dan Bi-LSTM.

% Namun, saat ini kontrol adaptif milik Kubernetes memiliki beberapa kekurangan. Salah satu kekurangan dari kontrol adaptif kubernetes adalah \textit{trigger autoscale} yang didasarkan oleh metriks umum seperti utilisasi memori dan prosesor berdasarkan waktu saat itu. Sedangkan, tidak semua \textit{container} dinyatakan memakai sumber daya komputasi secara efisien jika hanya memakai faktor tersebut sebab tidak semua \textit{container} bergantung hanya pada utilisasi prosesor dan memori. Terkadang ada beberapa fitur pada sebuah \textit{container} yang terus berjalan namun tidak terpakai atau disimpan pada \textit{cache} namun tidak dipakai. Semuanya kembali lagi kepada isi dari sebuah \textit{container} serta penggunaannya. Selain dari hal tersebut, kubernetes juga hanya memakai data \textit{metrics} yang sedang berlangsung. Sehingga, kubernetes tidak dapat memprediksi kebutuhan \textit{resource} pada waktu yang akan datang melalui pola penggunaan. Hal ini akan menjadi acuan untuk meningkatkan efisiensi \textit{autoscale} dari kubernetes terutama pada \textit{Elastic Search} melalui adaptif kontrol dengan model prediktif berbasis \textit{time series}.

\section{Analisis Solusi}

Untuk melakukan penelitian pengembangan metode \textit{autoscale} tersebut, dilakukan pemetaan tantangan, penanganan serta kebutuhan untuk melakukan penanganan tersebut.

\subsection{Pemetaan Tantangan dan Penanganan}
\label{sec:pemetaan-masalah}
Tantangan yang ada akan dipetakan dengan penanganan yang akan dilakukan. Pemetaan tersebut dapat dilihat pada Tabel \ref{tab:pemetaan-masalah}.

\begin{table}[h]
    \caption{Tabel Pemetaan Tantangan dan Penanganan}
    \vspace{0.25cm}
    \begin{center}
        \begin{tabular}{|c|p{2.5in}|p{2.5in}|}
            \rowcolor{gray!30}
            \hline
            \textbf{Nomor} & \textbf{Tantangan} & \textbf{Penanganan} \tabularnewline
            \hline
            M1 & Terdapat keperluan pengelola untuk menambah dan mengurangi secara manual terhadap konfigurasi \textit{deployment autoscaler} yang beracuan pada \textit{throughput} yang sebenarnya bisa diotomasi dengan sistem melalui aturan yang dibuat oleh pengelola. &
            Menambah variabel \textit{throughput} untuk operasi-operasi \textit{ElasticSearch} yang konfigurasinya dapat diatur otomatis oleh sistem kontrol fleksibel.
            \tabularnewline

            M2 &
            Kubernetes tidak bisa melakukan \textit{autoscale} berdasarkan variabel spesifik pada suatu kontainer, melainkan hanya generalisasi proses pada umumnya. &
            Mengembangkan \textit{autoscaler} pada ranah \textit{information retrieval} yang spesifik pada \textit{Elastic Search}. \tabularnewline

            M3 &
            \textit{Metrics} yang digunakan tidak melihat data waktu sebelumnya dan \textit{autoscaler} sederhana berpatokan dengan sebuah ambang batas sehingga tidak efisien terhadap kinerja sistem yang fluktuatif. &
            Mengembangkan \textit{autoscaler} dengan model prediksi berbasis \textit{time series} sehingga dapat melihat pola dari data historis dan bisa secara preventif menghindari \textit{scaling} akibat fluktuasi data dengan hasil prediksi tersebut.\tabularnewline

            M4 & \textit{Quality of service} dan toleransi kepada \textit{tradeoff} antara efisiensi dan \textit{cost} dapat berbeda antara pengguna. &
            \textit{Autoscaler} harus fleksibel sehingga terdapat ruang untuk pengguna membuat aturan yang berjalan sesuai kondisi-kondisi yang disesuaikan dengan kebutuhan pemakai.\tabularnewline

            % M5 & \textit{Rolling Update} yang terlalu sering mengharuskan kubernetes memiliki sumber daya minimal sejumlah dua kali lipat. &
            % Menggunakan teknologi \textit{In-Place Update of Pod Resources} \tabularnewline
            \hline
        \end{tabular}
    \end{center}
    \label{tab:pemetaan-masalah}
\end{table}

\section{Analisis Rancangan Solusi}
\label{sec:analisis-rancangan-solusi}

Berdasarkan beberapa pendekatan yang telah dijelaskan pada \textbf{Bagian \ref{sec:analisis-solusi}}, Membuat sistem \textit{remote deployment}, yang dirancang untuk provisioning skala besar dari aplikasi pada perangkat \textit{IoT} yang terbatas sumber daya, menggunakan Kubernetes menjadi solusi yang akan digunakan karena beberapa keuntungan yang tidak dimiliki oleh pendekatan lain. Kubernetes, sebagai sistem orkestrasi kontainer yang matang dan luas digunakan, menawarkan fitur dan kemampuan yang cocok untuk mengatasi tantangan yang dihadapi dalam lingkungan \textit{IoT} yang heterogen dan terdistribusi terutama dalam masalah skalabilitas, \textit{ready to use}, serta memakan waktu yang minimal sehingga solusi ini merupakan solusi paling feasible yang dapat diimplementasikan

\subsection{Analisis kebutuhan sistem}

Sebelum sistem dibuat, pertama tama akan dibuat dan dicari kebutuhan sistem yang diperlukan. Kebutuhan sistem akan dibagi menjadi kebutuhan fungsional dan non-fungsional.

\subsubsection{Deskripsi sistem}
Sistem yang akan dibuat merupakan sebuah sistem yang dapat melakukan sebuah aksi, command, ataupun \textit{remote deployment} ke perangkat yang terhubung ataupun ke sebagian perangkat yang diinginkan. Sistem memiliki dua komponen utama yaitu dashboard sebagai \textit{frontend} serta API sebagai \textit{backend}. Komponen \textit{backend} ini memiliki dua modul eksternal dan satu modul internal. Kubernetes dan database sebagai modul eksternal serta server sebagai penghubung modul eksternal sebagai modul internal. Modul kubernetes akan terhubung ke kluster yang diinginkan untuk melakukan proses \textit{remote deployment} ke masing masing perangkat yang terhubung.

\begin{figure}[h]
  \centering
  \includegraphics[width=1\textwidth]{resources/chapter-3/gambaran-umum-arsitektur-updated.jpg}
  \caption{Gambaran Umum Arsitektur}
  \label{fig:gambaran-umum-arsitektur}
\end{figure}

\pagebreak
\subsubsection{Karakteristik Pengguna}
Berdasarkan hasil analisis, terdapat dua pengguna pada sistem ini, yaitu pengguna dan administrator. Penjelasan lebih detil dapat dilihat pada tabel \ref{tab:karakteristik-pengguna}.

\bgroup
\begin{table}[ht]
  \def\arraystretch{1.7}
  \caption{Karakteristik Pengguna}
  \label{tab:karakteristik-pengguna}
  \centering
  \begin{tabular}{|p{2cm}|p{8cm}|}
    \hline
    Kategori Pengguna & Hak akses                                                                                                                                                                                                                                                                           \\
    \hline
    \textit{User}     & \textit{User} dapat melakukan login, registrasi, melihat \textit{user} lain di satu perusahaan, melakukan manajemen perangkat, melakukan manajemen groups, melakukan manajemen \textit{deployment}, melakukan \textit{remote deployment}, serta melihat riwayat \textit{deployment} \\
    \hline
    Admin             & Admin dapat melakukan manajemen perusahaan, manajemen user, serta seluruh kegiatan yang user dapat lakukan                                                                                                                                                                          \\
    \hline
  \end{tabular}
\end{table}
\egroup

\subsubsection{Kebutuhan Fungsional}
Berikut merupakan kebutuhan fungsional dari sistem yang dibuat, agar lebih jelas kebutuhan fungsional dibuat ke dalam bentuk tabel di bawah ini. Semua kebutuhan fungsional memiliki ID yang diawali dengan huruf F lalu diikuti dengan dua angka. Tabel dapat dilihat pada lampiran \ref{tab:kebutuhan-fungsional}



\subsubsection{Kebutuhan Non-Fungsional}
Sistem memiliki 5 parameter kebutuhan non-fungsional yang dapat dilihat pada tabel \ref{tab:kebutuhan-non-fungsional}. Semua kebutuhan non-fungsional memiliki awalan ID NF lalu diikuti oleh dua angka.

\bgroup
\begin{table}[ht]
  \def\arraystretch{1.7}
  \caption{Kebutuhan Non-Fungsional}
  \label{tab:kebutuhan-non-fungsional}
  \centering
  \begin{tabular}{|c|p{3cm}|p{8cm}|}
    \hline
    ID   & Parameter               & Kebutuhan                                                                                                                                                                                                                          \\
    \hline
    NF01 & \textit{Maintanability} & Sistem dapat dimaintain dari jauh dengan mudah dengan melakukan aksi melalui \textit{dashboard}.                                                                                                                                   \\
    \hline
    NF02 & \textit{Security}       & Sistem menjamin keamanan dari \textit{service} dengan cara menyediakan validasi pada \textit{middleware} serta setiap \textit{request} dapat terhindar dari serangan. Hanya \textit{user} yang terautentikasi yang bisa mengakses. \\
    \hline
    NF03 & \textit{Portability}    & Sistem dapat diakses dimana saja melalui perangkat laptop.                                                                                                                                                                         \\
    \hline
  \end{tabular}
\end{table}
\egroup

\subsubsection{Model \textit{Use case}}
\label{subsec:model-usecase}
Dari beberapa kebutuhan fungsional serta karakteristik pengguna, dapat dibuat \textit{use case} yang mengelompokkan serta menggambarkan relasi antara aktor dan aksi yang dapat dilakukan. \textit{Use case} memiliki identifikasi yang berawalan dengan UC diikuti oleh dua angka. \textit{Use case} dapat dilihat secara detail pada lampiran \ref{tab:penjelasan-usecase-diagram}.

Dari pemetaan \textit{use case} pada lampiran \ref{tab:penjelasan-usecase-diagram}, dapat dibuat sebuah diagram yang menghubungkan relasi antara aktor dengan \textit{use case}-nya. Relasi  aktor dengan kapabilitas fungsional sistem dapat dilihat pada diagram use case di Gambar \ref{fig:usecase-diagram}.

\begin{figure}[ht]
  \centering
  \includegraphics[width=1\textwidth]{resources/chapter-3/usecase-diagram.jpg}
  \caption{Usecase Diagram}
  \label{fig:usecase-diagram}
\end{figure}

\pagebreak


\section{Rancangan}

\section{Arsitektur Sistem}
Seperti yang telah digambarkan pada \textbf{Gambar \ref{fig:gambaran-umum-arsitektur}}, arsitektur ini memiliki dua komponen utama yang menjadi dasar dari sistem. Yaitu \textit{dashboard} dan \textit{service}. Dashboard memiliki dua modul yaitu modul untuk menampilkan halaman halaman yang bersesuaian dan modul untuk melakukan koneksi dengan \textit{service}.

Service memiliki tiga module yaitu server, database, serta kubernetes. Server akan menjadi pusat kontrol dari \textit{request} yang dikirimkan melalui \textit{dashboard}. Masing masing perangkat akan di \textit{manage} oleh modul kubernetes yang telah memiliki sistem cluster terpisah. Penjelasan arsitektur secara detail akan dijelaskan pada dua subbab yaitu arsitektur struktural serta arsitektur behavioural.

\subsection{Rancangan Struktural}
\label{subsec:arsitektur-struktural}

Arsitektur struktural yang digunakan ialah \textit{package diagram}. \textit{Package diagram} menggambarkan dengan jelas hubungan antara sistem maupun subsistem yang ada. Sistem digambarkan dengan sebuah box dan modul digambarkan dengan persegi panjang yang berada pada dalam box-boxnya. Sistem utama dari \textit{remote deployment} hanya terdiri atas \textit{service} dan \textit{dashboard}. Sistem \textit{Kubernetes cluster} sepenuhnya diatur oleh modul Kubernetes yang ada pada sistem \textit{service}. Ilustrasi dapat dilihat pada gambar \ref{fig:package-diagram}.

\begin{figure}[ht]
  \centering
  \includegraphics[width=0.9\textwidth]{resources/chapter-3/package-diagram.jpg}
  \caption{Package Diagram}
  \label{fig:package-diagram}
\end{figure}



\subsection{Arsitektur \textit{Behavioural}}

Berdasarkan \textit{use case} diagram yang telah dibuat, terdapat 13 Use case yang memiliki alur yang berbeda. Untuk menjelaskan interaksi aktor, sistem, serta objek secara terperinci akan digunakan \textit{sequence} diagram. Terdapat 13 \textit{sequence} pada sistem ini, diagram ini sudah termasuk interaksi antara sistem \textit{dashboard} dan service.

\subsubsection{Alur Mendaftarkan Perusahaan}

Pada \textit{use case} ini, admin yang berperan untuk mendaftarakan perusahaan baru yang ingin mendaftarkan ke dalam sistem. Admin dapat melakukan request melalui HTTP Client ke server. Server akan melakukan validasi data dan apabila telah pass, server akan memasukan informasi ke database. Akhirnya, server akan memberikan response berupa objek dari perusahaan yang dapat digunakan.

\begin{figure}
  \centering
  \includegraphics[width=1\textwidth]{resources/chapter-3/usecase/uc-01.jpg}
  \caption{\textit{Use case} Mendaftarkan perusahaan}
  \label{fig:usecase-01}
\end{figure}

\pagebreak

\subsubsection{Alur Mendaftarkan \textit{User}}

Pada \textit{use case} ini, admin yang berperan untuk mendaftarakan \textit{user} baru yang ingin mendaftarkan ke dalam sistem. Admin dapat melakukan request melalui HTTP Client ke server. Server akan melakukan validasi data dan apabila telah pass, server akan memasukan informasi ke database. Akhirnya, server akan memberikan response berupa objek \textit{user} yang dapat digunakan untuk login oleh \textit{user}.

\begin{figure}[ht]
  \centering
  \includegraphics[width=1\textwidth]{resources/chapter-3/usecase/uc-02.jpg}
  \caption{\textit{Use case} Mendaftarkan \textit{User}}
  \label{fig:usecase-02}
\end{figure}

\subsubsection{Alur Manajemen Perusahaan}

Pada \textit{use case} ini, admin yang berperan untuk melakukan manajemen perusahaan. Admin dapat menghapus ataupun mengupdate detail perusahaan. Server akan melakukan validasi data dan apabila telah pass, server akan melakukan update informasi yang diberikan ke database. Server akan memberikan repsonse berupa hasil update.

\begin{figure}
  \centering
  \includegraphics[width=1\textwidth]{resources/chapter-3/usecase/uc-03.jpg}
  \caption{\textit{Use case} Manajemen perusahaan}
  \label{fig:usecase-03}
\end{figure}

\subsubsection{Alur Manajemen \textit{User}}

Pada \textit{use case} ini, admin yang berperan untuk melakukan manajemen \textit{user}. Admin dapat menghapus ataupun mengupdate detail \textit{user}. Server akan melakukan validasi data dan apabila telah pass, server akan melakukan update informasi yang diberikan ke database. Setelah itu server akan memberikan repsonse berupa hasil update yang telah dilakukan.

\begin{figure}[ht]
  \centering
  \includegraphics[width=1\textwidth]{resources/chapter-3/usecase/uc-04.jpg}
  \caption{\textit{Use case} Manajemen \textit{User}}
  \label{fig:usecase-04}
\end{figure}

\subsubsection{Alur Login}

Pada \textit{use case} ini, \textit{user} dapat login ke dalam aplikasi dengan menginput kredensial berupa email dan password. Apabila data yang diberikan salah maka akan muncul modal untuk menandakan kesalahan yang dibuat. Apabila data benar, maka akan diberikan modal sukses lalu akan dilakukan \textit{redirect} ke halaman utama. Pada sisi server, terdapat beberapa validasi seperti apakah email terdapat di database ataupun password match dengan hashed password yang ada di database. Setelah itu, server akan memberikan response status ok dan memberikan \textit{user} akses ke halaman utama.

\begin{figure}[ht]
  \centering
  \includegraphics[width=1\textwidth]{resources/chapter-3/usecase/uc-05.jpg}
  \caption{\textit{Use case} Alur Login}
  \label{fig:usecase-05}
\end{figure}

\subsubsection{Alur Melihat detail perusahaan}

Pada \textit{use case} ini, \textit{user} dapat melihat detail perusahaan dengan cara mengunjungi halaman \textit{account}. Data akan diambil secara langsung melalui API Call ke server, apabila terdapat error maka terdapat pesan error yang muncul. Apabila data berhasil di dapatkan, akan ditampilkan detail dari perusahaan \textit{user}.

\begin{figure}[ht]
  \centering
  \includegraphics[width=1\textwidth]{resources/chapter-3/usecase/uc-06.jpg}
  \caption{\textit{Use case} Melihat detail perusahaan}
  \label{fig:usecase-06}
\end{figure}

\pagebreak

\subsubsection{Alur Melihat user lain di satu perusahaan}

Pada \textit{use case} ini, \textit{user} dapat melihat detail seluruh user yang berada pada satu perusahaan yang sama dengan cara mengunjungi halaman \textit{account}. Data akan diambil secara langsung melalui API Call ke server, apabila terdapat error maka terdapat pesan error yang muncul. Apabila data berhasil di dapatkan, akan ditampilkan daftar \textit{user} yang bersesuaian.

\begin{figure}[ht]
  \centering
  \includegraphics[width=1\textwidth]{resources/chapter-3/usecase/uc-07.jpg}
  \caption{\textit{Use case} Melihat user lain di satu perusahaan}
  \label{fig:usecase-07}
\end{figure}

\pagebreak

\subsubsection{Alur Manajemen Perangkat}

Pada \textit{use case} ini, \textit{user} dapat melakuakan manajemen perangkat dengan mengunjungi halaman \textit{devices}. Pada halaman ini \textit{user} dapat melakukan beberapa operasi yaitu mengambil daftar perangkat yang terdaftar, menambahkan perangkat baru, dan menghapus perangkat. Operasi pengambilan perangkat yang terdaftar dilakukan secara langsung ketika mengunjungi halaman. Operasi penghapusan ataupun penambahan perangkat dapat dilakukan oleh \textit{user} dengan cara menekan tombol yang ada pada laman. Ketika tombol ditekan terdapat validasi yang dilakukan pada halaman maupun pada server. Setelah melewati tahapan validasi, server akan melakukan update pada database. Apabila terdapat error maka terdapat pesan error yang muncul. Apabila data berhasil di dapatkan, akan ditampilkan sebuah modal yang menandakan operasi berhasil untuk dilakukan.

\begin{figure}[ht]
  \centering
  \includegraphics[width=1\textwidth]{resources/chapter-3/usecase/uc-08.jpg}
  \caption{\textit{Use case} Manajemen Perangkat}
  \label{fig:usecase-08}
\end{figure}

\pagebreak

\subsubsection{Alur Manajemen \textit{groups}}

Pada \textit{use case} ini, \textit{user} dapat melakuakan manajemen \textit{groups} dengan mengunjungi halaman \textit{groups}. Pada halaman ini \textit{user} dapat melakukan beberapa operasi yaitu mengambil daftar \textit{groups} yang terdaftar, menambahkan \textit{groups} baru, dan menghapus \textit{groups}. Operasi pengambilan \textit{groups} yang terdaftar dilakukan secara langsung ketika mengunjungi halaman. Operasi penghapusan ataupun penambahan \textit{groups} dapat dilakukan oleh \textit{user} dengan cara menekan tombol yang ada pada laman. Ketika tombol ditekan terdapat validasi yang dilakukan pada halaman maupun pada server. Setelah melewati tahapan validasi, server akan melakukan update pada database. Apabila terdapat error maka terdapat pesan error yang muncul. Apabila data berhasil di dapatkan, akan ditampilkan sebuah modal yang menandakan operasi berhasil untuk dilakukan.

\begin{figure}[ht]
  \centering
  \includegraphics[width=1\textwidth]{resources/chapter-3/usecase/uc-09.jpg}
  \caption{\textit{Use case} Manajemen \textit{groups}}
  \label{fig:usecase-09}
\end{figure}

\pagebreak

\subsubsection{Alur Manajemen \textit{deployment images}}

Pada \textit{use case} ini, \textit{user} dapat melakuakan manajemen \textit{deployment images} dengan mengunjungi halaman \textit{deployments}. Pada halaman ini \textit{user} dapat melakukan beberapa operasi yaitu mengambil daftar \textit{deployment images} yang terdaftar, menambahkan \textit{deployment images} baru, dan menghapus \textit{deployment images}. Operasi pengambilan \textit{deployment images} yang terdaftar dilakukan secara langsung ketika mengunjungi halaman. Operasi penghapusan ataupun penambahan \textit{deployment images} dapat dilakukan oleh \textit{user} dengan cara menekan tombol yang ada pada laman. Ketika tombol ditekan terdapat validasi yang dilakukan pada halaman maupun pada server. Setelah melewati tahapan validasi, server akan melakukan update pada database. Apabila terdapat error maka terdapat pesan error yang muncul. Apabila data berhasil di dapatkan, akan ditampilkan sebuah modal yang menandakan operasi berhasil untuk dilakukan.

\begin{figure}[ht]
  \centering
  \includegraphics[width=1\textwidth]{resources/chapter-3/usecase/uc-10.jpg}
  \caption{\textit{Use case} Manajemen \textit{deployment images}}
  \label{fig:usecase-10}
\end{figure}

\pagebreak

\subsubsection{Alur Manajemen \textit{deployment plan}}

Pada \textit{use case} ini, \textit{user} dapat melakuakan manajemen \textit{deployment plan} dengan mengunjungi halaman \textit{deployments}. Pada halaman ini \textit{user} dapat melakukan beberapa operasi yaitu mengambil daftar \textit{deployment plan} yang terdaftar, menambahkan \textit{deployment plan} baru, dan menghapus \textit{deployment plan}. Operasi pengambilan \textit{deployment plan} yang terdaftar dilakukan secara langsung ketika mengunjungi halaman. Operasi penghapusan ataupun penambahan \textit{deployment plan} dapat dilakukan oleh \textit{user} dengan cara menekan tombol yang ada pada laman. Ketika tombol ditekan terdapat validasi yang dilakukan pada halaman maupun pada server. Setelah melewati tahapan validasi, server akan melakukan update pada database. Apabila terdapat error maka terdapat pesan error yang muncul. Apabila data berhasil di dapatkan, akan ditampilkan sebuah modal yang menandakan operasi berhasil untuk dilakukan.

\begin{figure}[ht]
  \centering
  \includegraphics[width=1\textwidth]{resources/chapter-3/usecase/uc-11.jpg}
  \caption{\textit{Use case} Manajemen \textit{deployment plan}}
  \label{fig:usecase-11}
\end{figure}

\pagebreak

\subsubsection{Alur \textit{remote deployment}}

Pada \textit{use case} ini, \textit{user} dapat melakukan \textit{remote deployment} dengan \textit{deployment plan} yang telah dibuat. Aksi ini dilakukan dengan cara mengunjungi halaman \textit{deployments} lalu menekan tombol deploy. Akan ada modal yang muncul memilih deployment mana yang akan dilakukan. Ketika user memilih deployment plan yang ingin di \textit{deploy}, terdapat validasi pada server sebelum melakukan deployment pada kubernetes. Setelah semua validasi berhasil dilakukan, kubernetes akan menginformasikan control plane pada cluster yang terhubung untuk memberikan perintah deploy kepada target. Balikan dari seluruh operasi ini adalah response berupa deployment berhasil dilakukan. Terdapat operasi \textit{asynchronus} pada \textit{background} untuk mengecek status deployment user


\begin{figure}[ht]
  \centering
  \includegraphics[width=1\textwidth]{resources/chapter-3/usecase/uc-12.jpg}
  \caption{\textit{Use case} \textit{remote deployment}}
  \label{fig:usecase-12}
\end{figure}

\pagebreak

\subsubsection{Alur melihat riwayat \textit{deployment}}

Pada \textit{use case} ini, \textit{user} dapat melihat detail perusahaan dengan cara mengunjungi halaman \textit{deployments}. Data akan diambil secara langsung melalui API Call ke server, apabila terdapat error maka terdapat pesan error yang muncul. Apabila data berhasil di dapatkan akan ditampilkan \textit{deployment} apa saja yang telah dilakukan beserta statusnya.

\begin{figure}[ht]
  \centering
  \includegraphics[width=1\textwidth]{resources/chapter-3/usecase/uc-13.jpg}
  \caption{\textit{Use case} melihat riwayat \textit{deployment}}
  \label{fig:usecase-13}
\end{figure}

\pagebreak

\section{Rancangan Detail Komponen Dashboard}

\subsection{Rancangan Detail Komponen Service}
\label{sec:rancangan-service}

Berdasarkan \textit{sequence diagram} yang telah dibuat pada bagian \ref{subsec:arsitektur-behavioural}. Sistem \textit{service} dibagi menjadi beberapa domain. Pembagian domain berfungsi untuk memfokuskan implementasi serta memudahkan tahapan testing. Terdapat 6 domain yaitu \textit{company}, \textit{user}, \textit{devices}, \textit{groups}, \textit{deployment}, dan \textit{external services}. Setiap domain memiliki diagram kelas yang menjelaskan rancangan implementasi. \textit{package} diagram dan \textit{class} diagram secara keseluruhan dapat dilihat pada gambar \ref{fig:package-diagram} dan \ref{fig:package-class-domain-diagram}.
\begin{figure}[ht]
  \centering
  \includegraphics[width=1\textwidth]{resources/chapter-3/class/class-diagram-overall.jpg}
  \caption{\textit{Package Diagram Domain Service}}
  \label{fig:package-class-domain-diagram}
\end{figure}

\pagebreak

\subsubsection{Domain \textit{company}}

Domain ini mengatur konektivitas antara server dan \textit{database} dalam hal \textit{company}. Domain ini memiliki tiga lapisan yaitu \textit{handler, usecase, dan repository}. Lapisan \textit{repository} memiliki hubungan dengan \textit{database} serta terdapat lapisan \textit{handler} yang berinteraksi dengan request yang masuk. Ilustrasi \textit{class diagram} dapat dilihat pada gambar \ref{fig:company-class-diagram}.

\begin{figure}[ht]
  \centering
  \includegraphics[width=1\textwidth]{resources/chapter-3/class/company-class-diagram.jpg}
  \caption{\textit{Company Class Diagram}}
  \label{fig:company-class-diagram}
\end{figure}

\pagebreak

\subsubsection{Domain \textit{user}}

Domain ini mengatur konektivitas antara server dan \textit{database} dalam hal \textit{user}. Domain ini juga memliki tiga lapisan mulai dari lapisan paling luar \textit{handler}, diikuti dengan \textit{usecase} lalu terkahir \textit{repository} yang berhubungan dengan \textit{database}. Domain ini juga yang mengatur bagian autentikasi seperti login dan register. Ilustrasi \textit{class diagram} dapat dilihat pada gambar \ref{fig:user-class-diagram}.

\begin{figure}[ht]
  \centering
  \includegraphics[width=1\textwidth]{resources/chapter-3/class/user-class-diagram.jpg}
  \caption{\textit{User Class Diagram}}
  \label{fig:user-class-diagram}
\end{figure}

\subsubsection{Domain \textit{devices}}

Domain ini mengatur \textit{device} yang ada dalam sistem. Setiap \textit{device} memiliki ikatan dengan \textit{company}. Domain ini mengatur masalah CRUD dari satu \textit{company} yang dapat di \textit{manage} oleh banyak \textit{user}. Sama seperti domain lainnya, domain ini memiliki tiga lapisan yaitu \textit{handler, usecase, dan repository}. Ilustrasi \textit{class diagram} dapat dilihat pada gambar \ref{fig:device-class-diagram}.

\begin{figure}[ht]
  \centering
  \includegraphics[width=1\textwidth]{resources/chapter-3/class/device-class-diagram.jpg}
  \caption{\textit{Device Class Diagram}}
  \label{fig:device-class-diagram}
\end{figure}

\subsubsection{Domain \textit{groups}}

Domain ini mengatur \textit{groups} yang merupakan gabungan dari satu atau lebih \textit{device}. Seperti domain \textit{devices}, domain ini pun dapat dikelompokan berdasarkan \textit{company}. Seluruh \textit{user} dapat manage \textit{groups} selama masih dalam satu perusahaan yang sama. Domain ini juga memiliki tiga lapisan yaitu \textit{handler, usecase, dan repository}. Ilustrasi \textit{class diagram} dapat dilihat pada gambar \ref{fig:groups-class-diagram}.

\begin{figure}[ht]
  \centering
  \includegraphics[width=1\textwidth]{resources/chapter-3/class/groups-class-diagram.jpg}
  \caption{Groups \textit{Class Diagram}}
  \label{fig:groups-class-diagram}
\end{figure}

\pagebreak

\subsubsection{Domain \textit{deployment}}

Domain ini merupakan domain yang paling \textit{complex} pada service ini. Domain ini cukup luas karena berhubungan dengan \textit{deployment images} dan \textit{deployment history}. Selain itu domain ini juga memiliki hubungan dengan external service yaitu Kubernetes untuk proses deploymentnya. Ilustrasi \textit{class diagram} dapat dilihat pada gambar \ref{fig:deployment-class-diagram}

\begin{figure}[ht]
  \centering
  \includegraphics[width=1\textwidth]{resources/chapter-3/class/deployment-class-diagram.jpg}
  \caption{Deployment \textit{Class Diagram}}
  \label{fig:deployment-class-diagram}
\end{figure}

\subsubsection{Domain \textit{external services}}

Domain ini merupakan sebuah interface dari \textit{external service} yang digunakan oleh sistem. Terdapat dua external service yaitu \textit{database} dan Kubernetes. Namun, hanya \textit{kuberntes controller} saja yang dibuat \textit{class diagram} karena untuk \textit{database} itu sudah diimplementasikan di masing masing domain pada lapisan \textit{repository}. Ilustrasi \textit{class diagram} dapat dilihat pada gambar \ref{fig:kubernetes-controller-class-diagram}

\begin{figure}[ht]
  \centering
  \includegraphics[width=0.7\textwidth]{resources/chapter-3/class/kubernetes-controller}
  \caption{Kubernetes Controller \textit{Class Diagram}}
  \label{fig:kubernetes-controller-class-diagram}
\end{figure}
\chapter{Implementasi dan Pengujian}
Bab ini akan menjelaskan proses implementasi dari rancangan solusi yang telah dikaji pada Bab III. Setelah pembahasan terkait implementasi, akan dilanjutkan dengan pemaparan hasil uji terkait implementasi yang telah dibuat.

\section{Lingkungan}

\textit{Autoscaler} berbasis kontrol fleksibel akan diimplementasikan di lingkungan komputer lokal. Berikut adalah lingkungan perangkat keras dan perangkat lunak secara terperinci.

Implementasi sistem tugas akhir dilakukan dengan mengimplementasikan dengan bantuan beberapa kakas pada bahasa \textit{python}. Sistem akan hidup di luar \textit{kubernetes cluster} dan mengakses Kubernetes beserta \textit{pods}-nya melalui \textit{Kubernetes Client Library} dan \textit{service Elastic search} melalui \textit{web service} yang dapat diakses dari luar \textit{cluster}.

\textit{Autoscaler} berbasis kontrol fleksibel akan berjalan di \textit{cluster} Kubernetes lokal. Adapun spesifikasi dari komputer yang dipakai untuk pengembangan adalah sebagai berikut.
\begin{enumerate}
    \item \textbf{Perangkat Keras}
    
        \begin{enumerate}
            \item CPU: \textit{Apple M1 Chip}
            \item RAM: 16 GB
        \end{enumerate}
    
    \item \textbf{Perangkat Lunak}
        
        \begin{enumerate}
            \item Platform dan Sistem Operasi: Darwin AMD64, MacOS Monterey 12.6
            \item \textit{Containerization}: Docker
            \item \textit{Kubernetes Cluster}:
                \begin{enumerate}
                    \item Kubernetes Client v1.27.1-eks-2f008fe
                    \item Kubernetes Docker Desktop: Kubernetes v1.25.9
                \end{enumerate}
            \item Bahasa: Python 3.9.12
            \item Dependensi Lain:
                \begin{enumerate}
                    \item \textit{Kubernetes Client Library}
                    \item \textit{Pandas, numpy, statsmodels dan pmdarima}
                    \item \textit{Pickle}
                \end{enumerate}
        \end{enumerate}
\end{enumerate}

\section{Implementasi}

Bagian ini menjelaskan tentang implementasi sistem \textit{remote deployment} secara terperinci. Seperti yang telah dijelaskan pada bagian \ref{sec:rancangan-dashboard} dan \ref{sec:rancangan-service} terdapat dua komponen utama yaitu \textit{dashboard} dan \textit{service}. Penjelasan bagian ini dimulai dari batasan implementasi, dilanjutkan dengan kakas yang digunakan dalam proses pembuatan sistem dan diakhiri dengan penjelasan mengenai implementasi dari \textit{dashboard} dan \textit{service}

\subsection{Batasan Implementasi}
Berikut adalah batasan yang ditetapkan dalam melakukan implementasi \textit{sistem remote deployment}.

\begin{enumerate}
  \item Semua batasan masalah dan konfigurasi yang telah dibahas pada bagian \ref{sec:batasan-masalah}.
  \item Kubernetes cluster berjalan di lokal dengan menggunakan kakas \textit{kind} dan hanya dibatasi menjadi 4 node dengan spesifikasi \textit{1 master} dan \textit{3 worker}
  \item \textit{Device} sudah terkoneksi sebelumnya sehingga tidak perlu register \textit{device} dan menghubungkannya ke dalam \textit{cluster}.
  \item \textit{Dashboard} hanya memilki fungsionalitas untuk \textit{user}
\end{enumerate}

\subsection{Kakas yang Digunakan}
Dalam melakukan implementasi ini diperlukan beberapa kakas, diantaranya adalah sebagai berikut.
\begin{enumerate}
  \item \textit{Docker}, \textit{Docker Desktop} dan \textit{Docker Desktop Kubernetes} untuk dipakai sebagai \textit{containerization} dan \textit{cluster} kubernetes lokal.
  \item Pandas dan Numpy untuk keperluan \textit{data processing} serta bentuk data untuk dikirimkan ke komponen lain serta model prediksi ARIMA.
  \item \textit{Kubernetes Python Client} untuk mengontrol \textit{cluster} kubernetes melalui kode Python.
  \item \textit{Pickle} untuk menyimpan model ARIMA sehingga persisten meskipun sistem di-\textit{restart}.
  \item \textit{Statsmodels} dan \textit{pmdarima} untuk membangun model ARIMA serta melakukan otomasi pencarian orde atau lebih dikenal sebagai Auto-ARIMA.
\end{enumerate}

\subsection{Persiapan \textit{kubernetes cluster}}

Tahapan ini merupakan tahapan persiaspan sebelum proses \textit{development}. Pada tahapan ini akan dibuat kubernetes \textit{cluster} pada komputer lokal dengan kakas \textit{kind}. \textit{Cluster} yang dibuat akan memiliki 4 nodes dengan spesifikasi 1 \textit{master node} dan 3 \textit{worker node}. Digunakan \textit{command} \textbf{kind create cluster --config cluster.yaml} dengan file konfigurasi yang dapat dilihat dibawah ini.

\begin{figure}[h]
  \centering
  \includegraphics[width=1\textwidth]{resources/appendix/pembuatan-cluster.jpg}
  \caption{konfigurasi pembuatan \textit{cluster} dengan kakas \textit{kind}}
  \label{fig:konfigurasi-pembuatan-cluster}
\end{figure}

\begin{figure}[h]
  \centering
  \includegraphics[width=1\textwidth]{resources/chapter-4/cluster-kind.jpg}
  \caption{Hasil \textit{cluster} dengan kakas \textit{kind}}
  \label{fig:hasil-cluster-kind}
\end{figure}

\pagebreak

\subsection{Implementasi \textit{dashboard}}
Penjelasan dashboard lalallaa

\subsubsection{Halaman \textit{Login}}
penjelasan halaman login

\begin{figure}[ht]
  \centering
  \includegraphics[width=1\textwidth]{resources/chapter-4/dashboard/login-page.jpg}
  \caption{Halaman Login}
  \label{fig:halaman-login}
\end{figure}

\subsubsection{Halaman utama}
penjelasan halaman utama

\subsubsection{Halaman \textit{Account}}
penjelasan halaman account

\begin{figure}
  \centering
  \includegraphics[width=1\textwidth]{resources/chapter-4/dashboard/account-page.jpg}
  \caption{Halaman \textit{account}}
  \label{fig:halaman-account}
\end{figure}

\subsubsection{Halaman \textit{Device}}

\begin{figure}
  \centering
  \includegraphics[width=1\textwidth]{resources/chapter-4/dashboard/device-page.jpg}
  \caption{Halaman \textit{device}}
  \label{fig:halaman-device}
\end{figure}

\begin{figure}
  \centering
  \includegraphics[width=1\textwidth]{resources/chapter-4/dashboard/device-page-add.jpg}
  \caption{Modal menambahkan \textit{device}}
  \label{fig:halaman-device-add}
\end{figure}

\subsubsection{Halaman \textit{Device detail}}

\begin{figure}
  \centering
  \includegraphics[width=1\textwidth]{resources/chapter-4/dashboard/device-detail-page.jpg}
  \caption{Halaman \textit{device detail}}
  \label{fig:halaman-device-detail}
\end{figure}

\begin{figure}
  \centering
  \includegraphics[width=1\textwidth]{resources/chapter-4/dashboard/device-detail-add-group.jpg}
  \caption{Modal menambahkan group pada \textit{device detail}}
  \label{fig:halaman-device-detail-add-group}
\end{figure}

\begin{figure}
  \centering
  \includegraphics[width=1\textwidth]{resources/chapter-4/dashboard/device-detail-delete.jpg}
  \caption{Modal menghapus device pada halaman \textit{device detail}}
  \label{fig:halaman-device-detail-delete}
\end{figure}

\pagebreak

\subsubsection{Halaman \textit{Groups}}

\begin{figure}
  \centering
  \includegraphics[width=1\textwidth]{resources/chapter-4/dashboard/groups-page.jpg}
  \caption{Halaman \textit{groups}}
  \label{fig:halaman-groups}
\end{figure}

\begin{figure}
  \centering
  \includegraphics[width=1\textwidth]{resources/chapter-4/dashboard/groups-page-add.jpg}
  \caption{Modal menambahkan \textit{groups}}
  \label{fig:halaman-groups-add}
\end{figure}

\pagebreak

\subsubsection{Halaman \textit{Groups detail}}

\begin{figure}
  \centering
  \includegraphics[width=1\textwidth]{resources/chapter-4/dashboard/groups-detail-page.jpg}
  \caption{Halaman \textit{groups detail}}
  \label{fig:halaman-groups-detail}
\end{figure}

\begin{figure}
  \centering
  \includegraphics[width=1\textwidth]{resources/chapter-4/dashboard/groups-detail-add-device.jpg}
  \caption{Modal menambahkan group pada \textit{groups detail}}
  \label{fig:halaman-groups-detail-add-group}
\end{figure}

\begin{figure}
  \centering
  \includegraphics[width=1\textwidth]{resources/chapter-4/dashboard/groups-detail-delete.jpg}
  \caption{Modal menghapus groups pada halaman \textit{groups detail}}
  \label{fig:halaman-groups-detail-delete}
\end{figure}

\pagebreak

\subsubsection{Halaman \textit{Deployment}}

\begin{figure}
  \centering
  \includegraphics[width=1\textwidth]{resources/chapter-4/dashboard/deployment-page.jpg}
  \caption{Halaman \textit{deployment}}
  \label{fig:halaman-deployment}
\end{figure}


\begin{figure}
  \centering
  \includegraphics[width=1\textwidth]{resources/chapter-4/dashboard/deployment-page-add-deployment.jpg}
  \caption{Modal menambahkan \textit{deployment}}
  \label{fig:halaman-deployment-add-deployment}
\end{figure}

\begin{figure}
  \centering
  \includegraphics[width=1\textwidth]{resources/chapter-4/dashboard/deployment-page-add-repostory.jpg}
  \caption{Modal menambahkan \textit{image} pada halaman \textit{deployment}}
  \label{fig:halaman-deployment-add-repostory}
\end{figure}

\pagebreak

\subsubsection{Halaman \textit{deployments detail}}

\begin{figure}
  \centering
  \includegraphics[width=1\textwidth]{resources/chapter-4/dashboard/deployment-detail-page.jpg}
  \caption{Halaman \textit{deployment detail}}
  \label{fig:halaman-deployment-detail}
\end{figure}

\begin{figure}
  \centering
  \includegraphics[width=1\textwidth]{resources/chapter-4/dashboard/deployment-detail-delete.jpg}
  \caption{Modal menghapus deployment pada halaman \textit{deployment detail}}
  \label{fig:halaman-deployment-detail-delete}
\end{figure}

\pagebreak

\subsubsection{Halaman \textit{history detail}}
Penjelasan halaman history detail

\subsubsection{Halaman \textit{FAQ}}
Penjelasan halaman faq

\pagebreak

\subsection{Implementasi \textit{Service}}

Implementasi \textit{service} dibuat dengan menggunakan bahasa pemrogramman golang dan framework \textit{Echo}. Arsitektur kode yang dibuat memiliki tiga lapisan dimulai dari \textit{handler}, \textit{usecase}, dan \textit{repository}. Handler bertujuan membaca permintaan pengguna dan dapat disebut sebagai entrypoint. Data dari handler akan diberikan kepada \textit{usecase} untuk diproses. \textit{Usecase} merupakan lapisan yang hanya memiliki \textit{logic} proses bisnis. Setelah data berhasil melewati lapisan \textit{usecase}, data siap untuk dimasukkan ke database. Proses hubungan antara \textit{service} dengan \textit{database} diletakan pada lapisan \textit{repository}. 

Pemisahan lapisan ini mengikuti design pattern yaitu \textit{dependency injection}. Selain itu, pemisahan ini juga bertujuan memudahkan testing dan meningkatkan \textit{maintanability} karena mudah untuk dibaca dan dipahami.

\subsubsection{Domain \textit{company}}
\subsubsection{Domain \textit{user}}
\subsubsection{Domain \textit{devices}}
\subsubsection{Domain \textit{groups}}
\subsubsection{Domain \textit{deployment}}
\subsubsection{Domain \textit{external services}}

\pagebreak

\section{Pengujian}


Tujuan dari pengujian ialah untuk memastikan apakah seluruh kebutuhan fungsional dari sistem telah terpenuhi.
Pengujian dibagi menjadi dua bagian yaitu pengujian yang dilakukan per komponen lalu dilanjutkan dengan pengujian sistem. Setiap skenario pengujian dijelaskan tujuannya, skenario yang dilakukan, dan hasil pengujian yang didapatkan.

\subsection{Batasan Pengujian}
Berikut adalah batasan yang ditetapkan dalam melakukan pengujian \textit{sistem remote deployment}.

\begin{enumerate}
  \item Pengujian dilakukan di tiga kluster yang berbeda
        \begin{enumerate}
          \item Kubernetes lokal \textit{cluster} dengan jumlah 4 nodes
          \item \textit{Google Cloud Platform Compute Engine} Kubernetes \textit{cluster} dengan jumlah 2 nodes
          \item RaspberryPi Cluster dengan jumlah 2 nodes
        \end{enumerate}
  \item Sistem \textit{remote deployment} dijalankan pada komputer lokal yang memilki spesifikasi yang telah dijelaskan pada bagian \ref{sec:lingkungan-implementasi}.
  \item Untuk beberapa fungsionalitas admin digunakan \textit{HTTP Client} yaitu Postman untuk membuat \textit{request} kepada \textit{service}
  \item Cluster sudah tersedia dan siap diakses
  \item Setiap \textit{request} akan memiliki header X-Api-Key.
  \item Setiap \textit{request} yang mengarah ke /admin-api/ akan memiliki \textit{header} berupa X-Admin-API-Key.
  \item Database sudah terisi sebagian untuk memudahkan proses pengujian
\end{enumerate}

\subsection{Persiapan Pengujian}
Pada proses pengujian, terdapat tiga lingkungan pengujian yang digunakan untuk menguji sistem \textit{remote deployment}. Ketiga lingkungan tersebut yaitu kubernetes \textit{cluster} lokal, kubernetes \textit{cluster} yang terdapat di \textit{Cloud (GCP)}, serta \textit{cluster} pada RaspberryPi. Ketiga \textit{cluster} ini memiliki jumlah node yang berbeda sesuai dengan penjelasan pada \ref{subsec:batasan-pengujian}.

\subsubsection{Kubernetes Lokal}
Untuk pengujian pada kubernetes lokal, dilakukan pembuatan \textit{cluster} dengan kakas kind untuk membuat cluster yang bernama \textit{testing-cluster-two-nodes} yang memiliki 2 node. Konfigurasi pembuatan sama seperti konfigurasi pada bagian \ref{subsec:persiapan-kubernetes-cluster} hanya saja jumlah nodes yang digunakan yaitu 2.
Karena nodes berjumlah dua maka terdapat 1 \textit{master nodes} dan 1 \textit{slave} node pada \textit{cluster}. Konfigurasi pembuatan cluster dapat dilihat pada gambar \ref{fig:kubernetes-lokal-config-testing}.

\begin{figure}[ht]
  \centering
  \includegraphics[width=1\textwidth]{resources/chapter-4/pengujian/kubernetes-lokal-config.jpg}
  \caption{Konfigurasi Pembuatan \textit{Kubernetes Testing Cluster} Dengan Kakas \textit{Kind}}
  \label{fig:kubernetes-lokal-config-testing}
\end{figure}

Setelah itu jalankan perintah "kind create cluster --config testing-cluster.yaml" untuk membuat \textit{cluster} pada \textit{docker}. Hasil dari perintahh ini yaitu tercipta dua buah \textit{container} pada \textit{docker} yang memiliki peran \textit{master} dan \textit{slave} seperti pada gambar \ref{fig:kubernetes-lokal-config-testing-result}.

\begin{figure}[ht]
  \centering
  \includegraphics[width=1\textwidth]{resources/chapter-4/pengujian/kubernetes-lokal-config-result.jpg}
  \caption{Hasil \textit{Kubernetes Testing Cluster} pada \textit{Docker}}
  \label{fig:kubernetes-lokal-config-testing-result}
\end{figure}

\subsubsection{Kubernetes GCP}
\label{subsubsec:kubernetes-gcp}
Pada lingkungan ini dibuat dua buah \textit{compute engine (virtual machine)} pada \textit{GCP}. Masing masing dari \textit{virtual machine} akan berperan sebagai kubernetes \textit{cluster} yang bernama \textit{prod-cluster-example}.

\begin{enumerate}
  \item Buat dua \textit{virtual machine} pada \textit{compute engine GCP} dengan spesifikasi berikut. Hasil pembuatan \textit{virtual machine} dapat dilihat pada gambar \ref{fig:hasil-pembuatan-virtual-machine-gcp}
        \begin{enumerate}
          \item Ubuntu 24.04
          \item 2GB Memory
          \item 10GB Storage Persistent Disk
          \item 0.5 - 2Vcpu (1 shared core)
          \item Region Asia southeast2-c
        \end{enumerate}
  \item Membuat \textit{Firewall Rule} Untuk membuka port yang digunakan oleh \textit{kubernetes}. Untuk daftar opsi setiap port yang dibuka dapat dilihat pada gambar \ref{fig:daftar-kegunaan-port}. Untuk hasil pembuatan firewall dapat dilihat pada bagian \ref{fig:hasil-firewall-rule-pada-gcp}
  \item Konfigurasi \textit{gare-test-kubernetes-server} sebagai master nodes. Konfigurasi dilakukan dengan cara mengunduh instalasi dari k3s dengan perintah seperti pada gambar \ref{fig:instalasi-master-node-gcp}.
  \item Konfigurasi virutal machine lainnya yaitu \textit{gare-test-kubernetes-server-node} sebagai \textit{worker node}. Untuk meregistrasi \textit{node} ke dalam \textit{cluster} perlu adanya autentikasi untuk memasitikan hanya \textit{node} yang benar yang boleh masuk ke dalam \textit{cluster}. K3s memiliki token generator yang dapat digunakan untuk mencegah akses yang tidak diinginkan, registrasi token dapat dilihat pada gambar \ref{fig:pengambilan-token-registrasi-cluster}. Token tersebut akan digunakan untuk meregistrasi \textit{node} ini ke \textit{master} dengan \textit{public ip} node tersebut. Ilustrasi dapat dilihat pada gambar \ref{fig:instalasi-worker-node-gcp}.
  \item Ambil konfigurasi \textit{cluster} di \textit{master node} dan pindahkan ke lokasi \textit{server berjalan} untuk meregistrasi \textit{cluster} ke dalam sistem. Setelah melakukan kelima langkah ini \textit{cluster} sudah terintegrasi dengan sistem. Ilustrasi pemindahan konfigurasi dapat dilihat pada gambar \ref{fig:konfigurasi-cluster-master-node-gcp} dan \ref{fig:proses-pemindahan-konfigurasi-master-gcp}.
\end{enumerate}

\subsubsection{Kubernetes RaspberryPi}
Pada lingkungan ini dibuat cluster dengan dua nodes pada RaspberryPi. Cluster yang dibuat bernama cluster-raspi yang memiliki spesifikasi hardware RaspberryPi berikut.

\begin{enumerate}
  \item Master node menggunakan Raspberry Pi 3 Model B Rev 1.2 dengan 1GB RAM dan 4 CPU @ 1.2GHz dengan hostname masterpi. Informasi lebih lengkap dapat dilihat pada gambar \ref{fig:hostname-raspi-master-nodes} dan \ref{fig:spesifikasi-raspi-master-nodes}
  \item Worker node menggunakan Raspberry Pi 2 Model B Rev 1.1 dengan 1GB RAM dan 4 CPU @ 900MHz dengan hostname raspberrypi. Informasi lebih lengkap dapat dilihat pada gambar \ref{fig:hostname-raspi-worker-nodes} dan \ref{fig:spesifikasi-raspi-worker-nodes}
\end{enumerate}

Berikut merupakan tata cara pembuatan cluster kubernetes pada \textit{RaspberryPi}. Diasumsikan \textit{device} \textit{RaspberryPi} sudah terhubung ke dalam jaringan yang sama sehingga tidak perlu \textit{port forwarding} / \textit{public ip}.

\begin{enumerate}
  \item Konfigurasi \textit{hostname masterpi} sebagai master nodes. Perlu dilakukan konfigurasi tambahan untuk menambahkan \textit{cgroups} pada raspberrypi karena secara default opsi ini \textit{disabled}. \textit{Cgroups} merupakan kepanjangan dari \textit{Control Groups} yang berfungsi sebagai \textit{resource management} pada linux dan digunakan dalam proses kontainerisasi. Selanjutnya mirip serperti konfigurasi pada bagian \ref{subsubsec:kubernetes-gcp}, perlu mengunduh instalasi dari k3s dengan perintah seperti pada gambar \ref{fig:instalasi-master-raspi-nodes}.
  \item Konfigurasi \textit{node} lainnya yaitu \textit{hostname raspberrypi} sebagai \textit{worker node}. Untuk meregistrasi \textit{node} ke dalam \textit{cluster} perlu adanya autentikasi untuk memasitikan hanya \textit{node} yang benar yang boleh masuk ke dalam \textit{cluster}. K3s memiliki token generator yang dapat digunakan untuk mencegah akses yang tidak diinginkan, registrasi token dapat dilihat pada gambar \ref{fig:raspi-master-gen-token}. Token tersebut akan digunakan untuk meregistrasi \textit{node} ini ke \textit{master} dengan \textit{ip local} node master. Ilustrasi dapat dilihat pada gambar \ref{fig:instalasi-worker-raspi-node}.
  \item Ambil konfigurasi \textit{cluster} di \textit{master node} dan pindahkan ke lokasi \textit{server berjalan} untuk meregistrasi \textit{cluster} ke dalam sistem. Setelah melakukan langkah ini \textit{cluster} sudah terintegrasi dengan sistem. Ilustrasi pemindahan konfigurasi dapat dilihat pada gambar \ref{fig:raspi-kube-config} dan \ref{fig:raspi-add-kubeconfig}.
\end{enumerate}

\subsection{Pengujian Komponen}
Pengujian di level komponen memastikan bahwa seluruh fungsionalitas yang tidak melibatkan servis eksternal di dalam komponen bekerja dengan baik. Pengujian ini akan dibagi menjadi beberapa bagian sesuai dengan domain yang telah dijelaskan pada bagian \ref{subsec:implementasi-service}. Pada masing masing domain terdapat tabel yang memetakan hubungan antara kebutuhan fungsional serta pengujian yang bersesuaian.

\input{chapters/chapter-4/pengujian/02-01-domain-company.tex}
\input{chapters/chapter-4/pengujian/02-02-domain-user.tex}
\input{chapters/chapter-4/pengujian/02-03-domain-devices.tex}
\input{chapters/chapter-4/pengujian/02-04-domain-groups.tex}
\input{chapters/chapter-4/pengujian/02-05-domain-deployment.tex}
\input{chapters/chapter-4/pengujian/02-06-domain-external.tex}

\subsection{Pengujian Sistem}

\input{chapters/chapter-4/pengujian/03-01-pengujian-sistem-raspi.tex}


\chapter{Penutup}

Bab Kesimpulan dan Saran akan menjadi bagian akhir dan penutup dari penelitian tugas akhir ini. Bab ini akan membahas kesimpulan yang berisi ketercapaian tujuan penelitian tugas akhir dengan permasalahan yang diselesaikan dalam penelitian tugas akhir. Selain itu, bab ini akan membahas saran yang dapat dilakukan untuk pengembangan atau penelitian selanjutnya.

\section{Kesimpulan}
Penelitian tugas akhir ini mengimplementasikan cara untuk melakukan \textit{remote deployment} pada lingkungan IoT dengan menggunakan kubernetes. Setelah dilakukan analisis, implementasi, dan pengujian, dapat diambil kesimpulan sebagai berikut.
\begin{enumerate}
  \item  Penelitian ini berhasil merancang dan mengimplementasikan arsitektur sistem \textit{remote deployment} yang bersifat \textit{platform agnostic} menggunakan Kubernetes. Sistem ini memungkinkan untuk melakukan proses \textit{deployment} dengan metode tanpa kabel (non-serial) terbukti dari implementasi pada Raspberry Pi dengan dua node serta pada lingkungan \textit{virtual machine} dengan konfigurasi node yang sama. Pengujian fungsional dan \textit{end-to-end} mengkonfirmasi bahwa sistem berjalan dengan baik dan memenuhi semua kebutuhan fungsional yang telah didefinisikan.
  \item Sistem \textit{remote deployment} berbasis Kubernetes dapat beroperasi dengan baik pada perangkat dengan keterbatasan sumber daya hardware, seperti Raspberry Pi. Implementasi dengan menggunakan K3s, distribusi Kubernetes yang ringan, memungkinkan deployment pada perangkat low-resource, membuktikan bahwa solusi ini dapat diadaptasi untuk lingkungan IoT dengan sumber daya terbatas.
\end{enumerate}

\section{Saran}
Adapun banyak kekurangan dan kelemahan yang ditemukan dalam penelitian tugas akhir ini. Berikut adalah beberapa saran yang dapat dilakukan untuk pengembangan atau penelitian selanjutnya.
\begin{enumerate}
  \item Dapat diimplementasikan versioning pada deployment ketika melakukan \textit{rollback}. Ketika versi yang dicapai sudah tidak lagi ditemukan barulah \textit{deployment} di hapus dari \textit{cluster}.
  \item Dapat diimplementasikan	extit{dashboard}untuk \textit{admin} agar memudahkan proses manajemen \textit{company} dan \textit{user}.
  \item Dapat membuat proses \textit{remote deployment} lebih terkustomisasi dengan cara membuat \textit{deployment plan} lebih \textit{flexible} lagi.
  \item Menambahkan fitur seperti \textit{remote command} untuk melakukan eksekusi \textit{command} di setiap perangkat tanpa harus melakukannya satu persatu
  \item Proses registrasi \textit{node} pada \textit{cluster} masih dilakukan secara manual, dapat digunakan sistem \textit{device discovery} untuk mengeleminasi proses yang redundan.
  \item Sistem dapat dibuat sebagai \textit{microservice} untuk menghindari \textit{single point of failure}
  \item Pembuatan \textit{swagger} dalam membuat \textit{service} akan membantu dokumentasi serta memudahkan proses \textit{testing}.
\end{enumerate}
%---------------------------------------------------------------%

% Daftar pustaka
\printbibliography

% Setting judul lampiran
\titlespacing*{\chapter}{0pt}{0pt}{0pt}
\titlespacing*{\section}{0pt}{0pt}{*1}

% Setting judul anak lampiran
\titleformat*{\section}{\bfseries}

\appendix

\chapter{Arsitektur China Highways}

\begin{figure}[ht]
  \centering
  \includegraphics[width=0.8\textwidth]{resources/chapter-2/china-highways.jpg}
  \caption{Implementasi Sistem \textit{ETC} di China \parencite{penelitianterkait1}}
  \label{fig:china-highways}
\end{figure}

\begin{figure}[ht]
  \includegraphics[width=0.8\textwidth]{resources/chapter-2/arsitektur-china-highways.jpg}
  \caption{Arsitektur Sistem \textit{ETC} di China \parencite{penelitianterkait1}}
  \label{fig:architecture-china-highways}
\end{figure}

\chapter{Tabel Kebutuhan}

\bgroup
\begin{table}[ht]
  \def\arraystretch{1.5}
  \caption{Kebutuhan Fungsional}
  \label{tab:kebutuhan-fungsional}
  \centering
  \begin{tabular}{|c|p{12cm}|}
    \hline
    ID  & Penjelasan                                                                                                      \\
    \hline
    F01 & Admin dapat melakukan registrasi perusahaan baru ke dalam database sistem.                                      \\
    \hline
    F02 & Admin dapat melihat seluruh perushaan yang ada pada database sistem.                                            \\
    \hline
    F03 & \textit{User} dapat melihat detail dari perushaan yang ada pada database sistem.                                \\
    \hline
    F04 & \textit{User} dapat melihat \textit{User} lain pada satu perusahaan yang sama                                   \\
    \hline
    F05 & Admin dapat melakukan registrasi \textit{User} baru ke sebuah perusahaan pada sistem.                           \\
    \hline
    F06 & Admin dapat menghapus \textit{User} dari suatu perusahaan                                                       \\
    \hline
    F07 & \textit{User} dapat masuk ke dalam sistem dengan memasukan kredensial yang diberikan.                           \\
    \hline
    F08 & \textit{User} dapat \textit{logout} dari sistem.                                                                \\
    \hline
    F09 & \textit{User} dapat melakukan registrasi perangkat ke database sistem                                           \\
    \hline
    F10 & \textit{User} dapat melihat seluruh perangkat yang telah didaftarkan pada perushaannya                          \\
    \hline
    F11 & \textit{User} dapat menghapus perangkat yang telah didaftarkan di database                                      \\
    \hline
    F12 & \textit{User} dapat membuat \textit{groups} untuk mengelompokan beberapa perangkat                              \\
    \hline
    F13 & \textit{User} dapat melihat seluruh \textit{groups} yang telah terdaftar pada perusahannya                      \\
    \hline
    F14 & \textit{User} dapat menghapus \textit{groups} yang telah terdaftar pada perusahannya                            \\
    \hline
    F15 & \textit{User} dapat melihat hubungan antara perangkat dan \textit{groups} pada sistem dan begitupula sebaliknya \\
    \hline
    F16 & \textit{User} dapat membuat deployment images yang teraosiasi dengan perusahannya                               \\
    \hline
    F17 & \textit{User} dapat melihat deployment images yang teraosiasi dengan perusahannya                               \\
    \hline
  \end{tabular}
\end{table}
\egroup

\pagebreak

\bgroup
\begin{table}[ht]
  \def\arraystretch{1.7}
  \centering
  \begin{tabular}{|c|p{12cm}|}
    \hline
    ID  & Penjelasan                                                                                            \\

    \hline
    F18 & \textit{User} dapat menghapus deployment images yang teraosiasi dengan perusahannya                   \\
    \hline
    F19 & \textit{User} dapat membuat deployment plan yang teraosiasi dengan perushaannya                       \\
    \hline
    F20 & \textit{User} dapat melihat seluruh deployment plan yang terdaftar dan teraosiasi dengan perushaannya \\
    \hline
    F21 & \textit{User} dapat menghapus deployment plan yang terdaftar dan teraosiasi dengan perushaannya       \\
    \hline
    F22 & \textit{User} dapat melakukan deployment ke perangkat ataupun group dengan deployment plan pilihannya \\
    \hline
    F23 & \textit{User} dapat melihat riwayat dari deployment yang telah dilakukan ke perangkat ataupun group   \\
    \hline
  \end{tabular}
\end{table}
\egroup

\pagebreak

\bgroup
\begin{table}[ht]
  \def\arraystretch{1.7}
  \caption{Tabel Use case}
  \label{tab:penjelasan-usecase-diagram}
  \centering
  \begin{tabular}{|c|p{4cm}|p{8cm}|}
    \hline
    ID   & Use case                                   & Deskripsi                                                                                                                                  \\
    \hline
    UC01 & Mendaftarkan perusahaan                    & Sistem memberikan akses kepada admin untuk mendaftarkan perusahaan yang ingin mendaftar ke dalam sistem                                    \\
    \hline
    UC02 & Mendaftarkan \textit{user}                 & Sistem memberikan akses kepada admin untuk mendaftarkan \textit{user} ke perusahaan tertentu                                               \\
    \hline
    UC03 & Manajemen perusahaan                       & Sistem memberikan akses kepada admin untuk melakukan manajemen terhadap seluruh perusahaan yang terdaftar pada sistem                      \\
    \hline
    UC04 & Manajemen \textit{user}                    & Sistem memberikan akses kepada admin untuk melakukan manajemen terhadap seluruh \textit{user} yang terdaftar pada sistem                   \\
    \hline
    UC05 & Login                                      & Sistem memberikan akses kepada \textit{user}                                                                                               \\
    \hline
    UC06 & Melihat detail perusahaan                  & Sistem memberikan akses kepada \textit{user} untuk melihat perusahaanya                                                                    \\
    \hline
    UC07 & Melihat \textit{user} pada satu perusahaan & Sistem memberikan akses kepada \textit{user} user lainnya pada satu perusahaan                                                             \\

    \hline
    UC08 & Manajemen \textit{perangkat}               & Sistem memberikan akses kepada \textit{user} untuk melihat, membuat, serta menghapus \textit{perangkat} yang terdaftar pada sistem         \\
    \hline
    UC09 & Manajemen \textit{groups}                  & Sistem memberikan akses kepada \textit{user} untuk melihat, membuat, serta menghapus \textit{groups} yang terdaftar pada sistem            \\
    \hline
    UC10 & Manajemen \textit{deployment images}       & Sistem memberikan akses kepada \textit{user} untuk melihat, membuat, serta menghapus \textit{deployment images} yang terdaftar pada sistem \\
    \hline
    UC11 & Manajemen \textit{deployment plan}         & Sistem memberikan akses kepada \textit{user} untuk melihat, membuat, serta menghapus \textit{deployment plan} yang terdaftar pada sistem   \\
    \hline
    UC12 & Melakukan \textit{Remote deployment}       & Sistem memberikan akses kepada \textit{user} untuk melakukan \textit{deployment} kepada target perangakt ataupun \textit{groups}           \\
    \hline
    UC13 & Melihat riwayat \textit{deployment}        & Sistem memberikan akses kepada \textit{user} untuk melihat riwayat \textit{deployment} yang telah dilakukan                                \\
    \hline
  \end{tabular}
\end{table}
\egroup

\chapter{Tampilan halaman dashboard}
\label{appendix:tampilan-halaman-dashboard}

\begin{figure}[h]
  \centering
  \includegraphics[width=1\textwidth]{resources/chapter-4/dashboard/login-page.jpg}
  \caption{Halaman Login}
  \label{fig:halaman-login}
\end{figure}

\begin{figure}[h]
  \centering
  \includegraphics[width=1\textwidth]{resources/chapter-4/dashboard/account-page.jpg}
  \caption{Halaman \textit{Account}}
  \label{fig:halaman-account}
\end{figure}

\begin{figure}[h]
  \centering
  \includegraphics[width=1\textwidth]{resources/chapter-4/dashboard/device-page.jpg}
  \caption{Halaman \textit{Device}}
  \label{fig:halaman-device}
\end{figure}

\begin{figure}[h]
  \centering
  \includegraphics[width=1\textwidth]{resources/chapter-4/dashboard/device-page-actions.jpg}
  \caption{Actions pada Tabel \textit{Device}}
  \label{fig:halaman-device-actions}
\end{figure}

\begin{figure}[h]
  \centering
  \includegraphics[width=1\textwidth]{resources/chapter-4/dashboard/device-page-add.jpg}
  \caption{Modal Menambahkan \textit{Device}}
  \label{fig:halaman-device-add}
\end{figure}

\begin{figure}[h]
  \centering
  \includegraphics[width=1\textwidth]{resources/chapter-4/dashboard/device-detail-page.jpg}
  \caption{Halaman \textit{Device Detail}}
  \label{fig:halaman-device-detail}
\end{figure}

\begin{figure}[h]
  \centering
  \includegraphics[width=1\textwidth]{resources/chapter-4/dashboard/device-detail-add-group.jpg}
  \caption{Modal Menambahkan \textit{Group} pada \textit{Device Detail}}
  \label{fig:halaman-device-detail-add-group}
\end{figure}

\begin{figure}[h]
  \centering
  \includegraphics[width=1\textwidth]{resources/chapter-4/dashboard/device-detail-delete.jpg}
  \caption{Modal Menghapus Device pada Halaman \textit{Device Detail}}
  \label{fig:halaman-device-detail-delete}
\end{figure}

\begin{figure}[h]
  \centering
  \includegraphics[width=1\textwidth]{resources/chapter-4/dashboard/groups-page.jpg}
  \caption{Halaman \textit{Groups}}
  \label{fig:halaman-groups}
\end{figure}

\begin{figure}[h]
  \centering
  \includegraphics[width=1\textwidth]{resources/chapter-4/dashboard/groups-page-actions.jpg}
  \caption{\textit{Actions} pada Tabel \textit{Groups}}
  \label{fig:halaman-groups-actions}
\end{figure}

\begin{figure}[h]
  \centering
  \includegraphics[width=1\textwidth]{resources/chapter-4/dashboard/groups-page-add.jpg}
  \caption{Modal Menambahkan \textit{groups}}
  \label{fig:halaman-groups-add}
\end{figure}

\begin{figure}[h]
  \centering
  \includegraphics[width=1\textwidth]{resources/chapter-4/dashboard/groups-detail-page.jpg}
  \caption{Halaman \textit{groups detail}}
  \label{fig:halaman-groups-detail}
\end{figure}

\begin{figure}[h]
  \centering
  \caption{Modal Menambahkan \textit{Group} pada \textit{Groups Detail}}
  \includegraphics[width=1\textwidth]{resources/chapter-4/dashboard/groups-detail-add-device.jpg}
  \label{fig:halaman-groups-detail-add-group}
\end{figure}

\begin{figure}[h]
  \centering
  \includegraphics[width=1\textwidth]{resources/chapter-4/dashboard/groups-detail-delete.jpg}
  \caption{Modal Menghapus \textit{Groups} pada Halaman \textit{Groups Detail}}
  \label{fig:halaman-groups-detail-delete}
\end{figure}

\begin{figure}[h]
  \centering
  \includegraphics[width=1\textwidth]{resources/chapter-4/dashboard/deployment-page.jpg}
  \caption{Halaman \textit{Deployment}}
  \label{fig:halaman-deployment}
\end{figure}

\begin{figure}[h]
  \centering
  \includegraphics[width=1\textwidth]{resources/chapter-4/dashboard/deployment-page-add-deployment.jpg}
  \caption{Modal Menambahkan \textit{Deployment}}
  \label{fig:halaman-deployment-add-deployment}
\end{figure}

\begin{figure}[h]
  \centering
  \includegraphics[width=1\textwidth]{resources/chapter-4/dashboard/deployment-page-add-repostory.jpg}
  \caption{Modal Menambahkan \textit{Image} Pada Halaman \textit{Deployment}}
  \label{fig:halaman-deployment-add-repostory}
\end{figure}

\begin{figure}[h]
  \centering
  \includegraphics[width=1\textwidth]{resources/chapter-4/dashboard/deployment-detail-page.jpg}
  \caption{Halaman \textit{deployment detail}}
  \label{fig:halaman-deployment-detail}
\end{figure}

\begin{figure}[h]
  \centering
  \includegraphics[width=1\textwidth]{resources/chapter-4/dashboard/deployment-detail-delete.jpg}
  \caption{Modal menghapus deployment pada halaman \textit{deployment detail}}
  \label{fig:halaman-deployment-detail-delete}
\end{figure}

\chapter{Pengujian Komponen}

\begin{figure}[ht]
  \centering
  \includegraphics[width=0.8\textwidth]{resources/chapter-4/pengujian/p00.jpg}
  \caption{Daftar Nama Cluster yang Tersedia pada Service}
  \label{fig:list-cluster-tersedia}
\end{figure}

\begin{figure}[ht]
  \centering
  \includegraphics[width=0.8\textwidth]{resources/chapter-4/pengujian/kube-gcp-01.jpg}
  \caption{Hasil Pembuatan Virtual Machine pada GCP}
  \label{fig:hasil-pembuatan-virtual-machine-gcp}
\end{figure}

\begin{figure}[ht]
  \centering
  \includegraphics[width=0.8\textwidth]{resources/chapter-4/pengujian/kube-gcp-03.jpg}
  \caption{Daftar Kegunaan Port}
  \label{fig:daftar-kegunaan-port}
\end{figure}

\begin{figure}[ht]
  \centering
  \includegraphics[width=0.8\textwidth]{resources/chapter-4/pengujian/kube-gcp-02.jpg}
  \caption{Hasil Firewall Rule pada GCP}
  \label{fig:hasil-firewall-rule-pada-gcp}
\end{figure}

\begin{figure}[ht]
  \centering
  \includegraphics[width=0.8\textwidth]{resources/chapter-4/pengujian/kube-gcp-04.jpg}
  \caption{Instalasi Master Node di \textit{Virtual Machine}}
  \label{fig:instalasi-master-node-gcp}
\end{figure}

\begin{figure}[ht]
  \centering
  \includegraphics[width=0.8\textwidth]{resources/chapter-4/pengujian/kube-gcp-05.jpg}
  \caption{Pengambilan Token Registrasi Cluster}
  \label{fig:pengambilan-token-registrasi-cluster}
\end{figure}

\begin{figure}[ht]
  \centering
  \includegraphics[width=0.8\textwidth]{resources/chapter-4/pengujian/kube-gcp-06.jpg}
  \caption{Instalasi Worker Node di \textit{Virtual Machine}}
  \label{fig:instalasi-worker-node-gcp}
\end{figure}

\begin{figure}[ht]
  \centering
  \includegraphics[width=0.8\textwidth]{resources/chapter-4/pengujian/kube-gcp-07.jpg}
  \caption{Konfigurasi \textit{Cluster} pada \textit{Master Node GCP}}
  \label{fig:konfigurasi-cluster-master-node-gcp}
\end{figure}

\begin{figure}[ht]
  \centering
  \includegraphics[width=0.8\textwidth]{resources/chapter-4/pengujian/kube-gcp-08.jpg}
  \caption{Pemindahan Konfigurasi \textit{Cluster} pada \textit{Master Node GCP}}
  \label{fig:proses-pemindahan-konfigurasi-master-gcp}
\end{figure}

\begin{figure}[ht]
  \centering
  \includegraphics[width=0.8\textwidth]{resources/chapter-4/pengujian/raspi-01.jpg}
  \caption{Hostname Raspsi Master Nodes}
  \label{fig:hostname-raspi-master-nodes}
\end{figure}

\begin{figure}[ht]
  \centering
  \includegraphics[width=0.8\textwidth]{resources/chapter-4/pengujian/raspi-master-neofetch.jpg}
  \caption{Spesifikasi Raspsi Master Nodes}
  \label{fig:spesifikasi-raspi-master-nodes}
\end{figure}

\begin{figure}[ht]
  \centering
  \includegraphics[width=0.8\textwidth]{resources/chapter-4/pengujian/raspi-01-worker.jpg}
  \caption{Hostname Raspsi Worker Nodes}
  \label{fig:hostname-raspi-worker-nodes}
\end{figure}

\begin{figure}[ht]
  \centering
  \includegraphics[width=0.8\textwidth]{resources/chapter-4/pengujian/raspi-worker-neofetch.jpg}
  \caption{Spesifikasi Raspsi Worker Nodes}
  \label{fig:spesifikasi-raspi-worker-nodes}
\end{figure}

\begin{figure}[ht]
  \centering
  \includegraphics[width=0.8\textwidth]{resources/chapter-4/pengujian/raspi-02-additional.jpg}
  \caption{\textit{Additional Command} untuk RaspberryPi}
  \label{fig:additional-command-raspberrypi}
\end{figure}

\begin{figure}[ht]
  \centering
  \includegraphics[width=0.8\textwidth]{resources/chapter-4/pengujian/raspi-02-additional-02.jpg}
  \caption{Penambahan cgroups pada \textit{Cmdline}}
  \label{fig:penambahan-cgroups-pada-cmdline}
\end{figure}

\begin{figure}[ht]
  \centering
  \includegraphics[width=0.8\textwidth]{resources/chapter-4/pengujian/raspi-03.jpg}
  \caption{Instalasi Master Raspi \textit{Node}}
  \label{fig:instalasi-master-raspi-nodes}
\end{figure}

\begin{figure}[ht]
  \centering
  \includegraphics[width=0.8\textwidth]{resources/chapter-4/pengujian/raspi-token-gen.jpg}
  \caption{Master Raspi Generate Token}
  \label{fig:raspi-master-gen-token}
\end{figure}

\begin{figure}[ht]
  \centering
  \includegraphics[width=0.8\textwidth]{resources/chapter-4/pengujian/raspi-04.jpg}
  \caption{Instalasi Worker Raspi \textit{Node}}
  \label{fig:instalasi-worker-raspi-node}
\end{figure}

\begin{figure}[ht]
  \centering
  \includegraphics[width=0.8\textwidth]{resources/chapter-4/pengujian/raspi-kube-config.jpg}
  \caption{Kubernetes Config RaspberryPi}
  \label{fig:raspi-kube-config}
\end{figure}

\begin{figure}[ht]
  \centering
  \includegraphics[width=0.8\textwidth]{resources/chapter-4/pengujian/raspi-kube-config.jpg}
  \caption{Menambagkan Kubernetes Config RaspberryPi pada Sistem}
  \label{fig:raspi-add-kubeconfig}
\end{figure}

\begin{figure}[ht]
  \centering
  \includegraphics[width=0.8\textwidth]{resources/chapter-4/pengujian/p01.jpg}
  \caption{\textit{Request dan Response Pengujian} P01}
  \label{fig:pengujian-p01}
\end{figure}

\begin{figure}[ht]
  \centering
  \includegraphics[width=0.8\textwidth]{resources/chapter-4/pengujian/p02.jpg}
  \caption{\textit{Request dan Response Pengujian} P02}
  \label{fig:pengujian-p02}
\end{figure}

\begin{figure}[ht]
  \centering
  \includegraphics[width=0.8\textwidth]{resources/chapter-4/pengujian/p03.jpg}
  \caption{\textit{Request dan Response Pengujian} P03}
  \label{fig:pengujian-p03}
\end{figure}

\begin{figure}[ht]
  \centering
  \includegraphics[width=0.8\textwidth]{resources/chapter-4/pengujian/p04-1.jpg}
  \caption{Pembuatan \textit{company} Untuk Pengujian P04}
  \label{fig:pengujian-p04-1}
\end{figure}

\begin{figure}[ht]
  \centering
  \includegraphics[width=0.8\textwidth]{resources/chapter-4/pengujian/p04.jpg}
  \caption{\textit{Request dan Response Pengujian} P04}
  \label{fig:pengujian-p04}
\end{figure}

\begin{figure}[ht]
  \centering
  \includegraphics[width=0.8\textwidth]{resources/chapter-4/pengujian/p05.jpg}
  \caption{\textit{Request dan Response Pengujian} P05}
  \label{fig:pengujian-p05}
\end{figure}

\begin{figure}[ht]
  \centering
  \includegraphics[width=0.8\textwidth]{resources/chapter-4/pengujian/p06.jpg}
  \caption{\textit{Request dan Response Pengujian} P06}
  \label{fig:pengujian-p06}
\end{figure}

\begin{figure}[ht]
  \centering
  \includegraphics[width=0.8\textwidth]{resources/chapter-4/pengujian/p07.jpg}
  \caption{Redirect ke \textit{Halaman Index} pada Pengujian P07}
  \label{fig:pengujian-p07}
\end{figure}

\begin{figure}[ht]
  \centering
  \includegraphics[width=0.8\textwidth]{resources/chapter-4/pengujian/p08.jpg}
  \caption{Muncul Modal Penanda Login Gagal}
  \label{fig:pengujian-p08}
\end{figure}

\begin{figure}[ht]
  \centering
  \includegraphics[width=0.8\textwidth]{resources/chapter-4/pengujian/p08-09.jpg}
  \caption{Mengunjungi Halaman Account pada Pengujian P08 dan P09}
  \label{fig:pengujian-p08-09}
\end{figure}

\begin{figure}[ht]
  \centering
  \includegraphics[width=0.8\textwidth]{resources/chapter-4/pengujian/p12-success.jpg}
  \caption{Pembuatan \textit{Device} untuk Pengujian P12}
  \label{fig:pengujian-p12-success}
\end{figure}

\begin{figure}[ht]
  \centering
  \includegraphics[width=0.8\textwidth]{resources/chapter-4/pengujian/p12-failed.jpg}
  \caption{Pembuatan \textit{Device} untuk Pengujian P12}
  \label{fig:pengujian-p12-failed}
\end{figure}

\begin{figure}[ht]
  \centering
  \includegraphics[width=0.8\textwidth]{resources/chapter-4/pengujian/p14.jpg}
  \caption{Hasil Penghapusan \textit{Device} untuk Pengujian P15}
  \label{fig:pengujian-p15}
\end{figure}

\begin{figure}[ht]
  \centering
  \includegraphics[width=0.8\textwidth]{resources/chapter-4/pengujian/p16.jpg}
  \caption{Hasil Pembuatan \textit{Group} untuk Pengujian P16}
  \label{fig:pengujian-p16}
\end{figure}

\begin{figure}[ht]
  \centering
  \includegraphics[width=0.8\textwidth]{resources/chapter-4/pengujian/p17.jpg}
  \caption{Error ketika Membuat \textit{Group} baru untuk Pengujian P17}
  \label{fig:pengujian-p17}
\end{figure}

\begin{figure}[ht]
  \centering
  \includegraphics[width=0.8\textwidth]{resources/chapter-4/pengujian/p19.jpg}
  \caption{Hasil Penghapusan \textit{Group} untuk Pengujian P19}
  \label{fig:pengujian-p19}
\end{figure}

\begin{figure}[ht]
  \centering
  \includegraphics[width=0.8\textwidth]{resources/chapter-4/pengujian/pengujian-nonfungsional-1.jpg}
  \caption{Mengakses \textit{Dashboard} melalui mobile}
  \label{fig:akses-dashboard-mobile}
\end{figure}

\begin{figure}[ht]
  \centering
  \includegraphics[width=0.8\textwidth]{resources/chapter-4/pengujian/pengujian-nonfungsional-2.jpg}
  \caption{Mengakses \textit{Dashboard} Melalui Chromium \textit{Based} Browser}
  \label{fig:akses-dashboard-chromium}
\end{figure}

\begin{figure}[ht]
  \centering
  \includegraphics[width=0.8\textwidth]{resources/chapter-4/pengujian/pengujian-nonfungsional-3.jpg}
  \caption{Mengakses \textit{Dashboard} Melalui Safari Browser}
  \label{fig:akses-dashboard-safari}
\end{figure}

\begin{figure}[ht]
  \centering
  \includegraphics[width=0.8\textwidth]{resources/chapter-4/pengujian/pengujian-nonfungsional-4.jpg}
  \caption{Mengakses user \textit{service} tanpa kredensial}
  \label{fig:akses-service-user}
\end{figure}

\begin{figure}[ht]
  \centering
  \includegraphics[width=0.8\textwidth]{resources/chapter-4/pengujian/pengujian-nonfungsional-5.jpg}
  \caption{Mengakses admin \textit{service} tanpa kredensial}
  \label{fig:akses-service-admin}
\end{figure}

\begin{figure}[ht]
  \centering
  \includegraphics[width=0.8\textwidth]{resources/chapter-4/pengujian/pengujian-nonfungsional-6.jpg}
  \caption{Mengakses \textit{service} kubernetes tanpa kredensial}
  \label{fig:akses-service-kubernetes}
\end{figure}

\end{document}
